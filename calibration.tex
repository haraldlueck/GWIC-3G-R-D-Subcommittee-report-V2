%\chapterimage{Figures/balance.jpg} % Chapter heading image
% https://upload.wikimedia.org/wikipedia/commons/2/27/Bascula_9.jpg
\chapterimage{Figures1_3/urmeter1_3.jpg} % Chapter heading image
%https://www.lichtmikroskop.net/elektronenmikroskop/bilder/urmeter.jpg
\chapter{Calibration}
%\section{\ac{RaD}   misc.}
\label{sec:Calibration}
%\vspace{-1.8cm}

%Contacts (James Lough <james.lough@aei.mpg.de>, Yuki Inoue <iyuki@ncu.edu.tw>)

Extracting new science from the observed gravitational waves requires accurate knowledge of the amplitude and timing of the signals. With the very high \ac{SNR}, up to 1000, expected in the \ac{3G}   era, extremely low calibration uncertainty will be necessary for observations to be noise limited.

\section{Science-Driven Calibration Requirements}
The exact calibration requirements for the scientific objectives of \ac{3G}   gravitational-wave observatories are not yet known, but can be estimated. There are two aspects of calibration uncertainty, the absolute uncertainty and the relative uncertainty. The first tells us how well we understand the total calibration in absolute numbers, while the latter is a frequency-dependent calibration uncertainty with respect to some fixed reference frequency. 
Looking for deviations from a modeled waveform template, we are mainly concerned with relative calibration uncertainty. In order to estimate distances to the sources, we are most interested in absolute calibration.
In addition to the calibration error of the detector instrument, the waveform models have an uncertainty.
Studies are underway to determine the exact calibration requirements integrating both detector and waveform uncertainties.

The third generation science that will set our calibration requirements include \ac{BNS} tidal deformation, deviations from \ac{GR}, and measurements of the Hubble constant. The first two look for deviations from a modeled waveform template, while the latter is based on absolute distance measurements. With $\mathcal{O}(1000)$ low-redshift binary neutron-star events, the Hubble constant may be determined to a ${\sim}1\%$ level with Einstein Telescope~\cite{Cai:2016sby}.
This would require a sub-percent systematic absolute amplitude calibration.
The signals with the highest \acp{SNR} give the narrowest limits for \ac{GR}. With a network of three \ac{3G}   detectors we expect about one event per year with $\text{\ac{SNR}} \sim 1000$. To avoid being dominated by calibration errors, we need a calibration to an amplitude of <\,0.5\% and
about 0.1 Radian in phase integrated over 10\,Hz to 400\,Hz \cite{SathyaPers2019}.

%From a single binary neutron-star coalescence, the Hubble constant was determined to a fractional uncertainty of ${\sim}15\%$~\cite{Abbott:2017xzu}.
%If third-generation detectors are able to acquire $10^4$ events of similar quality, the Hubble constant could be determined to ${\sim}0.15\%$---a factor of $\sqrt{10^4}$ better.
%This would necessitate a systematic uncertainty in the calibration that is better than $0.1\%$ in order for the measurement to not be calibration-limited.

\section{State of the Art}
Overall calibration relies on both technologies and methods of measuring the detector’s control systems. The development of technologies important for calibration are discussed in the following subsections. The state of the art in calibration methods involve the use of \ac{MCMC} simulation to determine unknown systematic uncertainties.
\pagebreak
%\cite{CalUncert2017}.
The detector's control systems (sensing and actuation) have a considerable influence on the total calibration accuracy and therefore must be precisely characterized for the calibration of each gravitational wave detector.
Calibration lines are applied using the detector's length actuators in order to compute time dependent correction factors.
For Advanced LIGO, this process results in an absolute calibration uncertainty of a few percent in amplitude and a few degrees in phase across the majority of the frequency band~\cite{PhysRevD.96.102001}.
% have to be monitored online using  calibration lines.

\subsection{Photon Calibrators}
Photon calibrators provide a calibration reference starting from a traceable reference power meter. They use an amplitude modulated laser to apply a photon-recoil force to the test mass. For all interferometric gravitational-wave detectors with arm cavities, this is the current method of choice for absolute calibration.
%The calibration process involves periodic characterizations of the detectors' strain responses across the entire frequency band, along with continuous monitoring at a few select frequencies to track temporal variations in these responses.
%For Advanced LIGO, this process results in an absolute calibration uncertainty of a few percent in amplitude and a few degrees in phase across the majority of the frequency band~\cite{PhysRevD.96.102001}.
Ultimately, the current photon calibrator implementation in Advanced LIGO has an absolute systematic uncertainty of 0.5\%, set mainly by a combination of uncertainty in the calibration of the \ac{NIST}-traceable power standard and uncertainty in where the photon-calibrator beams and the interferometer beams impinge on the test masses~\cite{ALIGOPhotCalib2016, NISTWorkshop2019}. The national metrology institutes are improving the primary laser power calibration standards. On request of \ac{LIGO}, \ac{NIST} has improved the laser power standard from 0.44\% error to 0.31\% and is envisioning a level of 0.05\% in the next few years.
%Currently, these calibrators' reference uncertainty is roughly at the 1\% level. This was specifically stated at 0.76\% for the O2 calibration uncertainty for the advanced LIGO detectors.\magentacomment{citation needed} This resulted in a frequency dependent uncertainty between 2\% and 4\% for O2\magentacomment{How does an absolute calibration uncertainty of 0.76\% result in a relative uncertainty of 2\% - 4\%? ,citation needed}. With improvements between O2 and O3 the expectation is that this could be reduced to \magentacomment{a relative uncertainty? of} 0.3\%.\magentacomment{citation needed.} 

\subsection{Newtonian Calibrators}
Newtonian calibrators rely on the Newtonian gravitational interaction between an interferometer test mass and a known arrangement of rapidly rotating calibration masses, with the arrangement often approximating a dipole, hexapole, or other multipole distribution~\cite{Matone:2007vk,Inoue:2018okt}.
These devices have absolute systematic uncertainties related to how well the geometry of the calibrator, and its distance and orientation with respect to the test mass, can be characterized.
A Virgo prototype~\cite{0264-9381-35-23-235009} has already been deployed, and an improved prototype may achieve a 1\% systematic uncertainty.
\ac{KAGRA} is designing a dual-mass-distribution calibrator that, when combined with a photon calibrator, may achieve an absolute uncertainty of 0.17\%~\cite{PhysRevD.98.022005}.
A Newtonian calibrator prototype is also being developed for Advanced LIGO.
These Newtonian calibrator technologies are still in the very early stages of development, but there is significant effort in this direction.
%advanced LIGO, advanced Virgo, and KAGRA are all developing Newtonian calibrator technologies.\cite{PhysRevD.98.022005,}

%The design for the {\bf KAGRA} gravity field calibrator incorporates two distinct mass distributions in one rotating mass. One quadrupole and one hexapole, which provide a sort of self calibration of the distance to the test mass.\cite{PhysRevD.98.022005}\par
%{\bf Advanced LIGO} is also working on a Newtonian calibrator for confirmation of the photon calibrator in a similar way to KAGRA.\magentacomment{citation needed} \par
%{\bf Advanced Virgo} has recently published \cite{0264-9381-35-23-235009}  first tests of such a system and observed less than 1\% statistical uncertainty at some frequencies.\par

\subsection{Other Calibration Methods}
Laser frequency can also be used as a reference against which to calibrate an interferometer \cite{Leong2012, PhysRevD.95.062003}. Further \ac{RaD}   is required to cut down uncertainties from current levels of about 10\% (in Advanced LIGO) to the requirements of \ac{3G}   observatories.
It is also possible to fix certain aspects of the calibration using an astrophysical source whose properties are sufficiently well known~\cite{CalibrationGW170817,Pitkin:2015kgm}; more work is needed to understand how this technique will inform the calibration for \ac{3G}   detectors.
%Laser frequency can also be used as a reference against which to calibrate an interferometer.For detectors without arm cavities (such as GEO 600), test mass motion can be calibrated directly into fractions of a Michelson fringe, and thereby laser wavelength. \greencomment{Dave: But all \ac{3G}   detectors are planned to have arm cavities, so is this statement relevant?}This can result in test mass actuator calibrations with statistical uncertainties of less than 0.1\%~\cite{Leong2012}.For present and future detectors with arm cavities, this method requires multiple steps and hence introduces more error.Multicolor techniques are also possible~\cite{PhysRevD.95.062003}, where laser light from an arm cavity is interfered with a reference laser, yielding a beat note whose frequency fluctuations can be read out directly.
%Another calibration method would be to use frequencies as a reference. We can measure frequencies much more precisely than amplitudes. Free swinging Michelson is an example of this technique, though with arm cavities this method has to be done in steps and involves actuating the ITMs through the PUM stage which is weak and introduces more error than what is achievable with PCAL. 

%The arm cavity locking method outlined in the LIGO calibration for GW150914 paper is another example.
%\cite{PhysRevD.95.062003}
%This uses a frequency reference which in this case would be an RF LO to measure the beat frequency between the green arm locking signal and the frequency doubled infrared light.

\section{Outlook and Recommendations}
%\section{Roadmap and timeline}

Requirements on the absolute and relative calibration for third-generation detectors are not yet fully known, and dedicated modeling efforts are needed to understand the constraints set by various \ac{3G}   scientific objectives. \ac{3G}   detectors will likely require much better than sub-percent amplitude calibration, potentially necessitating \ac{RaD}   to improve the capabilities of the current calibration apparatuses; such \ac{RaD}   is already underway and cooperation activities between the different gravitational-wave collaborations are considered to be at an adequate and sufficient level.
The exact nature and combination of calibration methods to be used in \ac{3G}   can be designed soon, but left flexible enough such that the best performing technologies can be installed in the instruments with only a few years lead time before operations.