\chapterimage{Figures/Cryo1.jpg} % Chapter heading image
% frozen baltic sea @ Stralsund copyright Arne Lueck
\chapter{Cryogenics}
%\section{Cryogenics}
\label{sec:Cryogenics}
%\vspace{-4cm} %to get the table onto the first page. Needs to be adapted in final layout

%\magentacomment{hal: This chapter needs a complete overhaul. The section" Cryogenics below 20\,K feels like an appendix; which it probably is. If this is deemed an important technique for the future it should go into the main part, not after the requirements and pathways.DEM:  I think it is redeemable! }

% A common feature of most designs for future interferometers, is the cryogenic\footnote{Here we use the term cryogenic to refer to temperatures significantly below room temperature, including 123\,K.} operation of the test mass mirrors~\cite{ET2011, ISWP:2018} and suspensions.
To reduce thermal noise (see Box~\ref{Box:Thermal}), the Einstein Telescope~\cite{ET2011}, Voyager~\cite{Voyager:Inst} and Cosmic Explorer~\cite{CosmicExplorer2017} (CE2) are designed to operate with their mirrors and suspensions at cryogenic\footnote{Here we use the term cryogenic to refer to temperatures significantly below room temperature, including 123\,K.} temperatures (see Table~\ref{tab:CryoTemps}).

\begin{table}[h]
\centering
%\begin{tabular}{ |p{3cm}||p{3cm}|p{3cm}|p{3cm}|  }
\begin{tabular}{ |l||l|l|l|  }
 %\hline
 %\multicolumn{4}{|c|}{Cryogenic interferometers} \\
 \hline
 Interferometer & Mirror Temperature [K] & Mirror Material & Suspension \\
 \hline
 CLIO           &   20\,K           & glass     &  steel wires \\
 KAGRA          &   20\,K           & sapphire  &  sapphire fibers  \\
 ET-LF             &   20 or 123\,K    & silicon   &  silicon  \\
 CE             &   123\,K          & silicon   &  silicon ribbons \\
 Voyager        &   123\,K          & silicon   &  silicon ribbons \\
 \hline
\end{tabular}
\caption[Cryo IFOs]{Parameters of the cryogenic interferometers}
\label{tab:CryoTemps}
\end{table}

Gravitational-wave physics has a history of using cryogenics to improve the sensitivity of detectors, starting with resonant mass (bar) detectors~\cite{ColdBars}. Currently, the first cryogenic laser interferometers, the 100\,m prototype detector CLIO~\cite{CLIO:2008} and 3\,km gravitational-wave detector KAGRA~\cite{KAGRA2013}, both in Japan, are in operation and testing the cryogenic performance of mirrors and suspensions. 
%In the gravitational-wave field there is a long history of using cryogenics for improving the sensitivity of the resonant mass (bar) detectors~\cite{ColdBars}.
% The vibration noise requirements for cryogenic interferometers are much more demanding than they had been for resonant mass detectors.
%More recently, the 100\,m prototype detector CLIO~\cite{CLIO:2008} and the 3\,km GW detector KAGRA~\cite{KAGRA2013} in Japan are investigating the use of cryogenic mirrors and suspensions for laser interferometers.
In CLIO, the mirrors were successfully cooled to cryogenic temperatures without observing significant additional noise, indicating that active cryo-coolers are a viable candidate for use in future interferometers. KAGRA will be the first cryogenic interferometer large and sensitive enough to detect gravitational waves. It will thus provide a valuable test bed for uncovering the important unknown problems with sensitive cryogenic interferometery, a key step toward improving future cryogenic detectors. Several other laboratories worldwide operate cryogenic optical cavities for precision measurements~\cite{Mueller:03}, atomic clocks~\cite{JunYeGroup:2019} and tests of quantum mechanics~\cite{CaltechIQIM} and are valuable partners for the development of technologies in cryo-cooling and low noise cryostats. 
%Although low vibration cryo-technology was pioneered by the resonant mass detector groups, the optical cryostats pose the new challenge of low \emph{phase noise} measurements in a cryostat.\\
%The 3G detectors: Einstein Telescope~\cite{Sathyaprakash:2012jt} and Cosmic Explorer~\cite{CosmicExplorer2017}, and also Voyager~\cite{Voyager:Inst}, are all being designed to use cryogenics for their optics and suspensions.



%\begin{tcolorbox}[standard jigsaw,colframe=ocre,colback=blanchedalmond!10!white,opacityback=0.6,coltext=black]
%\paragraph{Thermal Noise}
%The primary benefit of cryogenic operation of the mirrors and suspensions is reduced thermal noise: Brownian noise of the mirror suspensions, substrates and coatings and thermo-optic (thermo-elastic plus thermo-refractive) noise of substrates and coatings. The relation between the dissipation and the power spectrum of the noise is described by Callen's Fluctuation-Dissipation Theorem~\cite{CaWe1951, Kubo:FDT, Callen:1959} and is given by:
%\begin{equation}
%S_x(f) = \frac{k_B T}{\pi^2 f^2} \left| Re \big[ Y(f) \big]\right|.
%\label{eq:FDT}
%\end{equation}
%The displacement noise, $x(f)$, scales as $\sqrt{T}$. More significantly, many material properties of mirror substrates (e.g., sapphire and silicon) and coatings (e.g., GaAs, GaP, $\alpha$-Si) scale favorably with decreasing temperature. This makes the improvement in the noise substantially larger than the naive $\sqrt{T}$ scaling.\\
%\magentacomment{hal: I suggest moving this to the introduction and expanding a little to also cover coating, core optics and suspension thermal noise}
%\end{tcolorbox}

% \paragraph{Robust Operation}
In addition to thermal noise improvement, there are a number of possible operational advantages due to a low a low temperature environment:
\begin{enumerate}
\item Increased thermal conductivity in crystalline substrates and at cryogenic temperatures; this dramatically reduces the thermal gradients in the mirror, and thereby, the induced wavefront distortions due to thermo-elastic deformations of the mirror surface and thermo-refractive lensing in the substrate bulk.
%\magentacomment{hal: Does dn/dT not also go down considerably at cryogenic temperatures?}
%\greencomment{Rana: Not really; it has a weak T dependence until very low T}
%\magentacomment{Depends on what you call weak: Appl. Phys. Lett. 101, 041905 (2012) and https://ntrs.nasa.gov/search.jsp?R=20070021411 2019-01-08T12:51:25+00:00Z}
\item Zero thermal expansion in silicon at 18 and 123\,K~\cite{Touloukian_Brett6,Wiens:14}. Not only does this suppress thermo-elastic noise for stress-free structures, but it also reduces greatly the thermally induced thermo-elastic deformation of the mirror surface from any residual temperature gradients.
\item Cryo-pumping of the residual gas by the cold (colder than the mirrors) shields surrounding the mirror reduces the gas pressure and thereby the presence of squeeze-film damping~\cite{Cavalleri:09,Bao:07}.
\item As temperature approaches 0\,K, all thermally induced parameter fluctuations (e.g. thermo-elastic, thermo-refractive, etc.) tend to zero, as a consequence of the Nernst theorem.
\item Electronic noise of sensors and actuators used to control the mirrors can be reduced by low temperature operation.
\end{enumerate}
 


%\section{Current State of the Art}
%In the gravitational-wave field there is a long history of using cryogenics for improving the sensitivity of the resonant mass (bar) detectors\cite{ColdBars}.
%In recent years, the 100\,m prototype detector, CLIO~\cite{CLIO:2008}, and the 3\,km GW detector, KAGRA\cite{KAGRA2013}, have investigated the idea of making cryogenic mirrors and suspensions for laser interferometers.
%In the CLIO, case, the cryogenic operation was done successfully: the mirrors became cold and there was no major introduction of noise. In this way, it was shown that active cryo-coolers are a possible candidate for use in future interferometers. KAGRA will be the first cryogenic interferometer operating at astrophysically interesting sensitivity; many of the import unknown problems with cryogenic interferometry should be exposed there. For those designing future cryogenic interferometers, this would be a valuable learning environment.
%In addition, there are several laboraties around the world which have been operating cryogenic optical cavities for precision measurement\cite{Holger:Munich}, atomic clocks\cite{JunYegroup}, and tests of quantum mechanics\cite{CaltechIQIM}. These prototypes are valuable in developing the technologies: cryo-cooling, NIR opto-electronics, low noise cryostats. Although low vibration cryo-technology has been pioneered by the resonant mass detector groups, the optical cryostats pose the new challenge of low \emph{phase noise} measurements in a cryostat.

\section{Requirements}
A sketch of possible requirements for operating future detectors with their mirrors and the final stage of their suspensions at cryogenic temperatures are as follows. 
% For the future 3G GW detectors (whether in the existing facilities or in new, larger facilities), it is envisaged that we will operate the mirrors and the final stage of the mirror suspensions at low temperatures.
% The major requirements are the following:
\begin{enumerate}
\item Maintain the target cryogenic temperature with sufficient accuracy:
      \begin{itemize}
        \item for silicon at 123\,K, and 18\,K, the fluctuations must be kept below 0.1\,K
        \item for sapphire at $\sim$20\,K, the requirement is less stringent since there is no null in the material properties to reach
        \item the cryocooling mechanism must be able to compensate for the power absorbed by the mirror as well as the laser power scattered by the mirror into wide angles. In a single interferometer design with typical 3G arm powers of around 3\,MW, and typical values of absorption (1\,ppm) and scattering (30\,ppm) this implies 3\,W of cooling for the test mass and 100\,W of cooling for the baffles. In ET's xylophone design, the injected power in the cold interferometer is very low and the cooling power is less critical. 
      \end{itemize}

\item The time required to cool from room temperature to cryogenic operation must be fast enough to avoid delays in operations. Currently, the cooling time in KAGRA is about one month. For 3G detectors, the goal is to have the cooldown time be less than the vacuum pumpdown time, which is expected to be about one week.  

% The cooldown from room temperature must be fast enough to minimize delays in commissioning time (i.e., less than the vacuum pumpdown time).  Currently the cooling time in KAGRA is about one month. The target should be to reduce that time to less than the evacuation time, considered about one week.     

\item Steady-state cooling should be accomplished with minimal disturbance:
      \begin{itemize}
        \item the surrounding cold shields must be designed to have low backscatter (into the main cavity mode) and low vibration (relative to the mirror), such that the combined amplitude/phase noise does not degrade the sensitivity of 3G detectors
        % is below the quantum noise level achieved with squeezing
        \item in designs where the suspension fibers/ribbons are used for the steady-state cryocooling, the suspension thermal noise, see Chapter~\ref{sec:Suspensions_Isolation}, must not degrade the sensitivity of 3G detectors,
        \item auxiliary cold links must be designed so as to not produce seismic shorts of the suspension system nor increase the overall suspension thermal noise;
        \item the required cryogenic machinery must be quiet or isolated enough to not increase the acoustic and gravity gradient noise.
      \end{itemize}
\end{enumerate}

% --- roughly the same for initial/future, so lets leave it out
% \subsection{3G initial}
% 100-200 kg mirrors.
% cold suspension ribbons/fibers.
% Shield from outside heat load.
% Absorb O(10 W) of heat from mirror scattered light,
% whilst avoiding backscatter (currently a limit in Advanced LIGO).
%
% \subsection{future}
% Mirrors of around 1000kg, and 60 cm diameter.
% How to produce such large optics of sufficient quality?
% Issues with the annealing / preparation.
% Large thermal gradients during cooldown?


\section{Required facilities and collaborations}

The successful operation of cryogenics in 3G instruments relies on continued R\&D
to meet the above requirements, while also demonstrating the practicality of integration in gravitational-wave detectors. Specifically, methods for reliable and fast heat extraction and ways to achieve the required temperature stability must be demonstrated. Cryogenic systems must be integrated with seismic isolation, suspensions, and vacuum systems.  
Other subsystems, such as auxiliary optics for wavefront control, must be modified to take into account the changed material and optical properties at cryogenic temperatures. Silent and vibration-free refrigeration machinery (e.g., using PT cryocoolers with symmetric cold heads) need to be designed and constructed (collaboratively with industry and the high energy physics community). Furthermore, the use of cryogenics in the electronics and magnetic actuators~\cite{cryo:OSEM} of gravitational-wave detectors has potential great benefit and should be explored. In all of these endeavors, cooperation with industry and high-energy particle physics, where comparable requirements are addressed, could create valuable synergies. 

% R\&D toward the design and construction of silent and vibration-free refrigerator machinery is required.  Several ideas have been proposed, for example the use of PT cryocoolers with symmetric cold heads. Collaboration with cryogenics industry and the accelerator community is strongly recommended. 

This R\&D will be carried out in tabletop experiments, operating detectors, upgrades to detectors, and prototype systems. KAGRA will provide an outstanding demonstration of many of these considerations, for sapphire core optics at 20\,K. However, it will not have as stringent requirements as 3G detectors.
% For sapphire core optics at 20\,K,   Items 1 and 3 have been/will be demonstrated using KAGRA.  As the current sensitivity goal of KAGRA is modest at low frequencies, it is not clear how well KAGRA performance will inform item 2 and 4.  May occur with KAGRA upgrade.
A number of facilities are planned or operational to test silicon core optics at 18\,K and 123\,K. 
% : Items 1-3:  a number of facilities are planned or under construction.  
Existing facilities at Stanford, Caltech and Gingin are currently being reconfigured for such tests. A new facility, ET Pathfinder, in Maastricht, the Netherlands, has been designed with an L-shaped vacuum system that will investigate both 123\,K and  18\,K, separately in each arm. Much of the technology required for cryogenic operation will also be tested in the near future in experiments exploring the material properties of mirrors, coatings, suspensions and in-vacuum materials. Tight coordination between all these groups is recommended to learn as much as possible about cryogenic operations to ensure rapid progress and to focus efforts on the most important open questions.

Ultimately, full system integration tests must be done in present multi-km facilities or large, sensitive prototypes to demonstrate sustained high performance in situations that are reliably scalable to 3G systems. New prototypes, such as ET Pathfinder, are very valuable, but are also large endeavors that require global coordination for the necessary resources and execution. Similarly, large investments in low temperature operation in 2G facilities, such as the Voyager concept, would not only demonstrate the technical readiness, but might also serve to expand the astronomical scope of the global gravitational wave observation network. 


% Ultimately, full system integration tests need to be done on large, sensitive, prototypes, demonstrating sustained high performance. The prototypes need to be large enough to allow reliable scaling to full size 3G systems.  \textit{\textbf{This is a major undertaking and global collaboration is needed to resource, develop and manage these prototypes}}.  These could be new facilities, or,  potentially, an existing 4\,km facility(ies) could be re-purposed.  This would not only demonstrate technical readiness but would produce detector(s) with a significantly extended range.  This is effectively the Voyager concept.

Cryogenic interferometers can be operated with a variety of temperatures, materials, and wavelengths, each with interdependence and implications for the facilities design. Current plans call for initial Cosmic Explorer, CE1, to operate at room temperature with fused silica optics. Its major upgrade, CE2, would operate at 123\,K with silicon, technologies that the Voyager concept would test and exploit. Einstein Telescope is designed to operate at 18\,K with silicon. However, there is flexibility in these plans to account for technological readiness and feasibility. Thus, planning must become more solid in the coming years to allow timely 3G operations. 

\section{Outlook}
Building on this document and on the growing experience gained with KAGRA and other experiments, we recommend that the 3G community develop a realistic roadmap for 3G cryogenic operation over the next 2-5 years, focusing on the best candidate technologies and materials to be moved to the next testing phase. For this planning, cooperation with industry and international laboratories where cryogenics are routinely used, such as CERN, is strongly recommended.

% Note that it is feasible that both Silicon 123\,K and Silicon 18\,K cryogenic subsystems will be deployed at different 3G facilities.  Depending on facility finding and  technology readiness, it is  feasible, perhaps even likely,  that 3G facilities will initially operate room temperature detectors whilst cryogenic development continues.

% \magentacomment{dem: cross referencing to suspension and isolation}

%\textit{Sapphire at 20\,K}


%\begin{itemize}
%\item  2019-2026  	Use KAGRA to diagnose  feasibility and performance.
%\item  2027  		First Cryogenic downselect.  
%\item  2028-2032  		If GO:  final design and fabrication.
%\end{itemize}



%\textit{Silicon at 	123\,K}
%•  \begin{itemize}
%\item 2019-2025  	R\&D in university labs on 1. and 3.
%\item 2022-2028	Integration testing with SIS and other subsystems  %(item 2).
%\item 2028……  	Cryogenic downselect
%\item 2029-2033	Large  Phase noise prototype
%\item 2034-2039		If GO:  final design and fabrication.
%\end{itemize}



%\textit{Silicon at 	18\,K}
%\begin{itemize}
%\item SAME CYCLE AS SILICON AT 123\,K.
%\end{itemize}


%The development of the cryogenics techniques will need dedicated
%testbeds. In the initial stages, these may be non-suspended masses where thermal control and cryo-cooling ideas can be developed.

%To test the low noise aspect of the cryocooling techniques, suspended
%interferometers must be used. In a medium scale  interferometer, one can test the near-complete cryocooling system and conduct integration testing of the overall design (laser, optics, opto-electronics).

%A major challenge is the low-noise operation of the GW detectors in the low (<20\,Hz) frequency band. This is not only due to the obvious problems of seismic and gravitational noise, but also to the dynamic range required for the control system.
%Since the mirrors are moving by a significant fraction of the laser wavelength at 0.01\,--\,0.1\,Hz, it is difficult to control that motion without introducing noise at the $10^{-18}$\,m level at $\sim5$\,--\,10\,Hz. Ultimately, phase noise prototype interferometers for 124K and below 20K must be built.   

%Such suspended prototypes should be enough to give confidence in the full design,
%but should not be designed for the full displacement/phase sensitivity
%Prototype systems at the scale of the GEO\,600, AEI, MIT, and Caltech interferometers
%would be appropriate for these purposes along with lesson learned from KAGRA operation


%\section{Type of collaboration required:  small/large}
%\begin{enumerate}
%\item
%\end{enumerate}
%\section{Suggested mechanisms}

%\section{Impact/relation to 2G and upgrades}

%\section{Cryogenics below 20\,K}
%\label{sec:cryo20}
%Below $\sim20$\,K most thermal noise sources are so small that they are negligible compared to other noise sources (e.g. quantum noise).
%The main motivations in going below 20\,K is to take advantage of the possible dramatic change in material properties (e.g. superconductivity) for the mirrors and suspensions, but also for the mirror sensors and actuators.





% \subsection{Initial Cooldown}
% The technological effort to cool the mirrors down to
% temperatures around 4\,K is not so different from that used to operate the interferometer at 20\,K.
% The engineering plant is  almost identical both if we use cryogenic fluids or commercial cryo-coolers.
% The main difference is in the improvement of the heat extraction from the payload and the reduction of the thermal inputs.
% It follows that the main issue is the optimization of the %\magentacomment{hal: not understandable as written}
% thermal links  used to transmit the refrigeration power and the fibers for the heat extraction from the mirror.


% The control of test masses plays a crucial role in gravitational wave detectors with independent test masses.
% Payload local control system is meant to slow down and align
% the mirrors, driving their motion within the range of interferometric error signals.
% In order to preserve the quasi-inertial state guaranteed at the level of payload suspension point by the seismic attenuator system, only internal forces
% %\magentacomment{hal: internal to?}
% should be used.
% The digital/analog control system needs to be improved; namely by reducing by more than 10x the overall low frequency noise re-injection.
% The related R\&D will naturally follow the developments of advanced detector implementations. The task will be complicated by the presence of cryostats surrounding the vacuum chambers hosting the payloads and the use of extra viewports should be avoided to limit the thermal input.
% As consequence the sensing access to the mirror position without significant perturbation of cryostat performance in mirror cooling.
% This can be done by an extensive use of optical fiber sensors or even inventing new capacitive detectors with SQUID pre-amplifiers from which an almost noiseless error signal can be extracted to drive the actuators.
% This last element is crucial for applying forces on the different  stages of the payload. The actuators acting on the marionette and the reaction mass are designed to allow active control of the locking and alignment during operation.
% %\magentacomment{hal:Should this go into the control section?}
% Small and fast corrections of the mirror position can be obtained, through the marionette, if the mechanical transfer function of the system is taken into account.
% For example, due to the response of pendulum mechanical filters, the displacement amplitude of the mirror face with respect to that applied to the marionette arms decreases with the frequency.
% In the advanced detector configuration  magnets are placed on the marionette arms and coils  are set in front of each of them. Through these actuators, it is possible to steer, around the directions perpendicular  to the line  joining the magnet pairs.
% In order to design the coil - magnet actuators for the marionette and the mirror of the LF interferometer, several constraints must be taken into account.
% We should consider the effect of the magnetic noise produced by the magnet at 4.2 K,  its magnetization change due to the cooling, the current noise of the coil and the power dissipated by the current flowing in the coil. The use of superconducting wires for the coils is the obvious solution to kill the last contribution, while for the magnet itself it has been demonstrated that with a suitable material choice the Barkausen noise can be kept  well below the threshold set by the low frequency sensitivity of a 3G detector\cite{cryo:OSEM}.
%
% In this temperature range it is possible also to develop new push-pull actuators based on Meissner effect, with the great advantage to replace the magnets glued on the mirror or attached to the marionette arms with superconducting thin films coated on the surfaces.


% \paragraph{R\&D for lowering the temperature}
%
% The optimization of the mirror suspension fibers is the main action to
% be pursued in order to keep the mirror at temperature below 20\,K.
% This implies to explore new geometric configurations and new materials
% being constrained by the need  to have a low suspension thermal noise and
% at the same time an efficient path to extract the heat power from the mirror.
%
% In principle a 4\,K cryostat is not  different from that for 20\,K.  In fact, the final temperature of a double stage cryocooler is in the 4\,K  range, so   this device  can be still used to  achieve this temperature.  It is  far obvious \magentacomment{hal:reword}  that, to get a lower temperature we should reduce the thermal input and/or  use more refrigeration power. An increased number of cry-cooler can imply higher vibration.  In this respect R\&D devoted to design and construct  silent refrigerator machine is highly desirable.
% Several ideas have been proposed, for example the use of PT cryocooler, with symmetric cold heads and driven at the same helium wave frequency in  opposite phase. However, these devices are not commercial yet and R\&D carried on in collaboration with a cryo-industry is highly recommended.
%  In order to increase the cooling efficiency and to get a further reduction of the thermal noise contribution, the most powerful approach is to make use \magentacomment{hal:how?} of the liquid helium II (He II), which has great advantages. It limits the vibration noise associated to the other cryogenic fluids  and it provides a powerful way for extracting the heat from the mirrors. Modern large engineering projects for high-energy physics require thermostatic control of working components at the level of 1.8 - 2 K and are constructed with lengths of channels containing He II. The uniqueness of He II is that it contains a superfluid component with zero entropy, which moves through other liquids and solids with zero friction to an extent dependent on the temperature of the liquid. He II is a liquid of extremely low viscosity and very high heat capacity, which prevents small transient temperature fluctuations. Moreover, thanks to its very high thermal conductivity is able to conduct away heat a thousand times better than any metallic conductor like copper.
%
% In He II, the heat from a hot surface is carried away by the superfluid component, so in any design with complicated geometry and helium flows, the entire heat load acts on the phase interface. The boiling mechanism involves evaporation from surfaces ad in a \magentacomment{hal:???}  flow of ordinary boiling liquid, the heat influx is uniformly distributed in unit volume of the two-phase mixture. In stratified He II, the heat influx is associated with the interface between the phases, so the He II evaporation rate is increased by a substantial factor. A major feature of boiling in He II is that the evaporation of the superfluid component predominates. The heat load is transported by convection in the superfluid component, and this consequently evaporates more rapidly than does the normal component.
% When a two-phase flow of He II moves in a heated channel, a droplet structure or mist is formed in the vapor space as the amount of liquid in the stratified flow decreases. In a stratified flow of an ordinary liquid in a large- diameter tube, an increase in the bulk vapor content leads to the vapor becoming superheated and the liquid evaporating completely. In He II one prevents the vapor becoming superheated by encouraging the spontaneous formation of a droplet structure with a large heat-transfer surface, which provides a constant temperature over the channel cross section.
% An efficient and quiet configuration for cooling the mirror by He II is the bain de Claudet. Here the idea is to provide superfluid helium at atmospheric pressure and to insure continuous refilling from the container of the helium in the normal state. In this way the He II bath is kept in a quiet hydrodynamic status well far from the boiling point . In this case the cold box on top of the seismic attenuator , ancillary to the main one  to which a the mirror is suspended, is an heat exchange filled by superfluid helium at atmospheric pressure and operating in the stationary condition of almost zero mass flow.
%\magentacomment{hal:a sketch may be helpful, although the level of technical detail is already too high. Can we point to some reference and dramatically shorten the subsection?}
