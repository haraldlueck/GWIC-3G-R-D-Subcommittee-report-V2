\chapterimage{Figures/Virgotube.jpg} % Chapter heading image
% Virgo beam tube copyright Virgo/EGO
\chapter{Facilities and Infrastructures}
%\section{Facilities and Infrastructures}
\label{sec:Fac_Inf}
The characteristics of the site location and of the infrastructure directly impacts the final sensitivity and achievable observation time of the detector. Ground vibrations not only affect the sensitivity directly, through seismic and Newtonian noise couplings, but 
% may also do it indirectly 
also indirectly though additional scattered light noise and 
%making the overall control more difficult).
complicating interferometer control.
Therefore, local noise (of natural or anthropic origin) is one of the key parameters for the evaluation of a candidate site.  Construction of underground detectors (such as ET) must consider the nature of the rock, homogeneity over the arm lengths, abundance of water, long term geological stability both as cost drivers and overall site quietness. 
%are aspects which can impact the infrastructure cost and stability. These aspects are particularly relevant for underground infrastructure such as the one planned for hosting the ET detector.

%Optimizing the site facility and infrastructure's influence on  detector performance is only one aspect of the R\&D work required.  
In addition to the of site, the design of the 
Civil facilities and vacuum systems are likely to dominate the total project construction costs.  Every effort to minimize costs, cost %uncertainties, 
contingencies, and other collateral impacts on society in these domains will reap rewards,  potentially pivotal, in approval and sponsorship. 

\section{High Level Design Considerations}
\label{Req:Fac_Inf}
%Quantitative requirements and figures-of-merit for site selection and infrastructure implementation still need to be worked out, but a 
A high-level list of considerations includes aspects such as:
\begin{itemize}
\item intensity of seismic noise (on surface and underground)
\item surface meteorological conditions (wind, rain)
\item anthropogenic noise (population density, presence of industrial activity)
\item geological stability
\item earthquake history
\item type of the underground rock (for tunnel construction feasibility and cost) 
\item underground water abundance
\item orography
% Dave R adds:
\item regional and national permitting and environmental clearances
\item site levelness (for above ground detectors)
\item proximity to urban centers (for live-ability)
\end{itemize}
\newpage
For underground detectors, the design of the %caverns
experimental halls %hosting the corners of the detector 
require a special attention. They have to be large enough %in order to ease 
to facility assembly and maintenance of detector components %equipment and operation  and, above all, not to prevent 
to accommodate future %evolution of the detector. 
upgrades. On the other hand, the larger the volume the larger the cost and the engineering challenge. Cavern volume and shape also contribute to the level of atmospheric Newtonian noise (Section \ref{sec:Newtonian_Noise}) which might limit the sensitivity.

For surface %concepts, 
detectors such as Cosmic Explorer, many of these same criteria are in effect; indeed, for a putative 40km laser-straight arm, the Earth’s sagitta is of order 30m, somewhat blurring the concept of “surface” construction. This approach does, however, promote surface topography and surface geology in their priorities as site criteria. The direct effect of wind on above-ground structures also joins its indirect influences on seismicity and gravitational gradients. Effects on the artificial and natural environment, flora and fauna are also concerns, and must be factored differently into site selection, project approval, and design. 

Advances in tunnel boring machines and in surface road and pipeline excavation have been seen to drastically reduce the cost per kilometer of public works structures over recent decades, e.g.,~\cite{BoringCompany}. Directed research into applying these modern methods to reduce the cost and collateral impact of projects such as ET and CE should be explored as a high priority. 
Attention must be also paid to the legal aspects specific to each candidate countries, which could impact the timing of the infrastructure realization. 

\subsection{3G initial/future}\magentacomment{hal: derive required action items (JRS: Subscale demos, current gen equipment noise surveys)}
\greencomment{Dave: I had to read this section before I understood what it was about.  It needs a few introductory sentences to make it clear  to the reader what this section is consider.}

I think this is better located at the end of the facilities section.  Also, if we become pushed to make cuts to save space; I think this section could be cut. 
As far as the civil infrastructure is concerned we cannot think of a phased approach. In the ET case the excavation of the underground infrastructure and the realization of the ground auxiliary works must be done once for all. It is a matter of realizing an expensive infrastructure, which must be working for decades and must therefore be stable  and capable of hosting the evolution of the detector.

With respect to vacuum systems, 1G/2G experience suggests different approaches. LIGO elected to bake out and qualify its beam tubes for asymptotic "2G" service (about $10^{-9}$ Torr $\rm H_2$ and $10^{-11} $ Torr $\rm H_2O$) immediately at inception, but invested in no durable means to repeat this in future; Virgo chose to provide a permanent bake capacity, but to delay the expense of exercising the bakeout until it is required for observing sensitivity. In the construction phase, the  KAGRA interferometer devoted  a special care to smooth   the inner surface of the tube via electropolishing, a costly approach that could fulfill  residual pressure requirements without baking the vacuum system  after assembly.  Each was a valid approach in its specific Project context; whether phased performance criteria (and some opportunity for cost savings) makes sense for 3G beam tubes will depend on many factors, including future operations funding and the future global observing plan.  


\section{Impact/relation to 2G and upgrades}

There is still room to\textbf{improve the sensitivity of the 2G detectors through incremental upgrades}. However, in the next years we expect to reach the "\textbf{infrastructure limit}". In other words, the level of environmental noise as well as the detector lengths and even the space available in the buildings will make it hardly possible or to expensive to improve further. The ET underground infrastructure is conceived to overcome such limits. \greencomment{Dave: I realize this may be a controversial opinion not shared by all, but an infrastructure that is limited to $L=10 km$ will require all the improvements to come from making better measurements of $Delta_L$.  That seems a harder road to me...} However, it is crucial to understand those limits and be sure that all the lessons have been learned. For instance, the quietness of an underground location can be spoiled by machine-induced noise. It is therefore very important to review the machine-induced noise in the current infrastructures and engineer the new site in order to minimize such disturbances. A joint effort based on the LIGO, Virgo and particularly KAGRA experience on this topic would be beneficial. 

In current practice, incremental upgrades of existing detectors like LIGO, GEO600  and Virgo have been an effective way to prove new technologies in context, at full sensitivity, while directly expanding the observing horizon. Enhanced LIGO, Advanced LIGO, Advanced Virgo, Virgo+ and A+ are examples of this effective combination of R\&D and practical deployment ``on the fly''  for improved observations.  

The incremental upgrade approach has limits, however.  One is the constraint to maintain compatibility with legacy systems and infrastructure; new topologies, footprints, and even wavelengths may be effectively off the table. Less obvious, but potentially more serious, is the growing imperative to minimize interruptions to observing. Upgrades require downtime.  The explosive discoveries of the last three years, particularly the multimessenger astronomy revolution triggered by GW170817,  have raised global desire to keep observing with existing 2G and 2G+ instruments. 

As a result, an increasing portion of 3G technology demonstration must rely on offline engineering development in subscale demonstrations, reprising the  development environment of initial LIGO and Virgo, before any large-scale testbed existed. 

\section{3G vacuum systems}
Vacuum systems for planned 3G detectors are likely to be the largest ultra-high vacuum systems built to date and will account for a significant part of the cost of building the observatories. Substantial innovation and research went into designing and building the LIGO, Virgo, GEO600 and KAGRA vacuum systems within economic constraints; the more stringent technical requirements and much greater size required for 3G vacuum envelopes threaten to render them infeasible without still further innovation. Many avenues for technical research have been proposed,  centered on themes of either improving an interferometer’s immunity to residual gas, or reducing the cost of suitable installations. Some questions that merit closer investigations include:

\begin{itemize}
\item  Are there economies in using materials other than stainless steel as the envelope material? What is really known about the vacuum properties of the inexpensive tubing (e.g., cold rolled steel) used for petroleum and gas transport? What are the vacuum experience and costs associated with aluminum alloys and plastics?

\item Are nested (e.g., differentially pumped) vacuum systems practical and would they offer cost advantages? For example, can an outer pressure vessel of inexpensive structural material protect a thin liner of UHV-compatible, bakeable material?

\item Can civil construction mass-production techniques, such as extrusion and spiral tube milling, be adapted to future UHV construction?

\item Are there newer mass production surface treatment and cleaning techniques that can be applied reduce outgassing? Is heating the best degassing method, or can UV or plasma excitation be cost-competitive at scale?

\item Are there ways to simultaneously meet surface outgassing and possibly distributed pumping together with other physical requirements of the system, such as (e.g. for gravitational wave applications) stray light attenuation, vibration damping and particulate mitigation?

\item Can optical pressure gauging and leak detection offer practical advantages for system construction, commissioning and maintenance?

\item What are the prospects for new getter materials and surface treatments in maintaining UHV conditions for very large systems with modest gas loads?

\item Are there new ideas for reliable, affordable large-aperture gate valves to isolate from atmospheric pressure during construction and service, and to isolate volumes with different requirements?

\item What are effective methods or surface treatments to minimize moisture adsorption during vented system service access? Are there efficient means to accelerate desorption during pumpdown and recovery?

\end{itemize}



\section{Pathways and required facilities}
\section{Type of collaboration required:  small/large}
collaboration with high energy particle physics community (know-how in building large vacuum systems from accelerator facilities)

\section{Suggested mechanisms}
\magentacomment{Josh: Would be good to get report, or whatever we can from the LLO vacuum workshop.}