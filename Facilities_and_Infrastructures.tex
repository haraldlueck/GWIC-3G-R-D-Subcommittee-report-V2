\chapterimage{Figures/Virgotube.jpg} % Chapter heading image
% Virgo beam tube copyright Virgo/EGO
\chapter{Facilities and Infrastructures}
%\section{Facilities and Infrastructures}
\label{sec:Fac_Inf}
%\vspace{-1cm}
3G observatory facility conceptual designs must be mature well before construction, as the infrastructure will dominate both cost and construction  new Observatories, and so carries great importance. The sensitivity and observation time achievable by the 3G detectors will be directly impacted by characteristics of the observatory sites and their infrastructure. Local noise (of natural and human origin) is a key parameter for evaluating candidate sites. Ground vibrations will limit the sensitivity directly, through seismic and Newtonian noise couplings, and indirectly by driving scattered light noise and by complicating interferometer control. Aspects such as the nature of the rock, abundance of water, and geological stability are particularly relevant for the stability and costs of underground detectors, such as ET. Anthropogenic activity can influence both underground but in particular surface installations. The construction of the sites, chiefly tunneling and leveling, together with the civil facilities and vacuum systems will also dominate the total project costs. Every effort to minimize costs, cost contingencies, and other collateral impacts on society in these domains will reap rewards, potentially pivotal, in approval and sponsorship.

% Josh: these are old paragraph bits I got rid of, let me know if anything is needed:
% 
% The characteristics of the site location and of the infrastructure directly impacts the final sensitivity and achievable observation time of the detector. 
% Construction of underground detectors (such as ET) must consider the nature of the rock, homogeneity over the arm lengths, abundance of water, long term geological stability both as cost drivers and overall site quietness. 
%are aspects which can impact the infrastructure cost and stability. These aspects are particularly relevant for underground infrastructure such as the one planned for hosting the ET detector.
%Optimizing the site facility and infrastructure's influence on  detector performance is only one aspect of the R\&D work required.  
% In addition to the site construction work (tunneling and leveling), the design of the civil facilities and vacuum systems are likely to dominate the total project construction costs.  

\section{High Level Design Considerations}
\label{Req:Fac_Inf}
%Quantitative requirements and figures-of-merit for site selection and infrastructure implementation still need to be worked out, but a 
High-level site-related design considerations include: Characteristics of seismic noise, surface meteorological conditions, anthropogenic noise, geological stability, site topography, specifically levelness for above ground detectors, rock type and the abundance of water, regional and national permitting and environmental clearances, and livability such as proximity to urban centers.  

% High-level site-related design considerations include:
% \begin{itemize}
% \item characteristics of seismic noise (on surface and underground)
% \item surface meteorological conditions (wind, rain)
% \item anthropogenic noise (population density, presence of industrial activity)
% \item geological stability and earthquake history
% \item type of underground rock and underground water abundance (for tunnel construction feasibility and cost) 
% \item regional and national permitting and environmental clearances
% \item site topography, specifically levelness (for above ground detectors) % orography
% \item proximity to urban centers (for live-ability)
% \end{itemize}

For underground detectors, the design of the experimental halls requires special attention. They have to be large enough to facilitate assembly and maintenance of detector components and to accommodate future upgrades. On the other hand, the larger the volume the larger the cost and the engineering challenge. Cavern volume and shape also affect atmospheric Newtonian noise (Section \ref{sec:Newtonian_Noise}), which might limit the sensitivity.

For surface detectors, 
% many of these same criteria are in effect; 
the Earth’s sagitta is of order 30\,m for a 40\,km laser-straight arm, somewhat blurring the concept of “surface” construction. This approach does, however, promote surface topography and surface geology in their priorities as site criteria. The direct effect of wind on above-ground structures also joins its indirect influences on seismicity and gravitational gradients. Effects on the artificial and natural environment, flora and fauna are also concerns, and must be factored differently into site selection, project approval, and design.
Advances in tunnel boring machines and in surface road and pipeline excavation have been seen to drastically reduce the cost per kilometer of public works structures over recent decades, e.g.,~\cite{BoringCompany}. Directed research into applying these modern methods to reduce the cost and collateral impact of 3G projects should be explored as a high priority. Attention must be also paid to the legal aspects of large-scale civil construction specific to each candidate country, which could impact the timing of the infrastructure realization. 

%\subsection{3G initial/future}

% \greencomment{Dave: I think this is better located at the end of the facilities section.  Also, if we become pushed to make cuts to save space; I think this section could be cut.} 

% As far as the civil infrastructure is concerned we cannot think of a step-wise approach. In the ET case the excavation of the underground infrastructure and the realization of the ground auxiliary works must be done once for all. It is a matter of realizing an expensive infrastructure, which must be working for decades and must therefore be stable  and capable of hosting the evolution of the detector.

%With respect to vacuum systems, 1G/2G experience suggests different approaches. LIGO elected to bake out and qualify its beam tubes for asymptotic "2G" service (about $10^{-9}$ Torr $\rm H_2$ and $10^{-11} $ Torr $\rm H_2O$) immediately at inception, but invested in no durable means to repeat this in future; Virgo chose to provide a permanent bake capacity, but to delay the expense of exercising the bake-out until it is required for observing sensitivity. In the construction phase, the  KAGRA interferometer devoted  a special care to smooth the inner surface of the tube via electro-polishing, a costly approach that could fulfill  residual pressure requirements without baking the vacuum system  after assembly.  Each was a valid approach in its specific Project context; Whether a gradual achievement of the respective required vacuum level in the 3G beam tubes (combined with the possibility of cost savings) is reasonable and sufficient depends on many factors, including the future financing of the detector upgrades and the future global observation plan.  
%%whether phased performance criteria (and some opportunity for cost savings) makes sense for 3G beam tubes will depend on many factors, including future operations funding and the future global observing plan.  

\section{Impact/relation to 2G and upgrades}
% \magentacomment{hal: derive required action items (Josh: Subscale demos, current gen equipment noise surveys)}

The sensitivity of second generation detectors, and their upgrades such as A+ and Virgo+, will continue to improve. 
% There is still room to \textbf{improve the sensitivity of the 2G detectors through incremental upgrades}. 
However, in the coming decade these detectors will reach their infrastructure limit. In other words, the level of environmental noise as well as the detector lengths and even the space available in the buildings will make it too challenging and expensive to improve further. The infrastructure for ET, with 10\,km-long underground triangular caverns, and CE, with a 40\,km-long L-shaped surface footprint, is designed to go well beyond these limits and house instruments that will continue to progress in sensitivity over their planned 50-year lifetime.   
% \greencomment{Dave: I realize this may be a controversial opinion not shared by all, but an infrastructure that is limited to $L=10 km$ will require all the improvements to come from making better measurements of $Delta_L$.  That seems a harder road to me...} 
% \magentacomment{hal: part of the improvement w.r.t. aLIGO comes from length also in the ET case cos (60 deg) x 10\,km = ca. 2 x 4\,km and the number of detectors = three, which makes it somewhat equivalent to a 13\,km detector, within the limits of comparability. I agree: it is still less than 40\,km} 
In creating these designs it is crucial to understand the 2G limits and to be sure that all related lessons have been learned. For instance, review of the machine-induced noise in the current infrastructures will allow the 3G site designs to ensure that otherwise quiet environments (and the benefits of going underground) are not spoiled by such disturbances. A joint effort based on the LIGO, Virgo and particularly KAGRA experience on this topic will be beneficial. 

In current practice, incremental upgrades of existing detectors like LIGO, GEO600  and Virgo have been an effective way to prove new technologies in context, at full sensitivity, while directly expanding the observing horizon. Enhanced LIGO, Advanced LIGO, Advanced Virgo, Virgo+ and A+ are examples of this effective combination of R\&D and practical deployment ``on the fly''  for improved observations.  
The incremental upgrade approach has limits, however.  One is the constraint to maintain compatibility with legacy systems and infrastructure; new topologies, footprints, and even wavelengths may be effectively off the table. Less obvious, but potentially more serious, is the growing imperative to minimize interruptions to observing. Upgrades require downtime.  The explosive discoveries of the last three years, particularly the multimessenger astronomy revolution triggered by GW170817,  have raised global desire to keep observing with existing 2G and 2G+ instruments. 
As a result, an increasing portion of 3G technology demonstration must rely on offline engineering development in subscale demonstrations, reprising the development environment of initial LIGO and Virgo, before any large-scale testbed existed. 

\section{3G vacuum systems}
Vacuum systems for planned 3G detectors are likely to be the largest ultra-high vacuum (UHV) systems built to date and will account for a significant part of the cost of building the observatories. Substantial innovation and research went into designing and building the LIGO, Virgo, GEO600 and KAGRA vacuum systems within economic constraints; the more stringent technical requirements and much greater size required for 3G vacuum envelopes threaten to render them infeasible without still further innovation. Many avenues for technical research have been proposed,  centered on themes of either improving an interferometer’s immunity to residual gas, or reducing the cost of suitable installations. Some questions that merit closer investigations include:

\begin{itemize}
\item  Are there economies in using materials other than stainless steel, such as mild steel or aluminum, as the envelope material? %What is really known about the vacuum properties of the inexpensive tubing (e.g., cold rolled steel) used for petroleum and gas transport? What are the vacuum experience and costs associated with aluminum alloys and plastics?
%For example, carbon steel (``mild'' steel), by virtue of improved manufacturing leading to extremely low hydrogen outgassing, is a potentially lower cost alternative to stainless steel. 
\item Are nested (e.g., differentially pumped) vacuum systems practical and economical? 
%For example, can an outer pressure vessel of inexpensive structural material protect a thin liner of UHV-compatible, bakeable vessel?  Can valves and pumping schemes be developed which ensure the integrity of the inner vessel?

\item Can civil construction mass production techniques, such as extrusion and spiral tube milling, be adapted to future UHV construction?

\item Are there newer mass production surface treatment and cleaning techniques that can be applied to reduce outgassing? %Is heating the best degassing method, or can UV or plasma excitation be cost-competitive at scale?

\item Are there ways to simultaneously meet surface outgassing requirements and possibly distributed pumping together with other physical requirements of the system, such as 
% (e.g. for gravitational wave applications) 
stray light attenuation, vibration damping and particulate mitigation?

\item Can optical pressure gauging and leak detection offer practical advantages for system construction, commissioning and maintenance?

\item Are there new getter materials and surface treatments for maintaining UHV conditions for very large systems with modest gas loads?

\item Are there new ideas for reliable, affordable large gate valves to isolate from atmospheric pressure during construction and service, and to isolate volumes with different requirements?

\item What are effective methods or surface treatments to minimize moisture adsorption during vented system access and to accelerate desorption during pumpdown and recovery?

\item What vacuum tube diameters are optimal taking into account light scattering issues, vacuum conductance etc. This will This will in turn influence tunnel size and costs.

\end{itemize}

In January 2019, an NSF-sponsored Workshop on Large Ultrahigh-Vacuum Systems was held to identify cost effective technologies for the design, construction and operation of the large vacuum systems required for 3G observatories~\cite{LLOVacWorkshop2019}. This workshop considered many of the issues above and particularly focused on two concepts: one based on an extrapolation of the single-walled vacuum pipes used by LIGO, Virgo, and KAGRA and another using double-walled nested vacuum pipes, with the outer wall handling the atmospheric load and the inner wall supporting ultra-high vacuum. Both of these concepts were found to be viable and recommended to be taken to the next level of detail in follow-on studies. Vacuum pumping and pipe surface treatment solutions were also considered. The participants reported confidence that positive impacts could be achieved for the cost of the construction and operation of vacuum systems for 3G detectors. They provided recommendations for further study including: investigation of beamtube materials such as mild steel and aluminum; further characterization (outgassing and optical properties) of potential surface coatings; the use of ultra-dry gas purging during venting (even in current detectors); and leveraging partnerships with NIST, JLab, CERN, and industrial contractors. A follow-up workshop is planned for Fall 2019.

\section{Outlook and timeline}
Facilities and infrastructure design and costing studies must evolve very soon to studies undertaken by professionals. 3G construction must start roughly 5 years before detector installation and ten years before observations. To allow 2035 observations, the responsible international collaborations and their working groups should be focused on bringing facility-related R\&D and detailed technical studies to maturity imminently.
Careful study of candidate sites considering the above points, as well as studies of drivers of facility decisions that can be made with 2G instruments and scaled prototypes, need to be completed now. For vacuum systems, facilities for testing outgassing rates and optical properties of vacuum pipe materials and coatings are needed. Additionally, as design studies progress, facilities to test scaled vacuum concepts may be required.
The NSF vacuum workshop is a good model for leveraging expertise and could be replicated for other areas of facilities and infrastructure work.
Collaboration on facilities and infrastructure research, design, and testing with high energy particle physics community and with industry will be particularly valuable to leverage experience in building large accelerator facilities and vacuum systems.

%\section{Pathways and required facilities}
%Careful study of candidate sites considering the above points, as well as studies of drivers of facility decisions that can be made with 2G instruments and scaled prototypes need to be completed now. For vacuum systems, facilities for testing outgassing rates and optical properties of vacuum pipe materials and coatings are needed. Additionally, as design studies progress, facilities to test scaled vacuum concepts may be required. 

%\section{Type of collaboration required:  small/large}
%Collaboration on facilities and infrastructure research, design, and testing with high energy particle physics community and with industry will be particularly valuable to leverage experience in building large accelerator facilities and vacuum systems.

%\section{Suggested mechanisms}
%Facilities design and costing studies must continue and become more concrete. The NSF vacuum workshop is a good model for leveraging expertise and could be replicated for other areas of facilities and infrastructure work. 

%\section{Timelines}
%3G Infrastructure and facilities construction will start roughly 5 years before the installation of detector hardware and ten years before observations. Thus, the timescales for study and design are short and critical. For site preparations for ET and CE to start as early as $\sim$\,2025, the respective R\&D and detailed technical studies have to reach conclusions imminently. The responsible international collaborations and their working groups should be focused on this goal.