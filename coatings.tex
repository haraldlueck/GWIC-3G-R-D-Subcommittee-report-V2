\chapterimage{Figures1_3/coatingsmall1_3.jpg} % Chapter heading image
% small sample mirror copyright Harald Lueck
\chapter{Coatings}
%\section{Coatings - roadmap to readiness --- Martin Fejer}
\label{sec:Coatings}

\vspace{1cm}

%\section{Background}

%Thermal noise of the mirror coatings (see Box~\ref{Box:Thermal}) sets a fundamental mid-frequency limit to the sensitivity achievable by the current room temperature gravitational-wave detectors, aLIGO and AdVirgo~\cite{AdvancedVirgo2015, AdvancedLIGO2015}. The coatings \ac{RaD}   community is working to deliver coatings with improved thermal noise for the imminent aLIGO+ and adVirgo+~\cite{Zucker:LIGOAplus, Cagnoli:VirgoAplus} detectors. 
% Mirror coatings with improved thermal noise performance are key to the imminent enhancements to these detectors, A+ and AdVirgo+~\cite{Zucker:LIGOAplus, Cagnoli:VirgoAplus}, and a current focus of the community. 

Coating thermal noise limits the mid-band design sensitivity of current (\ac{aLIGO}, \ac{AdVirgo})~\cite{ AdvancedLIGO2015,AdvancedVirgo2015}, future enhanced versions of those detectors (\ac{a+LIGO}, \ac{AdVirgo+}) \cite{Zucker:LIGOAplus, Cagnoli:VirgoAplus}, potential \ac{2.5G} detectors (\ac{Voyager})~\cite{VoyagerDCC2018}, and \ac{3G} detectors (\ac{ET}, \ac{CE})~\cite{ET2011,CosmicExplorer2017}.  While \ac{a+LIGO} and \ac{AdVirgo+} will operate at room temperature with 1.06\,$\mu$m lasers, the designs of the \ac{2.5G} and \ac{3G} detectors are still evolving, so therefore are the requirements on the mirror coatings. In the following discussion we assume the parameters for future detectors as set out in Table.\,\ref{Tab:FutIfos}. 

%Significant \ac{RaD}   on coating thermal noise is required to ensure that the sensitivity targets of \ac{3G} detectors are met.

%that Voyager will operate at 123\,K with a 2\,$\mu$m laser, \ac{ET-LF}    will operate at 20 K with a 1.5\,$\mu$m laser, and that the first stage of Cosmic Explorer will operate at room temperature and a wavelength of 1064nm and in phase 2 at 123\,K and 2\,µm.

Thermal noise is fundamentally connected to the mechanical and optical properties of the mirror coatings through the fluctuation-dissipation theorem (see Box~\ref{Box:Thermal}). For coatings, the thermal noise is proportional to the square root of the operating temperature \ac{T} of the mirrors, the \ac{phi}mechanical loss angle   of the coating at that temperature, the \ac{w}, and the coating thickness \ac{d} ~\cite{levin1998internal}. In terms of strain as measured by the interferometer the coating thermal noise is also inversely proportional to the arm length \ac{L}, such that,
\begin{equation}
\text{CTN} \propto \frac{1}{L} \, \sqrt{\frac{T}{f} \frac{1}{w^2} \, \varphi \, d   },
\end{equation}\label{eq:CTN}
where \ac{f} is the frequency. Each future detector has a specific set of parameters \ac{T}, \ac{w}, and \ac{L} different from the others, hence the requirements on coating material parameters, \ac{phi} and \ac{d}, vary significantly from one detector to the other. For the advanced-plus detectors a 4-fold reduction in mechanical loss is targeted.

%Coating thermal noise will also play a prominent role in next-generation detectors\,\cite{DawnIV2018}. 
%The Voyager concept~\cite{VoyagerDCC2018} requires low thermal noise coatings for operations at 123\,K and longer laser wavelengths; coating thermal noise limits the design sensitivity of Einstein Telescope's high frequency detector ET-HF~\cite{ET2011} and CE1, the initial stage of Cosmic Explorer~\cite{CosmicExplorer2017}.  

Thermal noise is not the only requirement that must be met. The total optical absorption for room temperature interferometers, (\ac{A+} and \ac{3G}) must be less than 0.5\,ppm, although some relaxation of this target is possible if required by low thermal noise coatings at the expense of making thermal management more challenging. Recently small defects have been recognized as affecting the performance of current detectors; scattering centres with sizes from tens of nanometers to several microns, and points with high absorption are clearly visible on the mirrors. Coating \ac{RaD}   has to take into account these optical properties as well as mechanical losses in order to meet the stringent requirements comparable to those for current detectors~\cite{AdvancedLIGO2015,AdvancedVirgo2015}. Advancements on the coating deposition technology are also necessary for new detectors, especially when considering the need for larger mirror diameters. 

%Other defects like formation of bubbles, cracks and delamination limit the coating realization.
%The lack of understanding of the physics of deposition is another limiting factor to coating development. 
%Coating design and monitoring is also becoming essential.

%In that respect the same considerations apply to amorphous and crystalline coatings. If molecular beam epitaxy (MBE) seems not be replaceable for perfect crystal growth, improvements for ion beam sputtering of amorphous coatings seem possible.

% Josh thinks the below paragraph is covered in the detector overviews in the introduction, esepcially the boxes and the tables. 

% While a+LIGO and AdVIRGO+ will operate at room temperature with 1.06\,$\mu$m lasers, the designs of the \ac{2.5G} and \ac{3G} detectors are still evolving, so therefore are the requirements on the mirror coatings. For definiteness in the following discussion, we assume that Voyager will operate at 123\,K with a 2\,$\mu$m laser, \ac{ET-LF}    will operate at 20 K with a 1.5\,$\mu$m laser, and that Cosmic Explorer will operate at a temperature and wavelength that emerges as most advantageous given the experience with the enhanced \ac{2G} and \ac{2.5G} detectors.

%The power spectral density of thermal noise is proportional to the operating temperature of the mirrors, the mechanical loss angle of the mirror coatings at that temperature, and the thickness of the mirror coating~\cite{levin1998internal}. 
%For all of the detectors above, a four-fold reduction, from the current coatings at room temperature, in the product of coating thickness and mechanical loss angle will be adequate to meet thermal noise requirements. Coating thickness can be reduced by identifying materials with a strong index of refraction contrast. Avenues for mechanical loss reduction include improved amorphous coatings, semiconductor and crystal coatings.
% Mechanical loss reduction relies on a cycle of mechanical measurements, structural characterization, and modeling.

% For all cases, a product of coating thickness and mechanical loss equal to approximately a 4-fold reduction on the performance at room temperature of the current coatings (and in the case of Voyager the reduction in thickness allowed by a-Si/SiO2 coatings) will be adequate to meet thermal noise requirements. 

% Josh thinks the below sentence has nothing to do with coatings

The situation in \ac{ET-LF}    is somewhat different, in that the thermal noise is dominated by the mirror suspension, which must be able to extract the thermal load imposed by the optical power absorbed in the mirror; in this case the optical absorption plays a key role in the thermal noise performance, and must be held in the range of 1\,ppm \cite{HiEA2011}.

% \magentacomment{hal: I realised that scattering and esp. absorption from point sources, which is an issue right now, is not mentioned. Shall this be included? }

Mechanical loss in amorphous materials results from the coupling of elastic energy into low energy excitations of the materials, generally thought of as \ac{TLS}~\cite{braginsky1985systems,bommel1956dislocations}. These \ac{TLS} typically involve motions of several dozen atoms, and different \ac{TLS} are responsible for losses at different temperatures (low barrier heights at low temperatures, higher barriers at higher temperatures)~\cite{hamdan2014molecular,trinastic2016molecular}. Reducing the mechanical loss thus requires reducing the density of two-level systems with the barrier heights pertinent to the operating temperature. 
% It is important to note that a process that reduces the losses for one temperature may increase it for another, if the distribution of two-level systems is reduced at one barrier height at the expense of increasing it at another.

%In addition to meeting thermal noise requirements, the mirrors also must meet stringent optical specifications, such as absorption, scatter and figure error, that are comparable to those for current detector designs~\cite{AdvancedVirgo2015, AdvancedLIGO2015}. These are near the state of the art but attainable for conventional ion-beam-sputtering (IBS) deposition of amorphous oxide mirrors; if other types of materials and deposition methods emerge as necessary to meet the mechanical loss requirements, significant deposition techniques and tool development will be necessary. Additionally, the causes of point absorption and point scattering observed currently in the optical coatings in aLIGO and AdVirgo must be understood and eliminated for next-generation coatings. 

\section{Current Approaches to Low Mechanical Loss Mirror Coatings}

The sensitivity limit due to the coating thermal noise is explained through equation\,\ref{eq:CTN} and in Appendix~\ref{sec:Appendix_Coatings} where the relevant parameters are presented. Among those is mechanical loss, which will be the subject of this section. The density and distribution of the \ac{TLS} responsible for mechanical loss depend on the coating materials, and they can be altered through deposition conditions and post-deposition treatments.

\noindent Promising materials to reducing mechanical losses of optical coatings are listed below and described in more detail in Appendix~\ref{sec:Appendix_Coatings} and in\,\cite{DawnIV2018}.
\begin{itemize}
    \item\textbf{Improved conventional amorphous oxides:} Increased annealing temperature by suppressing crystallization through mixing.
    \item\textbf{Alternative amorphous materials:} Semiconductors such as amorphous silicon or silicon nitride.
    \item\textbf{Multi-materials:} Amorphous semiconductor materials currently have adequate mechanical properties but excess optical absorption; Consider coatings with amorphous layers where the optical intensity is highest and semiconductor layers below them.
    \item\textbf{Crystalline coatings:} Alternating layers of \ac{AlGaAs/GaAs} have shown favorable mechanical loss properties; \ac{GaP/AlGaP} should also be explored.
\end{itemize}
Deposition parameters that are worth exploring are: 
\begin{itemize}
    \item\textbf{High temperature deposition:} producing ultrastable glasses where \ac{TLS} are significantly reduced.
    \item\textbf{Nanolayering:} nm-thick layers are able to frustrate crystallization and to modify the \ac{TLS} distribution.
    \item\textbf{Ion energy and deposition rates:} the energy of ions and the deposition rates are the parameters that seem to impact the mechanical losses the most. A model of the physics of deposition is necessary in order to clarify the relation between the optical and mechanical parameters of coatings with that of the physical condition of deposition.
\end{itemize}
Finally, \textbf{post deposition annealing} is able to change the \ac{TLS} distribution and therefore mechanical loss. The dynamics of the annealing process is poorly understood and should be explored further.
These investigations will benefit from synergy between 1) synthesis of samples, 2) microscopic and macroscopic characterization, and 3) modelling of deposition, of the amorphous materials and loss calculations.

\section{Current Research Programs}

There are several 
% current and planned 
programs devoted to developing low-thermal-noise mirror coatings suitable for enhanced \ac{2G}, \ac{2.5G}, and \ac{3G} detectors. Participants are involved in all aspects of coating research, including various deposition methods, characterization of macroscopic properties at room and cryogenic temperatures, and atomic structure modeling and characterization. The recent incorporation 
% into the project 
of several groups involved in coating deposition is particularly important, as the costs and time delays associated with commercial deposition of research coatings have been significant impediments to rapid progress.

There are about ten U.S. university research groups participating in 
% various aspects of 
coating research. In 2017, a more coordinated effort and additional funding for these groups were initiated under the \ac{CCR}, jointly funded by the Gordon and Betty Moore Foundation and the NSF. Work in the U.S. also importantly includes that in the \ac{LIGO} Laboratory. These efforts are complemented by groups not formally affiliated with the \ac{CCR}, notably by the large coatings group at U. Montreal and U. Laval. \ac{GEO} has a major coatings effort with five university research programs in the \acs*{U.K.} and Germany. 
% (U. Glasgow, U. Strathclyde, U. West of Scotland, U. Hamburg, U. Hannover). 
The efforts of all these groups are coordinated through 
% biweekly telecons of 
the \ac{LSC} Optics Working Group.

About ten universities are involved in the Virgo Coatings \ac{RaD}   (\ac{RaD}  ) Project that has its research plan focused on new materials, deposition conditions, post-deposition treatments and metrology. The \acs*{ViSIONs} project, supporting six French laboratories, including the \ac{LMA} and \ac{ILM}, is focused on studying the relationship between the physical properties of sputtered or evaporated materials and the structural and macroscopic properties of the deposited films. 
% Josh this sentence was in here twice, and I think its second use is better:
% LMA is the only facility currently operational that is capable of producing coatings of the size and optical quality required for gravitational-wave interferometers. 
\ac{LMA} is improving the uniformity of their coating deposition to meet the challenges of the Advanced+ detectors. They have also developed plans to upgrade their coaters and tools to accommodate the increased size and weight of the 55\,cm end mirrors being considered for \ac{AdVirgo+}.
% The AdV+ project is considering the use of end cavity mirrors of 55\,cm diameter. Therefore LMA has developed plans to upgrade their coaters and tools to deal with such increase of diameter and weight.

\section{Timelines}

The most pressing timeline is for the enhanced \ac{2G} detectors.The necessary research to identify a coating material and process should be completed by May 2020 for \ac{a+LIGO} and by the end of 2020 for \ac{AdVirgo+}. This timeline implicitly assumes that the coating will be a sputtered amorphous oxide or silicon nitride, possibly deposited at elevated temperature, slower than conventional rate and/or higher annealing temperatures than conventional. It is unlikely that there will be time during this period to identify coatings and develop the necessary equipment for a process with significantly different deposition processes than the current ones.

%In recognition of the fact that meeting the enhanced \ac{2G} detector timeline requires doing basic research on a development schedule, the efforts of the LSC community are of necessity focused on these mirrors. That said, it is also recognized that the community shouldn't put itself again in such a situation, so a portion of the current research effort is devoted to establishing approaches to mirrors for \ac{2.5G} and \ac{3G} detectors. It is difficult to set research timelines for this effort, since the funding and construction schedules for these systems are not yet established. Another open question is the deposition process that will be required for the \ac{2.5G} and \ac{3G} mirrors; the further that process is from conventional IBS, the longer it is likely to take to develop suitable tooling. It seems that in any plausible scenario at least five years are available for research into the best approach to mirrors meeting \ac{2.5G} requirements. Results for these \ac{2.5G} mirrors will, in turn, inform choices with respect to \ac{3G} mirrors. It is therefore too soon to argue for a large investment in scaling deposition tools alternative to elevated-temperature IBS for \ac{2.5G} and \ac{3G} mirrors. That said, as the funding trajectories and interferometer architectures become better defined, it will be important to regularly re-evaluate the current understanding of potential mirror technologies, and make critical decisions, especially with respect to deposition tools with long development times and requiring major financial investments.

The efforts of the \ac{LVC} community are currently focused on meeting the coating thermal noise requirements of the enhanced \ac{2G} detectors, which requires conducting basic research on a development timeline. It is recognized and recommended by the community that the development of coatings for future detectors should not be subject to such a constraining timeline. Therefore, a portion of the current research effort is devoted to establishing approaches to mirrors for \ac{2.5G} and \ac{3G} detectors. At the moment, it is difficult to set research timelines, since the funding and construction schedules for these detectors are not yet established.

The deposition process required for \ac{2.5G} and \ac{3G} mirrors remains an open question. The further that process deviates from the currently used \ac{IBS} technology, the longer it will take to develop. In any plausible scenario, at least 5 years are available for research into finding a viable \ac{2.5G} coating. The results for these \ac{2.5G} mirrors will then inform available choices for \ac{3G} mirrors. It is therefore too soon to argue for a large investment in scaling deposition tools alternative to \ac{IBS} for \ac{2.5G} and \ac{3G} mirrors.

As plans for future interferometers mature and funding is secured, it will be important to regularly re-evaluate potential mirror technologies that meet specifications, in order to focus research and development efforts. This is especially important for enabling critical decisions on down-selecting coating materials or technology in a timely manner, especially with respect to developing deposition tools that may require long development times and major financial investment.



\section{Outlook and Recommendations}
\label{coatings_Recomm}
We recommend
\begin{itemize}
\item Continued and deeper coordination among all gravitational-wave in order to maximize the possibility of developing new coatings with the requisite \ac{3G} performance in the limited time available;
\item  parallel research lines be developed, making a division of tasks among the research groups necessary;
\item support for several producers of large high quality coatings and to work with them on ongoing research efforts as an essential risk mitigation for the \ac{3G} effort.
\end{itemize}


%The additional funding provided by the CCR for U.S. efforts in coating research, combined with the previously existing NSF support, leave these efforts with reasonably adequate funding in the near future. There is currently more capacity for characterization of properties (other than cryogenic mechanical losses) than for synthesis. It is important that the LIGO Lab continue with at least its current effort level, as their contribution to high-throughput mechanical loss characterization, optical scatter and homogeneity measurements, and their overall coordination of sample fabrication, distribution, and characterization is critical to the coating research success.

In 2017, the U.S. efforts in coatings research received a significant funding increase enabling the creation of the \ac{CCR}.  It is important that this level of funding at least be maintained in order to carry out the required research and development for mirror coatings for future interferometers. It is also important that the \ac{LIGO Lab} continue with at least its current efforts, as their contribution to high-throughput mechanical loss characterization, optical scatter and homogeneity measurements, and their overall coordination of sample fabrication, distribution, and characterization is critical to the success of coatings research. Currently, adequate capacity exists for characterization of properties (other than cryogenic mechanical loss); an increase in modeling and synthesis capabilities would enhance our ability to inform and develop new coating materials.

In the \ac{GEO} collaboration, there is growing capacity for depositing coatings at Strathclyde, UWS and Hamburg. These coatings can be produced at a rate faster than it is possible to characterize their properties at cryogenic temperatures. At Strathclyde, a large chamber with an industrial ion source, which should be capable of producing large and uniform coatings is in development. This is an important priority that will provide an additional coating facility suitable for the deposition of large, high-quality coatings. Studies of \ac{GaP/AlGaP} crystalline coatings are planned using hardware now installed in an MBE chamber at Gas Sensing Solutions Ltd. This is an important parallel research direction for the development of amorphous coatings.

\ac{RaD}   is responsible for a significant research activity that is complementary to that carried out in the LSC and its planning has to be supported. Production of samples is done in four labs, modelling in two and characterizations at different scale is widely distributed in all the collaboration. The mechanical loss measurement setups known as GeNS is one example of fertilization of VCR\&D to the LSC community. 

With the imbalance of coating production and coating characterisation capacities in the various collaborations it is prudent to join forces in a globally coordinated way. 
%In VIRGO, LMA and University of Sannio are able to provide high quality coatings at a rate higher than the existing characterization capability in \ac{RaD}  . 
% The ViSIONs project is investigating the role played by deposition parameters in determining the properties of the coatings.

LMA is presently the only institution, capable of depositing coatings with the size and quality required by gravitational-wave detectors. LMA will continue to be supported as a research and coating facility for future gravitational-wave detectors by the French CNRS through EGO. Activities at CSIRO in Australia, which was able to provide similar coating quality, had been discontinued but are now being transferred to and revived at ANU. Supporting several producers of large high quality coatings and to work with them on ongoing research efforts is an essential risk mitigation for the \ac{3G} effort.


% Within these activities, the products of other commercial vendors can also be characterized.  


%At present, there is only one group, LMA, capable of depositing coatings of a size and quality suitable for gravitational wave interferometers. While French CNRS through EGO is fully committed to continue supporting LMA as a research and coating facility for future GW detectors, a single source for any critical component is viewed as a potential risk to the whole community. There was a second system capable of such depositions, though subsequently defunded, at CSIRO in Australia. It seems prudent to re-establish this program, both to provide a second source for full-scale coatings, and as a significant contributor to the ongoing research efforts, as well as to certify additional commercial vendors. 

%\greencomment{Dave:This is an important recommendation.  I would even broaden it to say that not just CSIRO, but also new manufacturers should be certified.}
%\magentacomment{hal: mention the existence or lack thereof of commercial companies}

% While there is a good level of coordination within the LSC coating research programs, and within the VIRGO collaboration, the interaction between LSC and VIRGO groups has been less effective. Current efforts underway to establish an agreement on how to manage these interactions to enable more efficient exchange of information will be invaluable in generating synergy between the programs, and avoiding unnecessary duplication of efforts.
%\greencomment{Dave: I get why this is in here, but this seems too 'political' to me as written.  I would say that continued and deeper coordination among all GW collaborations will be critically important in developing coatings with the requisite performance.} \\
%\redcomment{Geppo: Paragraph removed}

\subsection{Roadmap}
Since the development of low-thermal-noise coatings is in the stage of research rather than development, even for \ac{2.5G} detectors, a conventional roadmap is not the best model for describing the path towards identifying suitable mirror designs and fabricating corresponding full-scale mirrors for \ac{3G} detectors. Considering that the architectures and operating parameters for \ac{3G} detectors remain in flux, that the results for mirrors developed for \ac{2.5G} detectors will have a strong, perhaps decisive, influence on the designs for \ac{3G} mirrors, and that the funding and therefore construction schedules for \ac{3G} detectors are not yet clear, it is best to estimate schedule implications in terms of time requirements before installation. 

The currently plausible approaches to \ac{3G} mirrors fall into three broad categories which have different cost and schedule drivers: amorphous coatings deposited by methods similar to conventional IBS, amorphous coatings deposited by alternative means, e.g. chemical-vapor deposition (CVD), and crystalline coatings. 

\textbf{Amorphous coatings deposited by IBS methods:} 
Currently, IBS is the only mature technology for deposition of mirror coatings suitable for GW interferometers (GWI); the challenge is to find the appropriate combination of material, deposition conditions (rate, ion energy, substrate temperature, etc.) and post-deposition treatments to meet mechanical loss requirements and optical specifications. The time required for this step is unknown; a solution may be found next month or may not exist at all. Once the materials and conditions have been determined, the time to production readiness will depend on the differences in current IBS practice: one year would be sufficient for room temperature deposition at conventional rates. For more extreme conditions with increased substrate temperature, low rate, microwave annealing, etc., perhaps 3-5 years and USD 3-5M would be needed to develop appropriate equipment and processes. Multi-material coatings would fall between these extremes of time and equipment costs. 

\textbf{Amorphous coatings deposited by methods other than IBS:} 
if research shows that the optimal deposition method is other than IBS, e.g. CVD of a-Si/SiNx mirrors, in addition to open-ended research time (similar to the IBS case) sufficient time would be required to develop deposition tools and a scaled-up process suited to GWI requirements. While CVD tools are widely used in the semiconductor industry, adaptation to GWI mirror requirements would be a significant effort. Instrumentation development and the more complex pathfinder process to production readiness might take USD 25-30M and perhaps 10 years. These figures are order of magnitude estimates that can be tightened up by discussions with equipment vendors. 

\textbf{Crystalline coatings:} for AlGaAs crystalline mirrors, the materials research and modeling phase for small scale (ca. 15\,cm) coatings could be completed in perhaps three years. If these results showed high performance, the time and financial costs for substrate and tool development and the more complex pathfinder process to scale to 45\,cm optics might take about USD 25M and 5 years. 

%\greencomment{Stan: Still (Harald: at the time before the last edit 21.6.19) very disorganized, too disorganized for me to have anything useful to say about it.  Except that maybe reorganizing it into separate sections for the different coating tupes that need to be investigated (as was done in the core optics section) might make this seem less like a laundry list.  }
 
%\section{References}
%References will be included in text once final length and content are determined
%\cite{Levin:Direct}


