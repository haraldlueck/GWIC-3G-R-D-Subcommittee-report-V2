\chapterimage{Figures1_3/coatingsmall1_3.jpg} % Chapter heading image
% small sample mirror copyright Harald Lueck
\chapter{Coatings}
%\section{Coatings - roadmap to readiness --- Martin Fejer}
\label{sec:Coatings}

\vspace{1cm}

%\section{Background}

Thermal noise of the mirror coatings (see Box~\ref{Box:Thermal}) sets a fundamental mid-frequency limit to the sensitivity achievable by the current room temperature gravitational-wave detectors, aLIGO and AdVirgo~\cite{AdvancedVirgo2015, AdvancedLIGO2015}. The coatings R\&D community is working to deliver coatings with improved thermal noise for the imminent aLIGO+ and adVirgo+~\cite{Zucker:LIGOAplus, Cagnoli:VirgoAplus} detectors. 
% Mirror coatings with improved thermal noise performance are key to the imminent enhancements to these detectors, A+ and AdVirgo+~\cite{Zucker:LIGOAplus, Cagnoli:VirgoAplus}, and a current focus of the community. 
Coating thermal noise will also play a prominent role in next-generation detectors\,\cite{DawnIV2018}. The Voyager concept~\cite{VoyagerDCC2018} requires low thermal noise coatings for operations at 123\,K and longer laser wavelengths; Coating thermal noise limits the design sensitivity of Einstein Telescope's high frequency detector ET-HF~\cite{ET2011} and CE1, the initial stage of Cosmic Explorer~\cite{CosmicExplorer2017}. Significant R\&D on coating thermal noise is required to ensure that the sensitivity targets of 3G detectors are met. 

%Thermal noise limits the mid-band design sensitivity of current (aLIGO, AdVIRGO)\cite{AdvancedVirgo2015, AdvancedLIGO2015}, enhanced versions of those detectors (a+LIGO, AdVIRGO+) \cite{Zucker:LIGOAplus, Cagnoli:VirgoAplus}, 2.5G detectors (Voyager)\cite{VoyagerDCC2018}, and 3G detectors (ET, Cosmic Explorer) \cite{ET2011,CosmicExplorer2017}. In all these cases, the thermal noise is fundamentally connected to the mechanical and optical properties of the dielectric mirror coatings. 

% Josh thinks the below paragraph is covered in the detector overviews in the introduction, esepcially the boxes and the tables. 

% While a+LIGO and AdVIRGO+ will operate at room temperature with 1.06\,$\mu$m lasers, the designs of the 2.5G and 3G detectors are still evolving, so therefore are the requirements on the mirror coatings. For definiteness in the following discussion, we assume that Voyager will operate at 123\,K with a 2\,$\mu$m laser, ET-LF will operate at 20 K with a 1.5\,$\mu$m laser, and that Cosmic Explorer will operate at a temperature and wavelength that emerges as most advantageous given the experience with the enhanced 2G and 2.5G detectors.

The power spectral density of thermal noise is proportional to the operating temperature of the mirrors, the mechanical loss angle of the mirror coatings at that temperature, and the thickness of the mirror coating~\cite{levin1998internal}. 
For all of the detectors above, a four-fold reduction, from the current coatings at room temperature, in the product of coating thickness and mechanical loss angle will be adequate to meet thermal noise requirements. Coating thickness can be reduced by identifying materials with a strong index of refraction contrast. Avenues for mechanical loss reduction include improved amorphous coatings, semiconductor and crystal coatings.
% Mechanical loss reduction relies on a cycle of mechanical measurements, structural characterization, and modeling.

% For all cases, a product of coating thickness and mechanical loss equal to approximately a 4-fold reduction on the performance at room temperature of the current coatings (and in the case of Voyager the reduction in thickness allowed by a-Si/SiO2 coatings) will be adequate to meet thermal noise requirements. 

% Josh thinks the below sentence has nothing to do with coatings

% The situation in ET-LF is somewhat different, in that the thermal noise is dominated by the mirror suspension, which must be able to extract the thermal load imposed by the optical power absorbed in the mirror; in this case the optical absorption plays a key role in the thermal noise performance, and must be held in the range of 1\,ppm \cite{HiEA2011}.

% \magentacomment{hal: I realised that scattering and esp. absorption from point sources, which is an issue right now, is not mentioned. Shall this be included? }

Mechanical loss in amorphous materials results from the coupling of elastic energy into low energy excitations of the materials, generally thought of as two-level systems~\cite{braginsky1985systems,bommel1956dislocations}.
% in an effective double-well potential in some appropriate configuration coordinate 
These two-level systems typically involve motions of several dozen atoms, and different two-level systems are responsible for losses at different temperatures (low barrier heights at low temperatures, higher barriers at higher temperatures)~\cite{hamdan2014molecular,trinastic2016molecular}. Reducing the mechanical loss thus requires reducing the density of two-level systems with the barrier heights pertinent to the operating temperature. 
% It is important to note that a process that reduces the losses for one temperature may increase it for another, if the distribution of two-level systems is reduced at one barrier height at the expense of increasing it at another.

In addition to meeting thermal noise requirements, the mirrors also must meet stringent optical specifications, such as absorption, scatter and figure error, that are comparable to those for current detector designs~\cite{AdvancedVirgo2015, AdvancedLIGO2015}. These are near the state of the art but attainable for conventional ion-beam-sputtering (IBS) deposition of amorphous oxide mirrors; if other types of materials and deposition methods emerge as necessary to meet the mechanical loss requirements, significant deposition techniques and tool development will be necessary. Additionally, the causes of point absorption and point scattering observed currently in the optical coatings in aLIGO and AdVirgo must be understood and eliminated for next-generation coatings. 

\section{Current approaches to low-noise mirror coatings}

Promising approaches to reducing thermal noise of optical coatings are listed below and described in more detail in Appendix~\ref{sec:Appendix_Coatings} and in\,\cite{DawnIV2018}.
\begin{itemize}
    \item\textbf{Improved conventional amorphous oxides:} Increased annealing temperature by suppressing crystallization with dopants or using nano-layers; low deposition rates; guidance from theoretical molecular dynamics and atomic structure characterization; ultra-stable glasses.
    \item\textbf{Alternative amorphous materials:} Semiconductors such as amorphous silicon (a-Si) or silicon nitride (SiN).
    \item\textbf{Multi-materials:} Amorphous semiconductor materials currently have adequate mechanical properties but excess optical absorption; Consider coatings with amorphous layers where the optical intensity is highest and semiconductor layers below them.
    \item\textbf{Crystalline coatings:} Alternating layers of AlGaAs/GaAs have shown favorable mechanical loss properties; GaP/AlGaP should also be explored.
\end{itemize}

\section{Current research programs}

There are several 
% current and planned 
programs devoted to developing low-thermal-noise mirror coatings suitable for enhanced 2G, 2.5G, and 3G detectors. Participants are involved in all aspects of coating research, including various deposition methods, characterization of macroscopic properties at room and cryogenic temperatures, and atomic structure modeling and characterization. The recent incorporation 
% into the project 
of several groups involved in coating deposition is particularly important, as the costs and time delays associated with commercial deposition of research coatings have been significant impediments to rapid progress.

There are about ten U.S. university research groups participating in 
% various aspects of 
coating research. In 2017, a more coordinated effort and additional funding for these groups were initiated under the Center for Coatings Research (CCR), jointly funded by the Gordon and Betty Moore Foundation and the NSF. Work in the U.S. also importantly includes that in the LIGO Laboratory. These efforts are complemented by groups not formally affiliated with the CCR, notably by the large coatings group at U. Montreal and U. Laval. GEO has a major coatings effort with five university research programs in the U.K. and Germany. 
% (U. Glasgow, U. Strathclyde, U. West of Scotland, U. Hamburg, U. Hannover). 
The efforts of all these groups are coordinated through 
% biweekly telecons of 
the LSC Optics Working Group.

About ten universities are involved in the VIRGO Coatings R\&D Project, including new material research, metrology and 
%more recently 
simulation. The ViSIONs project, supporting six French laboratories, including LMA and ILM (two leading two coating laboratories in Lyon, France), is focused on studying the relationship between the physical properties of sputtered or evaporated materials and the structural and macroscopic properties of the deposited films. 
% Josh this sentence was in here twice, and I think its second use is better:
% LMA is the only facility currently operational that is capable of producing coatings of the size and optical quality required for gravitational-wave interferometers. 
LMA is improving the uniformity of their coating deposition to meet the challenges of the Advanced+ detectors. They have also developed plans to upgrade their coaters and tools to accommodate the increased size and weight of the 55\,cm end mirrors being considered for adVirgo+.
% The AdV+ project is considering the use of end cavity mirrors of 55\,cm diameter. Therefore LMA has developed plans to upgrade their coaters and tools to deal with such increase of diameter and weight.

\section{Timelines}

By far the most pressing timeline is for the enhanced 2G detectors. In the case of aLIGO+, allowing for a one-year pathfinder after the coating material and process are identified, the required research to identify a suitable mirror coating should be completed by May 2020. This timeline implicitly assumes that the coating will be a sputter-deposited amorphous doped oxide, perhaps deposited at an elevated temperature, a slower than conventional rate and/or with a higher than conventional annealing temperature. It is unlikely that there is time in such a schedule to identify coatings and develop the required equipment for a process with more significant deviations from current deposition methods.

In recognition of the fact that meeting the enhanced 2G detector timeline requires doing basic research on a development schedule, the efforts of the LSC community are of necessity focused on these mirrors. That said, it is also recognized that the community shouldn't put itself again in such a situation, so a portion of the current research effort is devoted to establishing approaches to mirrors for 2.5G and 3G detectors. It is difficult to set research timelines for this effort, since the funding and construction schedules for these systems are not yet established. Another open question is the deposition process that will be required for the 2.5G and 3G mirrors; the further that process is from conventional IBS, the longer it is likely to take to develop suitable tooling. It seems that in any plausible scenario at least five years are available for research into the best approach to mirrors meeting 2.5G requirements. Results for these 2.5G mirrors will, in turn, inform choices with respect to 3G mirrors. It is therefore too soon to argue for a large investment in scaling deposition tools alternative to elevated-temperature IBS for 2.5G and 3G mirrors. That said, as the funding trajectories and interferometer architectures become better defined, it will be important to regularly re-evaluate the current understanding of potential mirror technologies, and make critical decisions, especially with respect to deposition tools with long development times and requiring major financial investments.

\section{Recommendations}
\label{coatings_Recomm}

Continued and deeper coordination among all gravitational-wave collaborations will be critically important in developing coatings with the requisite 3G performance.

The additional funding provided by the CCR for U.S. efforts in coating research, combined with the previously existing NSF support, leave these efforts with reasonably adequate funding in the near future. It would be helpful to add another deposition group, as there is currently more capacity for characterization of properties (other than cryogenic mechanical losses) than for synthesis. It is important that the LIGO Lab continue with at least its current effort level, as their contribution to high-throughput mechanical loss characterization, optical scatter and homogeneity measurements, and their overall coordination of sample fabrication, distribution, and characterization is critical to the LSC efforts.
\magentacomment{Beverly: Is it practical at all to combine the groups with more samples with the groups with more characterizers? Should this be an explicit recommendation?}

In GEO, there is growing capacity for depositing coatings at Strathclyde, UWS and Hamburg. These coatings can be produced at a rate faster than it is possible to characterize their properties at cryogenic temperatures, and thus more resources devoted to such characterization are needed. 
% Currently, a novel type of ion-beam sputtering is being developed and tested at Strathclyde. While this is a promising research route, there are also 
At Strathclyde, here are plans to set up a large chamber with an industry ion source which should be capable of producing large and uniform enough coatings. This is an important priority in gaining access to more coating facilities which are suitable for the deposition of large, high-quality coatings. Initiating studies of GaP/AlGaP crystalline coatings, using hardware now installed in an MBE chamber at Gas Sensing Solutions Ltd, is also planned. This is an important parallel research direction to the development of amorphous coatings.

In VIRGO, LMA and University of Sannio are able to provide high quality coatings at a rate higher than the existing characterization capability in VCR\&D. 
% The ViSIONs project is investigating the role played by deposition parameters in determining the properties of the coatings.

At present, there is only one institution, LMA, capable of depositing coatings with the size and quality required by gravitational-wave detectors. LMA will continue to be supported as a research and coating facility for future gravitational-wave detectors by the French CNRS through EGO. Support for another institution, CSIRO in Australia, which was able to provide similar coating quality, was discontinued some time ago. It seems prudent to restart such a program, i.e., to support a second producer of large high quality coatings and to work with them on ongoing research efforts. 
% Within these activities, the products of other commercial vendors can also be characterized.  


%At present, there is only one group, LMA, capable of depositing coatings of a size and quality suitable for gravitational wave interferometers. While French CNRS through EGO is fully committed to continue supporting LMA as a research and coating facility for future GW detectors, a single source for any critical component is viewed as a potential risk to the whole community. There was a second system capable of such depositions, though subsequently defunded, at CSIRO in Australia. It seems prudent to re-establish this program, both to provide a second source for full-scale coatings, and as a significant contributor to the ongoing research efforts, as well as to certify additional commercial vendors. 

%\greencomment{Dave:This is an important recommendation.  I would even broaden it to say that not just CSIRO, but also new manufacturers should be certified.}
%\magentacomment{hal: mention the existence or lack thereof of commercial companies}

% While there is a good level of coordination within the LSC coating research programs, and within the VIRGO collaboration, the interaction between LSC and VIRGO groups has been less effective. Current efforts underway to establish an agreement on how to manage these interactions to enable more efficient exchange of information will be invaluable in generating synergy between the programs, and avoiding unnecessary duplication of efforts.
%\greencomment{Dave: I get why this is in here, but this seems too 'political' to me as written.  I would say that continued and deeper coordination among all GW collaborations will be critically important in developing coatings with the requisite performance.} \\
%\redcomment{Geppo: Paragraph removed}

\subsection{Roadmap}
Since the development of low-thermal-noise coatings is in the stage of research rather than development, even for 2.5G detectors, a conventional roadmap would not seem the best model for describing the path towards identifying suitable mirror designs and fabricating corresponding full-scale mirrors for 3G detectors. Noting that the architectures and operating parameters for 3G detectors remain in flux, that the results for mirrors developed for 2.5G detectors will have a strong, perhaps determinative, influence on the designs for 3G mirrors, and that the funding and thus construction schedules for 3G detectors are not yet clear, it is best to estimate schedule implications in terms of times before installation. Here we discuss aspects that will drive the necessary decisions, and estimate the time prior to anticipated installation in 3G detectors that various steps must be completed. 
The currently plausible approaches to 3G mirrors fall into three broad categories which have different cost and schedule drivers: amorphous coatings deposited by methods similar to conventional ion-beam sputtering (IBS), amorphous coatings deposited by alternative means, e.g. chemical-vapor deposition (CVD), and crystalline coatings. 
Amorphous coatings deposited by IBS methods: IBS is currently the only mature technology for depositing mirrors suitable for GW interferometers (GWI); the challenge is identifying suitable combination of materials and deposition conditions (rate, ion energy, substrate temperature, …) to meet mechanical loss and optical specs. Time required for this step is open-ended; a solution might be found next month, or might not exist at all. Once materials and conditions are determined, the time required to reach the production stage depends on the deviation from current IBS practice: for room-temperature deposition at conventional rates, perhaps a 1-2 year pathfinder would be adequate. For more extreme conditions of elevated substrate temperature, low rate, high ion energy, ion-beam assist, nanolayers, microwave annealing, … perhaps 3-5 years and USD 3-5M for developing suitable equipment and the pathfinder process might be required. Multi-material coatings would lie between these extremes of time and equipment cost. 
Amorphous coatings deposited by methods other than IBS: If research shows the optimum  deposition method is other than IBS, for example chemical vapor deposition (CVD) of a-Si/SiNx mirrors, in addition to open-ended research time similar to the IBS case, adequate time would be required to develop deposition tools and a scaled up process suited to GWI requirements. While CVD tools are widely used in the semiconductor industry, adaptation to GWI mirror requirements would be a significant effort. Instrumentation development and the more complex pathfinder process might require  USD 25-30M and perhaps 10 years. These figures are order of magnitude estimates that likely could be tightened up through discussions with equipment vendors. 
Crystalline coatings: For AlGaAs crystalline mirrors, the materials research and modeling phase for small scale (ca. 6 inch) coatings could be completed in perhaps three years. If these results showed high performance, the time and financial costs for substrate and tool development and the more complex pathfinder process to scale to gravitational-wave detector optics might be USD 25M and 10 years. 


 
%\section{References}
%References will be included in text once final length and content are determined
%\cite{Levin:Direct}


