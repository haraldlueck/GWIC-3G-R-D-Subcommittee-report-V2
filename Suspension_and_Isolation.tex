\chapterimage{Figures/Isolation.png} % Chapter heading image
% seismic isolation table 10m prototype AEI copyright AEI
\chapter{Suspensions and Seismic Isolation Systems}
%\section{Suspensions and Seismic Isolation Systems}
\label{sec:Suspensions_Isolation}

\vspace{-1cm}
\begin{samepage} % this vodoo is needed to stop Latex from skipping half a page

In this section we discuss suspensions and isolation systems for 3rd generation detectors. We address four areas: suspensions (especially the final stage), isolation, damping and control, and interface with cryogenics. The last two areas overlap Sections~\ref{sec:Sim_Controls} and~\ref{sec:Cryogenics} and thus are only covered briefly here.

\section{State of the art}
The use of fused silica fibers is a well-established technique for the final stage of the suspension of fused silica test masses, leading to a monolithic suspension which minimises suspension thermal noise. Currently there are four detectors operating with fused silica suspensions at room temperature: two Advanced LIGO detectors, Advanced Virgo and GEO600. Research is ongoing on silica within these collaborations.
The KAGRA project is pursuing the use of cryogenic sapphire suspensions (i.e. sapphire fibres supporting a sapphire test mass) working at about 20K. First full operation of KAGRA with its cooled sapphire suspensions is expected in 2019.
% Regarding seismic isolation and control, the 
Current detectors use different combinations of active and passive stages to achieve the necessary level of isolation and control. In Advanced LIGO, overall isolation is achieved using three sub-systems: the hydraulic external pre-isolator (HEPI) for low frequency alignment and control, a two-stage hybrid active and passive isolation platform and a quadruple pendulum suspension system that provides passive isolation above a few Hz and supports the test mass~\cite{Matichard_2015}. Advanced Virgo employs a combination of a tall inverted pendulum which is actively controlled, a passive seismic attenuation chain (the Super Attenuator), and a double pendulum supporting the test mass~\cite{AdvancedVirgo2015}. KAGRA also uses a passive attenuation chain supporting the cryogenic payload suspension system~\cite{KAGRA2013}. All detectors incorporate active damping for various resonances within their systems and their suspensions incorporate means to apply signals (magnetic, electrostatic) for global alignment and arm length control.

\section{Requirements, challenges and current/planned R \& D}
Suspension thermal noise and residual seismic noise are two of the dominant noise sources which limit the low frequency performance and define the low-frequency cut-off for ground based gravitational wave detectors. Thus the requirements of the suspension and isolation systems are to a great extent set by what the target low end of the operating frequency band is chosen to be, as well as by the intrinsic seismic levels of the chosen sites.
% , whether on or below ground. 
For reference, the design low frequency cut-off for aLIGO and AdVirgo is at $\sim$10\,Hz (see figure \ref{fig:3GSens}).

\end{samepage} % this vodoo is needed to stop Latex from skipping half a page


\noindent{\bf Suspensions, including cryogenic aspects} For future detectors at room temperature the aim will be to further reduce the suspension thermal noise. This is likely to involve suspension of heavy mirrors, up to several hundred kg, making the fibers as long as practicable, and making them relatively thinner (and thus requiring them to support higher stress) to push the bounce modes down in frequency and push the violin modes up~\cite{Heptonstall:2014, Bell:2014,aisa:2016advanced, Tokmakov:2012, Amico:2002_monolithic}.  Testing of individual elements and fully assembled prototypes will be required, as will upgrading current methods for pulling and welding fibers and of assembly procedures~\cite{Hammond:2014,Travasso:2018}. Ensuring that robust techniques are developed to handle the delicate fibers and heavier masses through assembly and installation will be an engineering challenge. For a general overview regarding reducing suspension thermal noise see~\cite{Hammond:2014, Hammond:2012}.

% For cryogenic operation, we note that 
The work of KAGRA is ground-breaking for the understanding and application of cryogenic techniques, and will lead to the first full stage sapphire suspensions operating at low temperature \cite{Kumar:2016_KAGRA}. The community will have a much better feel of where effort needs to be applied based on KAGRA's experience.
In general for detectors operating at cryogenic temperature, silicon or sapphire are the materials of choice for suspensions and test masses for low thermal noise, and these will have different challenges compared to the use of fused silica (see e.g., \cite{Cumming:2014Silicon, nawrodt:2013,Haughian:2016, Alshourbagy:2006_thermoelastic, Alshourbagy:2006,amico:2004, Cumming:2014Silicon, Alshourbagy:2005}). A significant challenge with 20\,K operation is the need to extract any deposited power via the fibers, which in turn drives their cross sectional area and vertical stiffness, and requires knowledge of their thermal properties. Suspending heavy mirrors with thick fibers/ribbons will need a smart design to soften the vertical and horizontal modes. Operation at 123\,K is less challenging for heat extraction since radiative cooling can be used.

All aspects of the monolithic assembly process will need development for 3G detectors. This includes hydroxy-catalysis bonding, which is already successfully used in GEO, LIGO and Virgo (silica-silica) and KAGRA, (sapphire-sapphire)~\cite{dari:2010, Amico:2002, vanVeggel:2014, Haughian:2016}. Properties of Si-Si bonds are being investigated, and indium or gallium bonding may also have applications in certain areas~\cite{Hofmann:2015, Murray:2015Low_Temp}.  Fiber fabrication of sapphire or silicon material with circular or ribbon geometries will be demanding and require significant R\&D: laser heated pedestal growth, micro-pull down, machining or etching are all possible techniques to be investigated~\cite{Cumming:2014Silicon, Alshourbagy:2005}. Assembly processes for sapphire and silicon including welding will need to be developed.

Other aspects which will need consideration include excess losses like clamping or bonding losses. The challenge is to have a suspension dissipation dominated by the material thermal noise and not by thermoelastic or other losses.
Simulations and modelling of suspensions will be important for understanding overall behavior, including dynamics of fiber suspensions, violin mode splitting and long term stability. Finite element analysis (FEA) is also an important tool as a cross check to design the best strategy to produce and realize the lower stage suspension \cite{Lorenzini:2010, Sorazu2017Sus}.
Consideration should also be put into upper stages of the suspensions to ensure they do not limit thermal noise performance of the final stage, for example due to noise from the maraging steel blade springs. Lower loss materials such as sapphire and silica could be used. Indeed KAGRA already incorporate sapphire springs at the final stage. Achieving a robust design with high breaking stress, low mechanical loss and good thermal conductivity, with the possible use of protective coatings are areas to be studied. 

As regards cryogenic operation, we have already noted that extracting power via heat conduction through the suspension elements is a major consideration for the design of a 20\,K detector. Finite element analysis with the various geometries, losses and thermal parameters will be valuable. The design of upper stages needs to be compatible with cryogenic operation and allow efficient heat extraction. A cryogenic suspension is by definition 'out of thermal equilibrium' and this needs to be evaluated. Cooling time is also a potential issue, as the timeline to commission such a detector may be driven by the several weeks to cool/warm up between vents. Currently there are efforts to work on mechanical heat links that can be removed. A cooling exchange gas is also a possibility, although may not be used due to concerns about residual vacuum level. Additionally, the development of vibration-free cryogenic suspensions is a pressing challenge. While results from KAGRA may help inform this work, it is deserving of more attention.

% Note that two of the above three are covered in the Cryogenics section

\noindent{\bf Isolation } For the 3G detectors, combinations of active and passive stages will be used to achieve the required isolation set by the site locations and the detectors' target sensitivities.
% and its low-frequency cut-off. 
% the required isolation, and thus how many stages are required, will depend on the site location and the target sensitivity and its low-frequency cut-off.
These considerations will also influence the overall height of the isolation systems and the number and type of stages it uses.
The lower seismic noise in an underground environment requires the use of less noisy inertial sensors to be used in loop for the control of the active platforms.
Additionally, increased vertical isolation will be needed for 3G since vertical to horizontal coupling increases with arm length. 
% {\bf , and note that increased vertical isolation for the longer arm lengths in 3G will be needed since vertical to horizontal coupling increases with arm length.} Materials and designs compatible with low temperature operation and good heat extraction will need to be specified and tested. For example passive damping materials will need to be investigated for cryogenic applications, since organic elastomers such as viton will not provide damping at low temperatures. Other considerations include thermal conductivity and thermal expansion, creep, stress and strain limits, internal friction, and acoustic emission under load. Suitable sensors and actuators for operation under cryogenic conditions, for both suspension and isolation systems, requires research and development.

More specifically, the ET design~\cite{ET2011} aims to reduce the low frequency sensitivity cutoff to 1.8\,Hz, in part using a longer (17\,m) superattenuator. However, it would be valuable either to reduce further this cutoff and/or to reduce the overall height of the vibration isolation system in order to save money for the realization of the underground caverns. Toward this end, a modified design of the superattenuator, using two cascaded inverted pendulums is being considered. On the other hand, it would be interesting, and open additional collaboration channels, to study the possibility of a merging of the technologies used so far by adVirgo and aLIGO. 
% This could open interesting collaboration perspectives inside the GW community.
The CE isolation system follows a scaled up approach from aLIGO and Voyager, with a more relaxed cut-off of 8\,Hz. For a 40\,km arm length detector such as CE, requirements for vertical seismic isolation will be particularly challenging to achieve, especially at low frequencies, requiring dedicated R\&D efforts.

\noindent{\bf Controls} The design of control systems for suspension and isolation systems has evolved significantly, and is expected to use modern controls that are now being tested, such as noise subtraction, automatic filter design and supervised machine learning and neural networks for feedback optimization. 
Noise subtraction is best done by directly actuating on the suspensions and seismic isolation systems; this needs careful design and modeling of actuation mechanisms for guaranteed dynamic range and low noise.  To achieve this, lower noise sensors may be required and the robustness of operation will have to be tested. Controls are discussed further in Section~\ref{sec:Sim_Controls}.

\section{Pathways, timeline, required facilities and collaborations}

Suspensions and isolation systems are a central part of the 3G interferometer designs and their R\&D must be mature roughly a decade before systems installation such that facilities and related subsystem choices can be made. It has typically taken 10-15 years to take suspension and seismic isolation hardware from prototype designs to interferometer installation. Thus prototypes for 3G suspension and isolation systems are an essential step to take in the near future. These can start with small scale (bench top) prototypes. However full-scale test facilities will also be needed. Upgrades to 2G facilities such as LASTI (MIT), the 40m detector (Caltech) and Gingin (Western Australia) can be used as test beds for 3G ideas. There is also a pressing need for new prototypes, especially for cryogenic testing, such as the new ET Pathfinder in Maastricht. Several small-scale collaborations across detector groups in different countries already exist and we expect these to continue.  Ideas and results can be shared and discussed at existing meetings such as GWADW or other meetings such as ET workshops. No other large collaborations are currently envisioned.

%\magentacomment{hal: timelines???}

%\section{References}
%This is a work in progress. In particular there are as yet no references for the isolation and controls sections.
%1) The LIGO Scientific Collaboration, Advanced LIGO, Class. Quantum Grav. 32 (2015) 074001

%2) Virgo suspension and isolation design reference

%3) Kagra suspension and isolation design reference

%4) Heptonstall:2014 Heptonstall, A. et al. (2014) Enhanced characteristics of fused silica fibers using laser polishing. Classical and Quantum Gravity, 31(10), p. 105006. (doi:10.1088/0264-9381/31/10/105006)

%5) Bell:2014 Bell, C. J., Reid, S., Faller, J., Hammond, G. D., Hough, J., Martin, I. W., Rowan, S. and Tokmakov, K. V. (2014) (need to get full info)

%6) aisa:2016advanced Aisa, D. et al. (2016) The Advanced Virgo monolithic fused silica suspension. Nuclear Instruments and Methods in Physics Research Section A-accelerators spectrometers detectors and associated equipment, 824

%7) Tokmakov:2012 Tokmakov, K.V., Cumming, A., Hough, J., Jones, R., Kumar, R., Reid, S., Rowan, S., Lockerbie, N.A., Wanner, A. and Hammond, G., (2012) A study of the fracture mechanisms in pristine silica fibres utilising high speed imaging techniques. Journal of Non-Crystalline Solids, 358(14), pp. 1699-1709. 

%8) Amico:2002_monolithic Amico, P., Bosi, L., Carbone, L., Gammaitoni, L., Marchesoni, F., Punturo, M., Travasso, F., Vocca, H. (2002) Monolithic fused silica suspension for the Virgo gravitational waves detector. Review of Scientific Instruments, 73(9). (DOI: 10.1063/1.1499540)

%9) Hammond:2014 Hammond, G., Hild, S. and Pitkin, M. (2014) Advanced technologies for future ground-based, laser-interferometric  gravitational wave detectors. Journal of Modern Optics, 61(Sup. 1), S10-S45. (doi:10.1080/09500340.2014.920934)

%10) Travasso:2018 Travasso, F. on behalf of Virgo Collaboration (2018) Status of the Monolithic Suspensions for Advanced Virgo. IOP Conf. Series: Journal of Physics: Conf. Series 957 (2018) 012012  (doi:10.1088/1742-6596/957/1/012012)

%11) Hammond:2012 Hammond, G.D., Cumming, A.V., Hough, J., Kumar, R., Tokmakov, K., Reid, S. and Rowan, S. (2012) Reducing the suspension thermal noise of advanced gravitational wave detectors. Classical and Quantum Gravity, 29(12), Art. 124009. (doi:10.1088/0264-9381/29/12/124009)

%12)Kumar:2016_KAGRA  R Kumar et al 2016 J. Phys.: Conf. Ser. 716 012017

%13) = 21) 

%14) nawrodt:2013 Nawrodt, R. et al. (2013) Investigation of mechanical losses of thin silicon flexures at low temperatures. Classical and Quantum Gravity, 30(11), p. 115008. (doi:10.1088/0264-9381/30/11/115008)

%15)  = 25) 

%16) Alshourbagy:2006_thermoelastic Alshourbagy, M. et al. (2006) Measurement of the thermoelastic properties of crystalline Si fibres. Classical and Quantum Gravity 23(8) (doi.org/10.1088/0264-9381/23/8/S35)

%17) Alshourbagy:2006 Alshourbagy, M. et al. (2006) First characterization of silicon crystalline fibers produced with the $\mu$--pulling technique for future gravitational wave detectors. Review of Scientific Instruments, 77(4) (doi.org/10.1063/1.2194486)

%18) amico:2004 Amico, P., Bosi, L., Gammaitoni, L., Losurdo, G., Marchesoni, F., Mazzoni, M., Parisi, D., Punturo, M., Stanga, R., Toncelli, A., Tonelli, M., Travasso, F., Vetrano, F., Vocca, H. (2004) Monocrystalline fibres for low thermal noise suspension in advanced gravitational wave detectors. Classical and Quantum Gravity, 21(5) (doi.org/10.1088/0264-9381/21/5/094)

%19)  = 21)

%20) Alshourbagy:2005 www.infn.it/thesis/PDF/getfile.php?filename=781-Alshourbagy-dottorato.pdf

%21) Cumming:2014Silicon Cumming, A.V. et al. (2013) Silicon mirror suspensions for gravitational wave detectors. Classical and Quantum Gravity, 31(2), 025017. (doi:10.1088/0264-9381/31/2/025017)

%22) dari:2010 Dari, A., Travasso, F., Vocca, H., Gammaitoni, L. (2010) Breaking strength tests on silicon and sapphire bondings for gravitational wave detectors. Classical and Quantum Gravity, 27(4). (doi.org/10.1088/0264-9381/27/4/045010)

%23) Amico:2002 Amico, P., Bosi, L., Carbone, L., Gammaitoni, L., Punturo, M., Travasso, F., Vocca, H. (2002) Fused silica suspension for the VIRGO optics: status and perspectives. Classical and Quantum Gravity, 19(7). (doi.org/10.1088/0264-9381/19/7/359)

%24) vanVeggel:2014 van Veggel, A.-M. A. and Killow, C. J. (2014) Hydroxide catalysis bonding for astronomical instruments. Advanced Optical Technologies, 3(3), pp. 293-307. (doi:10.1515/aot-2014-0022)

%25) Haughian:2016 K. Haughian et al., Mechanical loss of a hydroxide catalysis bond between sapphire substrates and its effect on the sensitivity of future gravitational wave detectors, Phys. Rev. D 94, 082003 – Published 12 October 2016

%26) Hofmann:2015 Hofmann, G. et al. (2015) Indium joints for cryogenic gravitational wave detectors. Classical and Quantum Gravity, 32(24), 245013. (doi:10.1088/0264-9381/32/24/245013)

%27) Murray:2015Low_Temp Murray, P. G., Martin, I. W., Cunningham, L., Craig, K., Hammond, G. D., Hofmann, G., Hough, J., Nawrodt, R., Reifert, D. and Rowan, S. (2015) Low-temperature mechanical dissipation of thermally evaporated indium film for use in interferometric gravitational wave detectors. Classical and Quantum Gravity, 32(11), 115014. (doi:10.1088/0264-9381/32/11/115014)

%28)  Lorenzini:2010 Lorenzini, M., Cagnoli, G., Campagna, E., Cesarini, E., Losurdo, G., Martelli, F., Vetrano, F., Vicere', A. (2010) The dynamics of monolithic suspensions for advanced detectors: A 3-segment model. J. Phys.: Conf. Ser. 228. (doi.org/10.1088/1742-6596/228/1/012017)

%29) Sorazu:2017Sus Sorazu, B. et al aLIGO suspensions characterisation https://dcc.ligo.org/LIGO-G1700038 

