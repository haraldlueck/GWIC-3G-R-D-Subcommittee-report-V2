%\chapterimage{Figures/balance.jpg} % Chapter heading image
% https://upload.wikimedia.org/wikipedia/commons/2/27/Bascula_9.jpg
\chapterimage{Figures/urmeter.jpg} % Chapter heading image
%https://www.lichtmikroskop.net/elektronenmikroskop/bilder/urmeter.jpg
\chapter{Calibration}
%\section{R\&D misc.}
\label{sec:Calibration}
%\section{Schumann resonances}
%Schumann resonances, global electromagnetic resonances propagating around the earth in the wave-guide formed by the earth and the ionosphere and excited by lightning discharges, can generate a correlated background in widely spaced gravitational wave detectors, which complicates the detection of a stochastic gravitational wave background. Dedicated detectors and models for the excitation and propagation of magnetic waves can be used to subtract the contributions from the gravitational wave signal. 

%Contacts (James Lough <james.lough@aei.mpg.de>, Yuki Inoue <iyuki@ncu.edu.tw>)
Extracting new science from the observed GWs requires accurate knowledge of the amplitude and timing of the signals. With high SNR detections expected in 3G detectors on the order of 1000, extremely low calibration uncertainty will be necessary for observations to be noise limited.

\section{Science driven calibration requirements}
%While the exact calibration requirements for the science we wish to do with 3G detectors is not well known, we can make some estimates. One might make the assumption that one would need something like 1/snr calibration accuracy, it's more complicated than this. The calibration uncertainty is frequency dependent. At some frequencies we don't know the calibration as well as other frequencies. At the same time, the signals we wish to analyze are frequency dependent. There are investigations underway to understand calibration requirements in a way that takes into account the signal, and the detector to determine required frequency dependent calibration uncertainty. With a detector network of 1 Einstein Telescope and 2 Cosmic Explorer detectors, we expect to see about 1 event per year of ~1000 SNR. At the highest SNRs, we can set the best limits on GR. To not be dominated by calibration error, we need to have calibration to within 0.5\% in amplitude.

The exact calibration requirements needed for the scientific objectives of 3G GW observatories are not yet known, but can be estimated. The calibration accuracy not only depends on the SNR of the signal, but both the detector calibration and the signal calibration are also frequency dependent. Studies are currently underway that take both aspects into account and evaluate the frequency response requirements of the calibration uncertainty.  With a detector network of three 3G detectors we expect about one event per year with SNR ~1000. The signals with the highest SNRs give the narrowest limits for GR. To avoid being dominated by calibration errors, we need a calibration \magentacomment{absolute or relative?} to an amplitude of 0.5\%.

The science we can do with detections in third generation detectors that will be limited by calibration include BNS tidal deformation, deviations from GR, and measurements of the Hubble constant. The first two look for deviations from a modeled waveform template, while the latter is based on absolute distance measurements.

There are two aspects of calibration uncertainty, the absolute uncertainty and the relative uncertainty. The first tells us how well we understand the total calibration in absolute numbers, while the latter is a frequency-dependent calibration uncertainty with respect to some fixed reference frequency. We have to consider each individually, depending on the physics we want to study. Looking for deviations from a modeled waveform template, we are mainly concerned with relative calibration uncertainty. In order to understand the absolute distance to the sources, we are most interested in absolute calibration.

\section{State of the art in calibration}
\subsection{Photon calibrators}
Photon calibrators (PCAL) provide a calibration reference starting from a traceable reference power meter. They use an amplitude modulated laser to apply a photon-recoil force to the test mass. For all large scale interferometric Gravitational Wave Detectors this is the current method of choice for absolute calibration. 
PCAL reference uncertainty is roughly at the 1\% level. This was specifically stated at 0.76\% for the O2 calibration uncertainty for the advanced LIGO detectors.\magentacomment{citation needed} This resulted in a frequency dependent uncertainty between 2\% and 4\% for O2\magentacomment{How does an absolute calibration uncertainty of 0.76\% result in a relative uncertainty of 2\% - 4\%? ,citation needed}. With improvements between O2 and O3 the expectation is that this could be reduced to \magentacomment{a relative uncertainty? of} 0.3\%.\magentacomment{citation needed.} 

\subsection{Newtonian calibrators}
This calibration method relies on gravitational interaction between an arrangement of rapidly rotating assymetrically arranged masses and one of the interferometer test masses. Adding a gravity field calibrator to the photon calibrator results in a theoretical absolute calibration uncertainty of 0.17\% .\cite{PhysRevD.98.022005}
advanced LIGO, advanced Virgo, and KAGRA are all developing Newtonian calibrator technologies.\cite{PhysRevD.98.022005,0264-9381-35-23-235009}

The design for the {\bf KAGRA} gravity field calibrator incorporates two distinct mass distributions in one rotating mass. One quadrupole and one hexapole, which provide a sort of self calibration of the distance to the test mass.\cite{PhysRevD.98.022005}\par
{\bf Advanced LIGO} is also working on a Newtonian calibrator for confirmation of the photon calibrator in a similar way to KAGRA.\magentacomment{citation needed} \par
{\bf Advanced Virgo} has recently published \cite{0264-9381-35-23-235009}  first tests of such a system and observed less than 1\% statistical uncertainty at some frequencies.\par
These Newtonian calibrator technologies are still in the very early stages of development, but there is significant work effort in this direction.

\subsection{Frequency based calibration}
Another calibration method would be to use frequencies as a reference. We can measure frequencies much more precisely than amplitudes. Free swinging Michelson is an example of this technique, though with arm cavities this method has to be done in steps and involves actuating the ITMs through the PUM stage which is weak and introduces more error than what is achievable with PCAL. For detectors without arm cavities (such as GEO 600) this method is much simpler and can result in very precise measurements of the absolute ESD actuator calibration, with statistical uncertainties of less than 0.1\%. \cite{Leong2012}

The arm cavity locking method outlined in the LIGO calibration for GW150914 paper is another example.
\cite{PhysRevD.95.062003}
This uses a frequency reference which in this case would be an RF LO to measure the beat frequency between the green arm locking signal and the frequency doubled infrared light.

