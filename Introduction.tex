\chapterimage{Figures/ArtisticView2.jpg} % Chapter heading image
%Einstein Telescope artits impression copyright Nikhef
%\section{Introduction}
\chapter{Introduction}
\label{sec:Intro}
\vspace{-1cm}
In this part of the report we review the Research and Development (R\&D) needed to construct and operate detectors with "third generation" or "3G" sensitivity. As with the first set of long baseline facilities, LIGO~\cite{AdvancedLIGO2015}, Virgo~\cite{AdvancedVirgo2015} and KAGRA~\cite{KAGRA2013}, the new 3G facilities will be designed to house a number of generations of detectors with increasing sensitivity as technology evolves and new ideas emerge. Chapter~\ref{sec:Fac_Inf} will cover the R\&D needed to select suitable sites and build such large, long-life-time facilities in a cost efficient way. The remainder of this part will focus on the R\&D required to deliver the first detectors operational in these 3G facilities.  Currently, there are two main concepts for these detectors, Einstein Telescope (ET)~\cite{ET2011}, a 10-km triangular underground detector, and Cosmic Explorer (CE)~\cite{CosmicExplorer2017}, a 40-km above-ground L-shaped detector. These concepts, along with the 2.5G detector LIGO Voyager~\cite{Voyager:Inst,VoyagerDCC2018} are highlighted in Box~\ref{Box:GWOs}.

The first instruments to be installed in the new 3G Observatories will be 10 to 20 times more sensitive than the current 
% second generation or 
"2G" instruments above 100\,Hz (Fig.~\ref{fig:3GSens}).  The improvement factor exceeds 100 by 10\,Hz and thousands below 10\,Hz. Despite their differences in design, the 3G detectors ET and CE rely on the same 'enabling technologies' -- the main pillars on which the predictions of sensitivity are based. These technologies are used to mitigate 'fundamental noise sources' affecting the instruments, in particular: \textbf{Quantum noise} associated with the laser light fields is modified by high laser power, quantum squeezing, massive mirrors and interferometer topology; \textbf{Thermal noise} in mirror substrates, coatings and suspensions is modified by temperature and material properties (and their behaviour as a function of temperature); and \textbf{Newtonian noise} caused by the gravitational forces of moving masses (such as air, the ground, and machinery) is modified by the location of the sites and subtraction schemes. 
% quantum noise, thermal (Brownian) noise and Newtonian noise. 
%Seismic noise and residual gas noise had been listed as fundamental noises, but as they are treatable by more expensive technology we do not list them as fundamental here.
   
These considerations are interdependent and have ramifications for the detector designs. Using low temperatures to reduce Brownian noise requires moving away from the fused silica optics used in Advanced LIGO and Advanced Virgo. Sapphire and Silicon are promising low temperature materials. Silicon would require changing the operating wavelength to 1.5 -- 2\,\micro m, necessitating the development of a new suite of light sources, optical components and detectors. Considering the material properties as a function of temperature and wavelength as well as the ability to handle high optical power while minimizing noise during heat extraction from the core optics leads to four possible operating temperatures:
room temperature and cryogenic temperatures 123\,K, 15\,K and below 5\,K.

%%%%%%%%%%%%%%%%%%%%%%%%%
%% BEGIN DETECTORS BOX
%%%%%%%%%%%%%%%%%%%%%%%%%

\begin{DetBox}{\bf Future Gravitational Wave Observatories}
\stepcounter{boxcount}
%see definition in structure.tex
\label{Box:GWOs}
\begin{tcolorbox}[standard jigsaw,colback=amber!10!white,colframe=red!70!black,coltext=black,size=small,  title=The Einstein gravitational--wave Telescope (ET)] 
%\begin{tcolorbox}[standard jigsaw,colback=amber!10!white,colframe=red!70!black,coltext=black, size=small, title=The Einstein gravitational--wave Telescope (ET)] 
\begin{wrapfigure}{r}{0.4\linewidth}
\vspace{-10pt}
\includegraphics*[width=0.4\textwidth]{Figures/ET_Thumb.png}
\label{fig:ET_Thumb}
\vspace{-25pt}
\end{wrapfigure}
ET~\cite{ET2011} is the European concept for a third generation gravitational-wave \emph{observatory}. To reduce the effects of seismic motion, the ET concept calls for the site to be located at a depth of about 100\,m to 200\,m below ground. In its final configuration it shall be arranged as an equilateral triangle of three interlaced detectors, each consisting of two interferometers. The configuration of each detector dedicates one interferometer (ET-LF) to detecting the \textbf{L}ow \textbf{F}requency components of the gravitational-wave signal (2--40\,Hz), while the other one (ET-HF) is dedicated to the \textbf{H}igh \textbf{F}requency components. Each interferometer will have a dual-recycled Michelson layout with Fabry-Perot arm cavities of about 10\,km arm length. In ET-LF, which operates at cryogenic temperature, thermal, seismic, gravity gradient and radiation pressure noise sources are particularly suppressed; in ET-HF, sensitivity at high frequencies is improved by high laser light power circulating in the Fabry--Perot cavities and the use of frequency-dependent squeezed light technologies.
\end{tcolorbox}


\begin{tcolorbox}[standard jigsaw,colframe=antiquefuchsia!80!black,colback=antiquefuchsia!20!white,opacityback=0.6,coltext=black,size=small, title=Cosmic Explorer (CE)] 
\begin{wrapfigure}{r}{0.4\linewidth}
\vspace{-10pt}
\includegraphics*[width=0.4\textwidth]{Figures/CE_Thumb.jpg}
\label{fig:CE_Thumb}
\vspace{-20pt}
\end{wrapfigure}
CE~\cite{CosmicExplorer2017} is a US concept envisioning an L-shaped above-ground observatory with 40\,km arm-length, operating a dual recycled Michelson interferometer with Fabry--Perot arm cavities. 
% CE~\cite{CosmicExplorer2017} is a US concept that envisions 40\,km long arms and new technologies to achieve a factor of 10 or more strain sensitivity improvement over the current generation of gravitational-wave detectors. 
% As with LIGO, it will employ a dual recycled Michelson interferometer with Fabry--Perot arm cavities and be implemented in stages. 
Its initial phase, called CE1, will employ scaled-up \emph{Advanced LIGO technology} including 320\,kg fused silica test masses, 1.4\,MW of optical power, and frequency-dependent squeezing. 
% Frequency-dependent squeezing will improve the quantum limited sensitivity by a factor of three. 
A major upgrade, CE2, will exploit the full potential of the new facility by using \emph{Voyager technology} such as silicon test masses and amorphous silicon coatings operating at 123\,K, with $1.5$ or $2\,\mu m$ laser light and 3\,MW of optical power in its arm cavities.

\end{tcolorbox}

%\begin{tcolorbox}[standard jigsaw,colframe=azure!70!black,colback=azure!20!white,opacityback=0.6,coltext=black, size=small, title= LIGO Voyager,sidebyside,righthand width=.3\textwidth,sidebyside gap=6mm,lower separated=false]
\begin{tcolorbox}[standard jigsaw,colframe=azure!70!black,colback=azure!20!white,opacityback=0.6,coltext=black, size=small, title= LIGO Voyager]

\begin{wrapfigure}{r}{0.4\linewidth}
\vspace{-10pt}
\includegraphics*[width=0.4\textwidth]{Figures/Voyager_Thumb.jpg}
\label{fig:Voyager_Thumb}
\vspace{-20pt}
\end{wrapfigure}

LIGO Voyager~\cite{Voyager:Inst,VoyagerDCC2018, VoyagerDCC2019} is the tentative concept for a new detector in the current LIGO observatory facilities designed to maximize the observational reach of the LIGO infrastructure and demonstrate the key technologies to be used for 3G observatories in new infrastructures.
Voyager would use heavy (ca.\,200\,kg) cryogenic mirrors with improved coatings and upgraded suspensions made of ultra-pure silicon at a temperature of 123\,K in the existing LIGO vacuum envelope and a laser wavelength of $\sim1.5\,-\,2\,\mu m$. 
A further factor of 3 increase in BNS range (to 1100\,Mpc) is envisioned along with a reduction of the low frequency cutoff down to 10 Hz. In the context of this report we use the term \emph{Voyager Technology} for this type of technology irrespective of plans to implement it in any existing or future infrastructure.
\end{tcolorbox}

\end{DetBox}
\newpage 

%%%%%%%%%%%%%%%%%%%%%%%%%
%% END DETECTORS BOX
%%%%%%%%%%%%%%%%%%%%%%%%%

\begin{wrapfigure}{r}{0.65\textwidth}
%\begin{figure}[ht]
\centering
\includegraphics*[width= 0.64\textwidth]{Figures/noises_percentiles-Voyager.pdf}
\caption{Target sensitivity curves for 3G gravitational-wave detectors Einstein telescope (ET)~\cite{ET2011}, shown in green, and Cosmic Explorer (CE)~\cite{CosmicExplorer2017}, shown in pink, compared with the design sensitivity of Advanced LIGO, shown in blue. 
%and the goal sensitivity for LIGO Voyager (orange curve). 
The shades of the curves represent sensitivity to sources with differently distributed locations.}
\label{fig:3GSens}
%\end{figure}
\end{wrapfigure}

Currently, Advanced LIGO and Advanced Virgo are in their third observing run. These detectors will soon thereafter be enhanced toward LIGO A+ and AdVirgo+ operations. Following these detector upgrades after a lead time of about 10 years we envision a network of new detectors, CE and ET, in new, longer-baseline 3G facilities, along with detectors that use 3G technology within 2G facilities, such as LIGO Voyager.
The key parameters of 2G and 3G instruments are summarized in Table~\ref{Tab:FutIfos}.
The 3G facilities are being designed with lifetimes up to 50 years and to be suitable for both initial instruments with sensitivities 10 times that of the advanced detectors and upgrades that may achieve another factor of 10.
Strategies for 3G will be modified according to observations made with current detectors, evolution in the science case, technology readiness and funds available. Nonetheless, this GWIC 3G report represents a milestone community vision for the future of ground based gravitational wave observations. 



%Two detector epochs are envisioned 
% post advanced detector baseline sensitivities 
%to follow the advanced detectors over the next 25\,years: upgrades in existing facilities; % operating at 1\,$\mu m$ wavelength; 
%and new detectors in new, longer baseline, third generation (3G) facilities. 

%The 3G facility infrastructure can to some extent be decoupled from the initial 3G instruments. 

% There are currently two conceptual designs for such 3G facilities:  The Einstein Telescope (ET) and Cosmic Explorer (CE). A short description is given in Box \ref{Box:GWOs} and the key parameters of 2G and 3G instruments are summarized in table \ref{Tab:FutIfos}.
%The Einstein Telescope (ET) is envisaged as an underground facility capable of housing three pairs of detectors in a nested triangular vacuum envelope with three 10\,km sides. Cosmic Explorer is proposed to be a single L-shaped facility with two 40\,km arms. The most effective 3G network would incorporate a pair of widely separated Cosmic Explorers working in unison with the Einstein Telescope~\cite{Hall:2019xmm}.

Planning for the 3G detectors began more than 20 years before they are envisioned to operate. This was based on experience with past and current detectors, for which there was a lead time of 15 years or more from conception to operation. Assuming a 2035 start date for initial 3G operations preceded by five years of construction and five years of commissioning, it is likely that only technologies with mature R\&D in 2025 will feed into final design and engineering for the initial 3G detectors. In this document we assess the state of R\&D for 3G enabling technologies toward readiness in 2025 while also looking ahead to the R\&D that will continue as a necessary preparation for subsequent 3G upgrades. 
% Prototyping such technology may include testing performance and science capability in an existing 2G facility. The term "Voyager" is used to describe a detector in a 2G facility that would prototype CE phase 2 technology whilst significantly improving detection range. 
Section \ref{sec:Fac_Inf} addresses requirements and design aspects of the 3G infrastructure.
In sections \ref{sec:Core_optics} to \ref{sec:Newtonian_Noise} we consider the state of the art in enabling technologies, the R\&D needed in each area, the level of resources needed (broadly bracketed as high, medium, low),
% \magentacomment{This is not implemented in the report} 
and how the R\&D should be organized and coordinated, in order to deliver fully tested subsystems for timely installation in new 3G facilities. It will also describe facilities needed for prototyping 3G technology including new test facilities and use of existing long baseline facilities. Sections~\ref{sec:Core_optics} and \ref{sec:Coatings} describe the closely related subjects of core optics and coatings, while Sections~\ref{sec:Cryogenics} and \ref{sec:Suspensions_Isolation} describe suspensions and isolation systems. A common theme among these chapters is thermal noise, described in Box~\ref{Box:Thermal}, which is the primary consideration for many choices in optics, coatings, suspensions, and operating temperature. Chapter~\ref{sec:Newtonian_Noise} describes Newtonian noise and its connection with facilities choices and reliance on modeling and subtraction schemes. In addition to the fundamental noises described above, a myriad of technical noise sources and control issues can limit interferometer performance: parametric instabilities, scattered light, and noise originating from auxiliary optics and control systems. The current state of the art in these areas, R\&D needed, and coordination for 3G are reviewed in sections \ref{sec:Aux-optics} and \ref{sec:Sim_Controls}. Finally, Section~\ref{sec:Calibration} describes plans for accurately calibrating the instruments to levels that will enable the dramatic science described in the first part of this report. 

\begin{table}[h]
\centering
\begin{tabular}{|l|l|l|p{1.6cm}|l|l|l|l|}
\hline
 % \multicolumn{8}{|c|}{Current and Future Interferometers}\\ 
 %\hline
 &aLIGO / AdV &A+/V+ &KAGRA &CE 1 &CE 2 &ET-LF &ET-HF\\
\hline
Arm Length [km] & 4 / 3 &4& 3& 40& 40& 10 &10\\
\hline
Mirror Mass [kg]& 40 / 42& 40& 23& 320& 320& 211& 200\\
\hline
Mirror Material& silica& silica& sapphire& silica& silicon& silicon& silica\\
\hline
Mirror Temp [K]& 295& 295& 20& 295& 123& 10& 290\\
\hline
Suspension Fiber& 0.6m/0.7m& 0.6m& 0.35m& 1.2m& 1.2m& 2m& 0.6m\\
& SiO2& SiO2&Al2O3&SiO2&Si&Si&SiO2\\
\hline
Fiber Type& Fiber& Fiber& Fiber& Fiber& Ribbon& Fiber& Fiber\\
\hline
Input Power [W]& 125& 125& 70& 150& 220& 3& 500\\
\hline
Arm Power [kW]& 710 / 700& 750& 350& 1400& 2000& 18& 3000\\
\hline
Wavelength [nm]& 1064& 1064& 1064& 1064& 1550& 1550& 1064\\
\hline
NN Suppression& 1& 1& 1& 10& 10& 1& 1\\
\hline
Beam Size [cm]& (5.5/6.2) / 6& 5.5/6.2& 3.5/3.5& 12/12& 14/14& 9/9& 12/12\\
\hline
SQZ Factor [dB]& 0& 6& foreseen& 10& 10& 10& 10\\
\hline
Filter Cavity & none& 300& unknown& 4000& 4000& 10000& 500\\
Length [m] &&&&&&&\\
\hline
\end{tabular}
\caption{Key parameters of the current Advanced detectors, their enhancements, LIGO A+ and adVirgo +, the two phases of Cosmic Explorer, CE1 and CE2, and the two interferometer types of Einstein Telescope.}
\label{Tab:FutIfos}
\end{table}
\begin{Infobox}{\bf Thermal Noise in Gravitational Wave Detectors}
\label{Box:Thermal}
%\paragraph{Infobox Thermal Noise in gravitational wave detectors}
Thermal noise is one of the fundamental noise limiting 2G detectors over a considerable frequency range. The main contributions come from Brownian noise of the mirror suspensions, substrates and coatings and thermo-optic (thermo-elastic plus thermo-refractive) noise of substrates and coatings. The relation between the dissipation and the power spectrum of the noise is described by Callen's Fluctuation-Dissipation Theorem~\cite{CaWe1951, Kubo:FDT, Callen:1959} and is given by:
\begin{equation}
S_x(\omega) = \frac{k_B T}{\omega^2} \left| Re \big[ Y(\omega) \big]\right| 
\end{equation}
\label{eq:FDT}
$k_b$ denotes the Boltzmann constant, $\omega$ the angular frequency, T the temperature and Y the mechanical admittance, defined as
\begin{equation}
Y(\omega) = i \omega\frac{X(\omega}{F(\omega)}
\end{equation}
where X($\omega$) and F($\omega$) are the Fourier components of the displacement of the system and force applied leading to the displacement, respectively. The real part of the admittance is proportional to mechanical losses, hence loss noise requires low mechanical losses.
Operating the mirrors and suspensions at reduced temperature reduces thermal noise since the displacement noise amplitude $x(\omega)$ scales with $\sqrt{T}$. More significantly, many material properties of mirror substrates (e.g., for sapphire and silicon) and coatings (e.g., AlGaAs, AlGaP, $\alpha$-Si) scale favorably with decreasing temperature. This makes the improvement in the noise substantially better than $\sqrt{T}$. Most glasses, e.g. fused silica, have increased mechanical losses at cryogenic temperatures, making them unsuitable for use. Further dependence of coating thermal noise on coating parameters in shown in Appendix~\ref{sec:Appendix_Coatings} equation~\ref{fig:Thermal_Noise}.
\end{Infobox}
%\end{tcolorbox}

