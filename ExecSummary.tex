\chapterimage{Figures1_3/summary1_3.jpg} % Chapter heading image
% Blackboard scribbles copyright Andreas Freise
\chapter{Executive Summary}
\label{sec:ExecSummary}
\pagenumbering{roman}% resets `page` counter to 1
\renewcommand*{\thepage}{ES\roman{page}}
%Charge: Coordination of the Ground-based GW Community R\&D: develop and facilitate coordination mechanisms among the current and future planned and anticipated ground-based GW projects, including identification of common technologies and R\&D activities as well as comparison of the specific technical approaches to 3G detectors. Including identifying primary (enabling or fundamental) and secondary (or technical) technologies.
%The next generation of gravitational-wave detector will require a tight coordination of R\&D topics. 
Third generation facilities will house detectors with sensitivities at least ten times better than the 2G detectors in the whole audio band (from about 10 Hz to about 10 kHz) and more than 10000 times better at lower frequencies (below 10 Hz).
%Third generation facilities will house detectors with sensitivities at least 10 times better than 2G detectors across the audio-band (from a few Hz up to  ca.\,10\,kHz), and more than 1000 times better at the lower frequencies (below 20\,Hz). 
As with 2G facilities, the 3G infrastructures will allow detectors of successive sensitivity levels  %changed from: will enable successive generations of detectors to be installed [strictly spoken a generation is 2G, 3G, etc.]
to be installed as new technologies and techniques mature.  This report presents the main technological challenges, approaches, timelines and where possible required decision points for the realisation of a 3G network. While there are different designs and implementations for the first 3G detectors (e.g. ET, CE), the enabling technologies (substrates, coatings, cryogenics, suspensions, Newtonian noise cancellation, lasers and quantum enhancement) are similar at the research level.  Timely progress in the development of these enabling technologies requires global collaboration and coordination. To accomplish the scientific program of the 3G network, a broad coherent detector R\&D program is now required, addressing key technological challenges in the next 5-7\,years.

Four areas in particular are of such a scale that global coordination will need to be accompanied by global R\&D funding: 

\begin{enumerate}
\item facility and vacuum infrastructure; 
\item substrates; 
\item coatings; 
\item large scale prototyping for demonstration of technology readiness. 
\end{enumerate}

\noindent Progress in areas 1, 2 and 3 will require significant involvement with industry, with some areas, e.g. coatings,  potentially requiring the field to build its own plants. Area 4 has seen recent growth, with the establishment of the 3G Pathfinder in Maastricht, but more prototyping facilities are likely to be required. Depending on R\&D progress, the first 3G detectors may adopt technologies proven in 2G facilities, scaled up to much longer baselines. Conversely, many 3G techniques may be tested or employed to improve the sensitivity of 2G detectors. 

\section*{Recommendations}
\begin{itemize}
\item \textbf{Recommendation 1}:  An international 3G R\&D coordination committee should be formed, with broad and inclusive membership representing GW groups across the world.

 A series of workshops on enabling technologies shall be held in order to stimulate exchange of ideas and allowing (if deemed useful) for coordination of the person-power intensive experimental activities.  Each of the major R\&D tasks should generate a list of  goals with quantitative metrics,  timelines and required resources.   Activities requiring global collaboration and coordination should be laid down and pathways identified.

\item \textbf{Recommendation 2}:  International consortia should be formed to work on key issues with industry partners, establish a governance and organisational structure with teeth (i.e. controls purse strings) and seek funding through joint proposals submitted across funding agencies.
\pagebreak
\item \textbf{Recommendation 3}: 
National funding agencies should take a proactive role in ensuring that the R\&D activities are well-focused and effective.  Coordination of funding plans through GWAC would be a first step, followed by supporting the role of the international 3G R\&D coordination committee (recommendation 1) in organizing the global effort.  This is likely to require increased funding for staff and instrumentation to enable the required long-term research programs at relevant laboratories and prototype interferometers.

\item \textbf{Recommendation 4}: Existing GW collaborations embrace 3G R\&D tasks and integrate it into their programs and deliverables, in order to ensure sufficient support for the long-term future of the field (as it has been done so successfully for 2G R\&D during the operation of the initial detectors). Over the next 5 years, mature 2G enabling technologies (e.g. 1064\,nm laser, fused silica optics, coatings) should be demonstrated to be up-scaled and ready for application in 3G facilities.



\end{itemize}

\section*{}
\begin{figure}[ht]
\centering
\includegraphics*[width= \textwidth]{Figures/3G_Readiness_Levelsblue.pdf}
\caption{Approximate timelines for the required maturity levels for 3G instruments and resource levels needed from now to installation of the first phase. The estimates for the required R\&D resource level shown on the right hand side only include investment (not R\&D person costs, assumed to be supplied form the labs) and are roughly categorized into \textit{low, mid} and \textit{high}.\\
}
\label{fig:maturity}
\end{figure}

An overview of required timelines to reach installation maturity is depicted in figure \ref{fig:maturity}. The figure shows required maturity levels for the various subsystems, depending on the foreseen time of installation and anticipated lead times. Infrastructure and facilities naturally will have to reach maturity earliest, followed by the other subsystems in the sequence of installation. The timings for the lower maturity levels ML1 - ML3 (relative to the highest level ML4) depend on the duration required between the individual steps; e.g. once technical readiness is achieved for core optics it still takes a few years to demonstrate full scale prototypes and manufacture the final optics substrates. Despite considerable differences for the different observatories (like ET and CE), large variations within the subsystems and an inevitable uncertainty in timelines, we summarise the timelines  for each subsystem in a single bar in figure \ref{fig:maturity}.  

The resources required to reach operational readiness are roughly divided into \textit{low, mid} and \textit{high}, with indicative financial investments for R\&D for the various subsystems over the entire period from now to the start of installation. 

The highest costs for the detector elements (as distinct from the civil and vacuum construction) are expected for developing the capabilities to manufacture the main optics (presumably fused silica and silicon) and for the development of coatings and coating facilities. 
Producing ultra-pure optics substrates of approx. 200-300\,kg weight requires an international effort and tight collaboration with industry. Developing coatings of outstanding optical quality and uniformity over the whole mirror surface, combined with the required low mechanical losses at  room temperature and cryogenic temperatures is currently regarded as the biggest hurdle to overcome for building 3G gravitational wave observatories. International collaboration and building redundancy in coating capabilities is essential for success.

In particular for underground infrastructures and facilities, R\&D will incur significant costs for exploration and prototyping. Exploratory efforts have already been started at the ET candidate sites. In the construction phase, building the infrastructure and facilities will be the biggest cost items and consequently have the largest cost saving potential. R\&D efforts to minimise costs while satisfying the strict technical demands is mandatory.


%\greencomment{Stan: Good recommendations, but they would be stronger if a path to getting them accomplished was included.  For example:  instead of number4, maybe something like:
%National funding agencies should take a proactive role in ensuring that the R\&D activities are well-focused and effective.  Coordination of funding plans through GWAC would be a first step, followed by supporting the role of the international 3G R\&D coordination committee (recommendation 1) in organizing the global effort.  This is likely to require increased funding for staff and instrumentation to enable the required long-term research programs at relevant laboratories and prototype interferometers.}

% \magentacomment{Josh: Here I pasted some remarks from quantum noise that Dave suggested be moved to general considerations.} 

% \begin{itemize}
% \item National funding agencies to increase funding (staff and instrumentation) of the relevant lab activities and to enable the required long-term research programs at the relevant prototype interferometers.
% \item Existing GW collaborations to take ownership of 3G R\&D and integrate it into their programs and deliverables, in order to ensure sufficient support for the long-term future of the field (as it has been done so successfully for 2G R\&D during the operation of the initial detectors). 
% \item Establishment of worldwide discussion and coordination forum (meeting several times a year) across all relevant collaborations, in order to stipulate exchange of ideas and allowing (if deemed useful) for a coordination of the person-power intensive experimental activities.
% \end{itemize}

