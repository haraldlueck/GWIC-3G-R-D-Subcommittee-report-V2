\chapterimage{Figures1_3/summary1_3.jpg} % Chapter heading image
% Blackboard scribbles copyright Andreas Freise
\chapter{Summary}
\label{sec:Summary}
\vspace{1cm}
Progress in the development of enabling technologies for 3G detectors will require global collaboration and coordination. A broad coherent detector R\&D program is needed now addressing key technological challenges in the next 5-7 years. In order to facilitate timely development, we propose four broad recommendations on top of the earlier presented subsystem specific recommendations.

\begin{itemize}
\item \textbf{Recommendation 1}:  An international 3G R\&D coordination committee should be formed, with broad and inclusive membership representing GW groups across the world. A series of workshops on enabling technologies shall be held in order to stimulate exchange of ideas and allowing (if deemed useful) for a coordination of the person-power intensive experimental activities.  Each of the major R\&D tasks should generate a list of  goals with quantitative metrics,  timelines and required resources.   Activities requiring global collaboration and coordination should be laid down and pathways identified.
\item \textbf{Recommendation 2}:  Following on from R1, international consortia should be formed to work on key issues with industry partners, establish a governance and organisational structure with teeth (i.e. controls purse strings) and seek funding through joint proposals submitted across funding agencies. 
\item \textbf{Recommendation 3}: National funding agencies should take a proactive role in ensuring that the R\&D activities are well-focused and effective.  Coordination of funding plans through GWAC would be a first step, followed by supporting the role of the international 3G R\&D coordination committee (recommendation 1) in organizing the global effort.  This is likely to require increased funding for staff and instrumentation to enable the required long-term research programs at relevant laboratories and prototype interferometers.
\item \textbf{Recommendation 4}: Existing GW collaborations embrace 3G R\&D tasks and integrate it into their programs and deliverables, in order to ensure sufficient support for the long-term future of the field (as it has been done so successfully for 2G R\&D during the operation of the initial detectors). Over the next 5 years, mature 2G enabling technologies (e.g. 1064\,nm laser, fused silica optics, coatings) should be demonstrated to be up-scaled and ready for application in 3G facilities.
\end{itemize}
