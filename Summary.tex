\chapterimage{Figures/summary.jpg} % Chapter heading image
% Blackboard scribbles copyright Andreas Freise
\chapter{Summary}
\label{sec:Summary}


Prioritize R\&D items by timelines and complexity?
\begin{itemize}
\item Infra and Fac most urgent, solutions known, large cost saving potential), 
\item Coatings and 3G optics (no clear path forward but a little less time critical), 
\item simulation (on a good path, adequate level of activities ongoing and globally organised), 
\item controls (regained attention from efforts to push current detectors to lower frequencies where control noises dominate; benefiting from another wave in machine learning activities throughout different engineering and science sectors; not time critical for 3G but potentially very rewarding for 2G+ detectors), 
\item Quantum enhancements (squeezing work considered to be understood and mostly technologically limited now; beneficial for 2G+ and
\end{itemize}
Provide some kind of Matrix with technical readiness requirements vs. time?
See figure \ref{fig:maturity}
\begin{figure}[h]
\centering
\includegraphics*[width= \textwidth]{Figures/3G_Readiness_Levels.pdf}
\caption{Required and expected technical readiness levels for various subsystems.\\
OR required technical readiness levels for ET and CE\\
or something similar\\
Technical readiness Levels: TR1: xxx, TR2:yyy, TR3:zzz, TR4: implementation}
\label{fig:maturity}
\end{figure}

\magentacomment{Josh: Here I pasted some remarks from quantum noise that Dave suggested be moved to general considerations.} 

\begin{itemize}
\item National funding agencies to increase funding (staff and instrumentation) of the relevant lab activities and to enable the required long-term research programmes at the relevant prototype interferometers.
\greencomment{Dave: Certainly.  But this is a generic recommendation in the sense that all R\&D 3G efforts need more R\&D funding.  I don't think this needs to be here (or in any other chapter), but somewhere in the main set of recommendations.}
\item Existing GW collaborations to take ownership of 3G R\&D and integrate it into their programs and deliverables, in order to ensure sufficient support for the long-term future of the field (as it has been done so successfully for 2G R\&D during the operation of the initial detectors). 
\item Establishment of worldwide discussion and coordination forum (meeting several times a year) across all relevant collaborations, in order to stipulate exchange of ideas and allowing (if deemed useful) for a coordination of the person-power intensive experimental activities.
\end{itemize}

\section{collection of recommendation section from the various chapters}
\subsection{Facilities + Infrastructure}
collaboration with high energy particle physics community (know-how in building large vacuum
systems from accelerator facilities)

3G Infrastructure and facilities construction will start roughly 5 years before the installation of detector hardware, hence the timescales for technical readiness are shorter and more critical here.
If site preparation for the Einstein telescope is to start as early as $\sim$\,2025 the respective R\&D and detailed technical studies have to start immediately. Respective international and collaboration overarching working groups should be initiated right now.

\subsection{Core Optics}
There is not a research laboratory that is dedicated to develop and optimize the large size crystal growth with the specifications imposed by the GW detectors. This could have a severe impact on the development of 3G detectors.
\magentacomment{hal: The lead times for this R\&D are huge. Steering the direction needs decision points well ahead of time. }
timelines missing
\subsection{Coatings}
By far the most pressing timeline is for the enhanced 2G detectors. In the case of a+LIGO, allowing for a one-year pathfinder after the coating material and process are identified, the required research to identify a suitable mirror coating should be completed by May 2020. This timeline implicitly assumes that the coating will be a sputter-deposited amorphous doped oxide, perhaps deposited at an elevated temperature, a slower than conventional rate and/or with a higher than conventional annealing temperature. It is unlikely that there is time in such a schedule to identify coatings and develop the required equipment for a process with more significant deviations from current deposition methods.

In recognition of the fact that meeting the enhanced 2G detector timeline requires doing basic research on a development schedule, the efforts of the LSC community are of necessity focused on these mirrors. That said, it is also recognized that the community shouldn't put itself again in such a situation, so a portion of the current research effort is devoted to establishing approaches to mirrors for 2.5G and 3G detectors. It is difficult to set research timelines for this effort, since the funding and construction schedules for these systems are not yet established. Another open question is the deposition process that will be required for the 2.5G and 3G mirrors; the further that process is from conventional IBS, the longer it is likely to take to develop suitable tooling. It seems that in any plausible scenario at least five years are available for research into the best approach to mirrors meeting 2.5G requirements. Results for these 2.5G mirrors will, in turn, inform choices with respect to 3G mirrors. It is therefore too soon to argue for a large investment in scaling deposition tools alternative to elevated-temperature IBS for 2.5G and 3G mirrors. That said, as the funding trajectories and interferometer architectures become better defined, it will be important to regularly re-evaluate the current understanding of potential mirror technologies, and make critical decisions, especially with respect to deposition tools with long development times and requiring major financial investments.

\subsubsection{Recommendations}

\noindent The additional funding provided by the CCR for U.S. efforts in coating research, combined with the previously existing NSF support, leave these efforts with reasonably adequate funding in the near future. It would be helpful to add another deposition group to the effort, as there is currently more capacity for characterization of properties (other than cryogenic mechanical losses) than for synthesis. It is also important that the LIGO Lab continue with at least its current effort level, as their contribution to high-throughput mechanical loss characterization, optical scatter and homogeneity measurements, and their overall coordination of sample fabrication, distribution, and characterization is important to the LSC efforts.

In GEO, there is growing capacity for depositing coatings at Strathclyde, UWS and Hamburg. These coatings can be produced at a rate faster than it is possible to characterize their properties at cryogenic temperatures, and thus more resources devoted to such characterization are desirable. Currently, a novel type of ion-beam sputtering is being developed and tested at Strathclyde. While this is a promising research route, there are also plans to set up a large chamber with an industry ion source which should be capable of producing large and uniform enough coatings. This is an important priority in gaining access to more coating facilities which are suitable for the deposition of large, high-quality coatings. Initiating studies of GaP/AlGaP crystalline coatings, using hardware now installed in an MBE chamber at Gas Sensing Solutions Ltd, is also planned. This is an important parallel research direction to the development of amorphous coatings.

In VIRGO, the activities are progressing at the pace compatible with the funding available from other projects. \magentacomment{Josh: I do not understand what the previous sentence intends to say.} LMA and University of Sannio are able to provide high quality coatings at a rate higher than the existing characterization capability in VCR\&D. The project ViSIONs takes care of only one specific aspect of the research, that is the impact and the understanding of deposition parameters on the coatings properties.

At present, there is only one institution, LMA, capable of depositing coatings of a size and quality suitable for gravitational wave interferometers. LMA will continue to be supported as a research and coating facility for future GW detectors by the French CNRS through EGO, while support for another institution, CSIRO in Australia, which was able to provide similar coating quality, was discontinued some time ago. It seems prudent to restart this program to create redundancy through a second source of high quality coatings and to have another significant contributor to ongoing research efforts. Within these activities, the products of other commercial vendors can also be characterized. 
Continued and deeper coordination among all GW collaborations will be critically important in developing coatings with the requisite 3G performance.

\subsubsection{Roadmap}
Since the development of low-thermal-noise coatings is in the stage of research rather than development, even for 2.5G detectors, a conventional roadmap would not seem the best model for describing the path towards identifying suitable mirror designs and fabricating corresponding full-scale mirrors for 3G detectors. Noting that the architectures and operating parameters for 3G detectors remain in flux, that the results for mirrors developed for 2.5G detectors will have a strong, perhaps determinative, influence on the designs for 3G mirrors, and that the funding and thus construction schedules for 3G detectors are not yet clear, it is best to estimate schedule implications in terms of times before installation. Here we discuss aspects that will drive the necessary decisions, and estimate the time prior to anticipated installation in 3G detectors that various steps must be completed. 
The currently plausible approaches to 3G mirrors fall into three broad categories which have different cost and schedule drivers: amorphous coatings deposited by methods similar to conventional ion-beam sputtering (IBS), amorphous coatings deposited by alternative means, e.g. chemical-vapor deposition (CVD), and crystalline coatings. 
Amorphous coatings deposited by IBS methods: IBS is currently the only mature technology for depositing mirrors suitable for GW interferometers (GWI); the challenge is identifying suitable combination of materials and deposition conditions (rate, ion energy, substrate temperature, …) to meet mechanical loss and optical specs. Time required for this step is open-ended; a solution might be found next month, or might not exist at all. Once materials and conditions are determined, the time required to reach the production stage depends on the deviation from current IBS practice: for room-temperature deposition at conventional rates, perhaps a 1-2 year pathfinder would be adequate. For more extreme conditions of elevated substrate temperature, low rate, high ion energy, ion-beam assist, nanolayers, microwave annealing, … perhaps 3-5 years and USD 3-5M for developing suitable equipment and the pathfinder process might be required. Multi-material coatings would lie between these extremes of time and equipment cost. 
Amorphous coatings deposited by methods other than IBS: If research shows the optimum  deposition method is other than IBS, for example chemical vapor deposition (CVD) of a-Si/SiNx mirrors, in addition to open-ended research time similar to the IBS case, adequate time would be required to develop deposition tools and a scaled up process suited to GWI requirements. While CVD tools are widely used in the semiconductor industry, adaptation to GWI mirror requirements would be a significant effort. Instrumentation development and the more complex pathfinder process might require  USD 25-30M and perhaps 10 years. These figures are order of magnitude estimates that likely could be tightened up through discussions with equipment vendors. 
Crystalline coatings: For AlGaAs crystalline mirrors, the materials research and modeling phase for small scale (ca. 6 inch) coatings could be completed in perhaps three years. If these results showed GWI level performance, the time and financial costs for substrate and tool development and the more complex pathfinder process to scale to GWI size optics might be USD 25M and 10 years. 


\subsection{Cryogenic}
\subsubsection{Pathways, required facilities and collaborations}
%
For each operating temperature (below 20\,K, 123\,K)


\begin{enumerate}
\item demonstrate reliable and timely  heat extraction and required final temperature stability (varies depending on goal material parameters).
\item demonstrate integration with SIS   (Nb: requirements differ for CE and ET).
\item demonstrate integration with other subsystems.  eg auxiliary optics  for  wave front control .
\item system integration and performance  testing on large scale prototypes/detectors.
\end{enumerate}

A key element of any cryogenic implementation is a refrigerator that is as quiet and vibration free as possible. R\&D devoted to design and construct silent refrigerator machine is required.  Several ideas have been proposed, for example the use of PT cryocoolers with symmetric cold heads.  Collaboration with cryogenics industry and the accelerator community is strongly recommended. 

For sapphire core optics at 20\,K:   Items 1 and 3 have been/will be demonstrated using KAGRA.  As the current sensitivity goal of KAGRA is modest at low frequencies, it is not clear how well KAGRA performance will inform item 2 and 4.  May occur with KAGRA upgrade.

Silicon core optics at 18\,K, 123\,K : Items 1-3:  a number of facilities are planned or under construction.  Existing facilities  at Stanford, Caltech and Gingin are currently being reconfigured. A  new facility at Maastricht in Europe has been designed to investigate  both 123K and  below 20\,K. Much of the required technology for cryogenic operation is also needed in the near term for  exploration of  the material properties of mirrors, coatings, etc.  \textit{Tight coordination is needed between these groups to  ensure optimum focus and timely progress.}

Ultimately full system integration tests (Item 4) need to be done on large, sensitive, prototypes, demonstrating sustained high performance.  The prototypes need to be large enough to allow reliable scaling to full size 3G systems.  \textit{\textbf{This is a major undertaking and global collaboration is needed to resource, develop and manage these prototypes}}.  These could be new facilities, or,  potentially, an existing 4\,km facility(ies) could be re-purposed.  This would not only demonstrate technical readiness but would produce detector(s) with a significantly extended range.  This is effectively the Voyager concept.

Cooperation with high-energy particle physics, where comparable requirements are addressed, could create valuable synergies. Furthermore, the use of cryogenics in the electronics and the magnetic actuators\cite{cryo:OSEM} has the potential to be of great benefit. These techniques should be explored in parallel with the more fundamental cryogenic interferometry.

\subsubsection{Roadmap}

\textit{Sapphire at 20\,K}


\begin{itemize}
\item  2019-2026  	Use KAGRA to diagnose  feasibility and performance.
\item  2027  		First Cryogenic downselect.  
\item  2028-2032  		If GO:  final design and fabrication.
\end{itemize}



\textit{Silicon at 	123\,K}
•  \begin{itemize}
\item 2019-2025  	R\&D in university labs on 1. and 3.
\item 2022-2028	Integration testing with SIS and other subsystems  (item 2).
\item 2028……  	Cryogenic downselect
\item 2029-2033	Large  Phase noise prototype
\item 2034-2039		If GO:  final design and fabrication.
\end{itemize}



\textit{Silicon at 	18\,K}
\begin{itemize}
\item SAME CYCLE AS SILICON AT 123\,K.
\end{itemize}

Note that it is feasible that both Silicon 123\,K and Silicon 18\,K cryogenic subsystems will be deployed at different 3G facilities.  Depending on facility finding and  technology readiness, it is  feasible, perhaps even likely,  that 3G facilities will initially operate room temperature detectors whilst cryogenic development continues.

\subsection{Newtonian Noise}
The main facilities of NN R\&D are the natural environments of 3G detector candidate sites. In addition, since we have very little experience in observing and modelling underground seismic and sound fields, underground studies done anywhere can provide important information. There are significant differences between sites in terms of sources of environmental disturbance, local geology and topography, so transferring results from one site to another should be done with caution.

Finite-element simulations of environmental fields and algorithms for the optimization of array configurations for noise cancellation are computationally expensive. Significant computing time on clusters will be required. 

Collaboration between groups is strongly recommended with respect to site characterization. It requires substantial expertise to set up robust, high-quality environmental measurements, to understand, which properties of environmental fields need to be understood, and how to analyze environmental data. Collaboration is also encouraged between groups working on numerical simulations to share generic knowledge, e.g., of how to assess accuracy of numerical simulations, e.g., by comparing simulations using different software, or by running simple simulations that allow comparison with analytic models.
\subsection{Light Sources}
Like in the past a strong collaboration between a group within the GWD community and a laser research lab or industry is required (like e.g.  AEI/LZH, Artemis/Alphanov, ICRR/Mitsubishi) to design and build suitable HPLs at the different wavelengths. As the different wavelengths need different solutions a loose collaboration between the respective wavelength groups would be advantageous. It would be desirable to have at least two collaborations to work on laboratory prototype solutions for each wavelength to explore different concepts and alternative technical solutions. These groups should have a strong connection with regular meetings to exchange results and ideas. This approach would possibly avoid a single supplier problem.
No particular collaborations are required for the squeezing research. The normal exchange of concepts and results at collaboration meetings and conferences seems sufficient. 

\subsubsection{Road map}
\subsection*{HPL 1064\,nm}
\begin{itemize}
	\item 2019 - 2020 : continue development and reliability studies of 2G PSL systems at the 250\, W level and perform coherent combination demonstration experiments at high powers
	\item 2021 - 2024 : engineering prototype (see section \ref{sec:pathway}) 500\,W HPL and conceptual test of stabilization and spatial filter solutions
	\item 2024 - ... : final design and fabrication of HPL with the initial 3G requirements within specific 3G GWD project, in parallel R\&D on path toward the final 3G requirements 
\end{itemize} 


\subsection*{HPL ${\bf 1.5 - 2.1 \, {\bf \mu m}}$}
\begin{itemize}
	\item 2019 - 2021 : identify concepts for a 1\,kW HPL that fulfills the stringent 3G GWD HPL requirements (most likely several coherently combined stages)
	\item 2022 - 2024 : 1\,kW HPL functional prototype phase (see section \ref{sec:pathway})
	\item 2024 - 2028 : 1\,kW HPL engineering prototype phase (see section \ref{sec:pathway})
	\item 2028 - ... : final design and fabrication of HPL with the initial 3G requirements within specific 3G GWD project, in parallel R\&D on path toward the final 3G requirements
\end{itemize} 


\subsection*{squeezed light sources for 3G GWDs}
\begin{itemize}
	\item 2019 - 2026 : continue laboratory based R\&D on squeezed light sources at all wavelength, potentially involve industrial partners to design and fabricate low loss optical components
	\item 2026 - ... : final design and fabrication within specific 3G GWD project
\end{itemize}
\subsection{Quantum enhancements}
A significant R\&D effort is required to make an informed selection of the optimal quantum noise reduction strategies for 3G detectors and to enable their risk free application to maximise the science return of 3G observatories. Additionally, we recommend the development of unified classification scheme of QND techniques and common approach to analysing and comparing their performances.  

\magentacomment{hal: timelines???}

\subsection{Suspensions and Seismic Isolation Systems}

It is important to note that it has typically taken 10-15 years to take suspension and seismic isolation hardware from prototype designs to interferometer installation. Thus prototypes for 3G suspension and isolation systems are an essential step to take in the near future. These can start with small scale (bench top) prototypes. However full-scale test facilities will also be needed. 2G facilities such as LASTI (MIT), the 40m detector (Caltech), Gingin (Western Australia)
% (others e.g. in Virgo and Kagra) 
or upgrades to those can be used as test beds for 3G ideas. There is also a pressing need for new prototypes, especially for cryogenic testing, such as ``ET pathfinder."

\subsubsection{Type of collaboration required and suggested mechanisms}

Several small-scale collaborations across detector groups in different countries already exist and we expect these to continue.  Ideas and results can be shared and discussed at existing meetings such as GWADW or other meetings such as ET workshops. No other large collaborations are currently envisioned.

\magentacomment{hal: timelines???}
\subsection{Auxiliary Optics}

\paragraph{\bf Light sources} Many of the auxiliary optics subsystems will be strongly impacted by the choice of laser wavelength. Beyond this, the input optics has a direct interface with the light source, and so information exchange between groups performing R\&D in these two areas will be critical.
\subsubsection{\bf Quantum noise} The output optics subsystem will be closely linked to the quantum noise working groups as squeezing is foreseen in all future detectors. Close collaboration between these groups will therefore be beneficial.

\magentacomment{hal: timelines???}
\subsection{Simulation and Controls}
\subsubsection{roadmap Sim}
Simulation tools are strongly defined through the context in which they are used. Many of the priorities for the research and development directly translate into a priority task for modelling work or a required effort in developing new capabilities for simulation tools. In particular the progress in modelling tools or modelling itself is not limited by development of technologies or methods, but by the available person power.

%%% -------------------------------------------------------------------------
\subsubsection{Recommendations Sim}
To address the above challenges, the current portfolio of software tools must be updated, either by extending the existing software or by providing new dedicated tools. The detailed list of code changes or required features goes beyond the scope of this document. The following are recommendations for the higher-level actions to support an effective and open environment for this effort.

\noindent{\textbf{Additional software}}
Most of the required modelling tasks can be performed by extending and updating the available tools, some of which is already well underway. However some missing functionality might be better achieved by developing new software. Needed packages include an easy to use and flexible time-domain simulation, a 3D beam tracing software dedicated to ground based detectors and a comprehensive modelling software for various suspension systems capable of computing the thermal noise of all elements. In addition, commercial tools should be reviewed to understand when these are superior to custom made tools, for example, for more common optics tasks such as modelling stray light.

\noindent{\textbf{Resources}}
We encourage institutions to increase support for the development and use of simulation tools over the next 5 years, a crucial time for the design of 3G instruments, and for upgrades to current detectors.
%, while the commissioning of detectors will push the limits of current technology. 
Experience has shown that
individual post-docs and PhD students can provide effective tools that are quickly adopted and used by a wide community. However, those tools often come only with rudimentary or outdated documentation and code reviews or formal testing of simulation results are not common practice. We recommend additional, dedicated post-doc support in this area to mitigate this.

\noindent{\textbf{Collaboration and coordination}}
Coordination between research groups and projects is important in three ways:
code development is often done by few individuals in each project who will benefit from having a forum to discuss priorities as well as technical challenges. The same is true for the collaboration between people
%undertaking the code development and those 
doing modelling, often 
%Modelling tasks (design or commissioning) are often 
investigating a new or not-understood behaviour;
collaboration with other scientists investigating similar or related aspects has been shown to greatly reduce the time to reaching a conclusion. We recommend to continue (or establish) working groups dedicated to the interferometer modelling within projects and 
%for these working groups 
to organize workshops/meetings adjacent to international meetings.

\noindent{\textbf{Software and code distribution}}
Accessibility of the software, and ideally the source code, should be improved. Each software tool should have: a) an active maintainer who is responsible for the current code base and who can be identified and reached by any users of the software, b) a single, discoverable web page hosting the master version of the tool or code under a clear and permissible software license, c) documentation about the implemented models and their limitations, including descriptions of mathematical algorithms, or parameter sets used and d) training material, such as a set of examples and tutorials for new users, especially graduate students. Where possible (without breaking existing compatibility) the adoption of common  input and output formats, for example, for files storing interferometer parameters  should be encouraged.

The effectiveness of specific tools is often not defined by the a single feature but by a network effect based on many factors. Any tool benefits greatly from a large user base, for example, through receiving bug reports and the availability and diversity of examples as well as experts on how to use the specific tool. And the impact of a modelling tool is improved significantly by a strong track record and trust by the wider community. We recommend that software maintainers adopts the aim of making the software as accessible as possible without compromising its core functionality. %Similarly we recommend that groups distributing software make an effort in providing training material or training opportunities for new users. The quality and effectiveness of simulation tools can be greatly enhanced by a large and informed active user base.

Those software tools that aim at providing standard results for the wider community must also provide the official data sets or model files, such as, for example default LIGO models for \textsc{Finesse}. The GWINC software package should provide standard noise budgets for all envisaged detectors and the
collaborations should establish a mechanism to review its noise models by the international community.

\noindent{\textbf{Source code maintenance}}
Some important simulation tools date back to their original development in the 80s and 90s. Given the rapid development of computing and computer languages, it could be beneficial to re-implement these in modern frameworks. At the same time the experience and knowledge acquired with the originals should not be lost in the process. A careful development process and code design that finds a balance between modern technology and backwards compatibility
is recommended. Some codes, such as \textsc{Finesse} and GWINC are currently undergoing such a process. Other software should be reviewed for similar updates.

\subsubsection{Roadmap Controls}
Similarly to simulations, progress in the development of the required controls is not defined by future technical breakthroughs in the topic itself. Instead we can adapt techniques that have been developed in other areas, such as modern control techniques. Simulation tools can play a crucial role for designing control techniques and they should be further developed or extended for this purpose.
More advanced techniques, such as adaptive control or machine learning, might need different and more powerful hardware to run, for example dedicated FPGAs or GPUs. 
It is important to advance the control design compared to other subsystems, so that controls can inform the overall system design, which allows to avoid mistakes that make the control harder or impossible. Prototype interferometers have to be used for the experimental validation of control schemes when these incorporate a more fundamental change, whereas other techniques might be implemented and tested directly in current detectors.


\subsubsection{Recommendations Controls}
Control noise, or cross-coupling of fundamental noise through control channels, has historically limited GW detectors, especially at the low-frequency end of the measurement band. The requirements of 3G detectors will pose a great challenge to control systems. Control schemes should be developed and tested early as an integral part of detector designs. Synergies with commissioning and detector upgrades should be sought out. At the same time, care should be taken that sufficient theoretical work and tests at prototype interferometers are performed for new control designs that cannot be tested with current detectors.

\subsection{Calibration}

Requirements on the absolute and relative calibration for third-generation detectors are not yet fully known, and dedicated modeling efforts are needed to understand the constraints set by various 3G scientific objectives. 3G detectors will likely require well sub-percent amplitude calibration, potentially necessitating R\&D to improve the capabilities of the current calibration apparatuses; such R\&D is already underway and cooperation activities between the different GW collaborations are considered to be at an adequate and sufficient level. 

\chapterimage{Figures/summary.jpg} % Chapter heading image
% Blackboard scribbles copyright Andreas Freise
\chapter{Executive Summary}
\label{sec:ExecSummary}
\pagenumbering{roman}% resets `page` counter to 1
\renewcommand*{\thepage}{ES\roman{page}}
%Charge: Coordination of the Ground-based GW Community R\&D: develop and facilitate coordination mechanisms among the current and future planned and anticipated ground-based GW projects, including identification of common technologies and R\&D activities as well as comparison of the specific technical approaches to 3G detectors. Including identifying primary (enabling or fundamental) and secondary (or technical) technologies.
%The next generation of gravitational-wave detector will require a tight coordination of R\&D topics. 

Third generation facilities will house detectors with sensitivities at least 10 times better than the 2G detectors across the audio-band (from about 10 Hz up to ca. 10 kHz), and more than 1000 times better at the lower frequencies (below 10 Hz).
%Third generation facilities will house detectors with sensitivities at least 10 times better than 2G detectors across the audio-band (from a few Hz up to  ca.\,10\,kHz), and more than 1000 times better at the lower frequencies (below 20\,Hz). 
As with 2G facilities, 3G infrastructure will enable successive generations of detectors to be installed as new technologies and techniques mature.  This report presents the main technological challenges, approaches, timelines and where possible required decision points for the realisation of a 3G network. While there exist different designs and implementations for the first 3G detectors (ET, CE for example), the enabling technologies (substrates, coatings, cryogenics, suspensions, Newtonian noise cancellation, lasers and quantum enhancement) are similar at the research level.  Timely progress in the development of these enabling technologies will require global collaboration and coordination. In order to accomplish the scientific program of the 3G network, a broad coherent detector R\&D program is needed now, addressing key technological challenges in the next 5-7\,years.

Four areas in particular are of such a scale that global coordination will need to be accompanied by global R\&D funding: 

\begin{enumerate}
\item facility and vacuum infrastructure; 
\item substrates; 
\item coatings; 
\item large scale prototyping for demonstration of technology readiness. 
\end{enumerate}

\noindent Progress in areas 1, 2 and 3 will require significant involvement with industry, with some areas, e.g. coatings,  potentially requiring the field to build its own plants. Area 4 has seen recent growth, with the establishment of the 3G Pathfinder in Maastricht, but more prototyping facilities are likely to be required. Depending on R\&D progress, the first 3G detectors may adopt technologies proven in 2G facilities, scaled up to much longer baselines. Conversely, many 3G techniques may be tested or employed to improve the sensitivity of 2G detectors. 

\section*{Recommendations}
\begin{itemize}
\item \textbf{Recommendation 1}:  An international 3G R\&D coordination committee should be formed, with broad and inclusive membership representing GW groups across the world.

 A series of workshops on enabling technologies shall be held in order to stimulate exchange of ideas and allowing (if deemed useful) for coordination of the person-power intensive experimental activities.  Each of the major R\&D tasks should generate a list of  goals with quantitative metrics,  timelines and required resources.   Activities requiring global collaboration and coordination should be laid down and pathways identified.

\item \textbf{Recommendation 2}:  International consortia should be formed to work on key issues with industry partners, establish a governance and organisational structure with teeth (i.e. controls purse strings) and seek funding through joint proposals submitted across funding agencies.

\item \textbf{Recommendation 3}: National funding agencies to increase funding (staff and instrumentation)   to enable the required long-term research programmes at relevant laboratories and prototype interferometers.

\item \textbf{Recommendation 4}: Existing GW collaborations take ownership of 3G R\&D tasks and integrate it into their programs and deliverables, in order to ensure sufficient support for the long-term future of the field (as it has been done so successfully for 2G R\&D during the operation of the initial detectors). Over the next 5 years, mature 2G enabling technologies (e.g. 1064\,nm laser, fused silica optics, coatings) should be demonstrated to be up-scaled and ready for application in 3G facilities.

\end{itemize}

\section*{}
\begin{figure}[ht]
\centering
\includegraphics*[width= \textwidth]{Figures/3G_Readiness_Levelsblue.pdf}
\caption{Approximate timelines for the required maturity levels for 3G instruments and resource levels needed from now to installation of the first phase. The estimates for the required R\&D resource level shown on the right hand side only include investment, not full costing and are roughly categorized into \textit{low, mid} and \textit{high}.\\
}
\label{fig:maturity}
\end{figure}

An overview of required timelines to reach installation maturity is depicted in figure \ref{fig:maturity}. The figure shows required maturity levels for the various subsystems, depending on the foreseen time of installation and anticipated lead times. Infrastructure and facilities naturally will have to reach maturity earliest, followed by the other subsystems in the sequence of installation. The timings for the lower maturity levels ML1 - ML3 (relative to the highest level ML4) depend on the duration required between the individual steps; e.g. once technical readiness is achieved for core optics it still takes a few years to demonstrate full scale prototypes and manufacture the final optics substrates. Despite considerable differences for the different observatories (like ET and CE), large variations within the subsystems and an inevitable uncertainty in timelines, we summarise the timelines  for each subsystem in a single bar in figure \ref{fig:maturity}.  

The resources required to reach operational readiness are roughly divided into \textit{low, mid} and \textit{high}, with indicative financial investments for R\&D for the various subsystems over the entire period from now to the start of installation. 

The highest costs for the detector elements (as distinct from the civil and vacuum construction) are expected for developing the capabilities to manufacture the main optics (presumably fused silica and silicon) and for the development of coatings and coating facilities. 
Producing ultra-pure optics substrates of approx. 200-300\,kg weight requires an international effort and tight collaboration with industry. Developing coatings of outstanding optical quality and uniformity over the whole mirror surface, combined with the required low mechanical losses at  room temperature and cryogenic temperatures is currently regarded as the biggest hurdle to overcome for building 3G gravitational wave observatories. International collaboration and building redundancy in coating capabilities is essential for success.

In particular for underground infrastructures and facilities, R\&D will incur significant costs for exploration and prototyping. Exploratory efforts have already been started at the ET candidate sites. In the construction phase, building the infrastructure and facilities will be the biggest cost items and consequently have the largest cost saving potential. R\&D efforts to minimise costs while satisfying the strict technical demands is mandatory.

% \magentacomment{Josh: Here I pasted some remarks from quantum noise that Dave suggested be moved to general considerations.} 

% \begin{itemize}
% \item National funding agencies to increase funding (staff and instrumentation) of the relevant lab activities and to enable the required long-term research programs at the relevant prototype interferometers.
% \item Existing GW collaborations to take ownership of 3G R\&D and integrate it into their programs and deliverables, in order to ensure sufficient support for the long-term future of the field (as it has been done so successfully for 2G R\&D during the operation of the initial detectors). 
% \item Establishment of worldwide discussion and coordination forum (meeting several times a year) across all relevant collaborations, in order to stipulate exchange of ideas and allowing (if deemed useful) for a coordination of the person-power intensive experimental activities.
% \end{itemize}

