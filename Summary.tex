\chapterimage{Figures/summary.jpg} % Chapter heading image
% Blackboard scribbles copyright Andreas Freise
\chapter{Summary}
\label{sec:Summary}

Prioritize R\&D items by timelines and complexity (Infra and Fac most urgent, solutions known, large cost saving potential), 
Coatings and 3G optics (no clear path forward but a little less time critical) 

\magentacomment{Josh: Here I pasted some remarks from quantum noise that Dave suggested be moved to general considerations.} 

\begin{itemize}
\item National funding agencies to increase funding (staff and instrumentation) of the relevant lab activities and to enable the required long-term research programmes at the relevant prototype interferometers.
\greencomment{Dave: Certainly.  But this is a generic recommendation in the sense that all R\&D 3G efforts need more R\&D funding.  I don't think this needs to be here (or in any other chapter), but somewhere in the main set of recommendations.}
\item Existing GW collaborations to take ownership of 3G R\&D and integrate it into their programs and deliverables, in order to ensure sufficient support for the long-term future of the field (as it has been done so successfully for 2G R\&D during the operation of the initial detectors). 
\item Establishment of worldwide discussion and coordination forum (meeting several times a year) across all relevant collaborations, in order to stipulate exchange of ideas and allowing (if deemed useful) for a coordination of the person-power intensive experimental activities.
\end{itemize}

\section{collection of recommendation section from the various chapters}
\subsection{Facilities + Infrastructure}
collaboration with high energy particle physics community (know-how in building large vacuum
systems from accelerator facilities)

3G Infrastructure and facilities construction will start roughly 5 years before the installation of detector hardware, hence the timescales for technical readiness are shorter and more critical here.
If site preparation for the Einstein telescope is to start as early as $\sim$\,2025 the respective R\&D and detailed technical studies have to start immediately. Respective international and collaboration overarching working groups should be initiated right now.

\subsection{Core Optics}
There is not a research laboratory that is dedicated to develop and optimize the large size crystal growth with the specifications imposed by the GW detectors. This could have a severe impact on the development of 3G detectors.
\magentacomment{hal: The lead times for this R\&D are huge. Steering the direction needs decision points well ahead of time. }
timelines missing
\subsection{Coatings}
By far the most pressing timeline is for the enhanced 2G detectors. In the case of a+LIGO, allowing for a one-year pathfinder after the coating material and process are identified, the required research to identify a suitable mirror coating should be completed by May 2020. This timeline implicitly assumes that the coating will be a sputter-deposited amorphous doped oxide, perhaps deposited at an elevated temperature, a slower than conventional rate and/or with a higher than conventional annealing temperature. It is unlikely that there is time in such a schedule to identify coatings and develop the required equipment for a process with more significant deviations from current deposition methods.

In recognition of the fact that meeting the enhanced 2G detector timeline requires doing basic research on a development schedule, the efforts of the LSC community are of necessity focused on these mirrors. That said, it is also recognized that the community shouldn't put itself again in such a situation, so a portion of the current research effort is devoted to establishing approaches to mirrors for 2.5G and 3G detectors. It is difficult to set research timelines for this effort, since the funding and construction schedules for these systems are not yet established. Another open question is the deposition process that will be required for the 2.5G and 3G mirrors; the further that process is from conventional IBS, the longer it is likely to take to develop suitable tooling. It seems that in any plausible scenario at least five years are available for research into the best approach to mirrors meeting 2.5G requirements. Results for these 2.5G mirrors will, in turn, inform choices with respect to 3G mirrors. It is therefore too soon to argue for a large investment in scaling deposition tools alternative to elevated-temperature IBS for 2.5G and 3G mirrors. That said, as the funding trajectories and interferometer architectures become better defined, it will be important to regularly re-evaluate the current understanding of potential mirror technologies, and make critical decisions, especially with respect to deposition tools with long development times and requiring major financial investments.

\subsubsection{Recommendations}

\noindent The additional funding provided by the CCR for U.S. efforts in coating research, combined with the previously existing NSF support, leave these efforts with reasonably adequate funding in the near future. It would be helpful to add another deposition group to the effort, as there is currently more capacity for characterization of properties (other than cryogenic mechanical losses) than for synthesis. It is also important that the LIGO Lab continue with at least its current effort level, as their contribution to high-throughput mechanical loss characterization, optical scatter and homogeneity measurements, and their overall coordination of sample fabrication, distribution, and characterization is important to the LSC efforts.

In GEO, there is growing capacity for depositing coatings at Strathclyde, UWS and Hamburg. These coatings can be produced at a rate faster than it is possible to characterize their properties at cryogenic temperatures, and thus more resources devoted to such characterization are desirable. Currently, a novel type of ion-beam sputtering is being developed and tested at Strathclyde. While this is a promising research route, there are also plans to set up a large chamber with an industry ion source which should be capable of producing large and uniform enough coatings. This is an important priority in gaining access to more coating facilities which are suitable for the deposition of large, high-quality coatings. Initiating studies of GaP/AlGaP crystalline coatings, using hardware now installed in an MBE chamber at Gas Sensing Solutions Ltd, is also planned. This is an important parallel research direction to the development of amorphous coatings.

In VIRGO, the activities are progressing at the pace compatible with the funding available from other projects. \magentacomment{Josh: I do not understand what the previous sentence intends to say.} LMA and University of Sannio are able to provide high quality coatings at a rate higher than the existing characterization capability in VCR\&D. The project ViSIONs takes care of only one specific aspect of the research, that is the impact and the understanding of deposition parameters on the coatings properties.

At present, there is only one institution, LMA, capable of depositing coatings of a size and quality suitable for gravitational wave interferometers. LMA will continue to be supported as a research and coating facility for future GW detectors by the French CNRS through EGO, while support for another institution, CSIRO in Australia, which was able to provide similar coating quality, was discontinued some time ago. It seems prudent to restart this program to create redundancy through a second source of high quality coatings and to have another significant contributor to ongoing research efforts. Within these activities, the products of other commercial vendors can also be characterized. 
Continued and deeper coordination among all GW collaborations will be critically important in developing coatings with the requisite 3G performance.

\subsubsection{Roadmap}
Since the development of low-thermal-noise coatings is in the stage of research rather than development, even for 2.5G detectors, a conventional roadmap would not seem the best model for describing the path towards identifying suitable mirror designs and fabricating corresponding full-scale mirrors for 3G detectors. Noting that the architectures and operating parameters for 3G detectors remain in flux, that the results for mirrors developed for 2.5G detectors will have a strong, perhaps determinative, influence on the designs for 3G mirrors, and that the funding and thus construction schedules for 3G detectors are not yet clear, it is best to estimate schedule implications in terms of times before installation. Here we discuss aspects that will drive the necessary decisions, and estimate the time prior to anticipated installation in 3G detectors that various steps must be completed. 
The currently plausible approaches to 3G mirrors fall into three broad categories which have different cost and schedule drivers: amorphous coatings deposited by methods similar to conventional ion-beam sputtering (IBS), amorphous coatings deposited by alternative means, e.g. chemical-vapor deposition (CVD), and crystalline coatings. 
Amorphous coatings deposited by IBS methods: IBS is currently the only mature technology for depositing mirrors suitable for GW interferometers (GWI); the challenge is identifying suitable combination of materials and deposition conditions (rate, ion energy, substrate temperature, …) to meet mechanical loss and optical specs. Time required for this step is open-ended; a solution might be found next month, or might not exist at all. Once materials and conditions are determined, the time required to reach the production stage depends on the deviation from current IBS practice: for room-temperature deposition at conventional rates, perhaps a 1-2 year pathfinder would be adequate. For more extreme conditions of elevated substrate temperature, low rate, high ion energy, ion-beam assist, nanolayers, microwave annealing, … perhaps 3-5 years and USD 3-5M for developing suitable equipment and the pathfinder process might be required. Multi-material coatings would lie between these extremes of time and equipment cost. 
Amorphous coatings deposited by methods other than IBS: If research shows the optimum  deposition method is other than IBS, for example chemical vapor deposition (CVD) of a-Si/SiNx mirrors, in addition to open-ended research time similar to the IBS case, adequate time would be required to develop deposition tools and a scaled up process suited to GWI requirements. While CVD tools are widely used in the semiconductor industry, adaptation to GWI mirror requirements would be a significant effort. Instrumentation development and the more complex pathfinder process might require  USD 25-30M and perhaps 10 years. These figures are order of magnitude estimates that likely could be tightened up through discussions with equipment vendors. 
Crystalline coatings: For AlGaAs crystalline mirrors, the materials research and modeling phase for small scale (ca. 6 inch) coatings could be completed in perhaps three years. If these results showed GWI level performance, the time and financial costs for substrate and tool development and the more complex pathfinder process to scale to GWI size optics might be USD 25M and 10 years. 


\subsection{Cryogenic}
\subsubsection{Pathways, required facilities and collaborations}
%
For each operating temperature (below 20\,K, 123\,K)


\begin{enumerate}
\item demonstrate reliable and timely  heat extraction and required final temperature stability (varies depending on goal material parameters).
\item demonstrate integration with SIS   (Nb: requirements differ for CE and ET).
\item demonstrate integration with other subsystems.  eg auxiliary optics  for  wave front control .
\item system integration and performance  testing on large scale prototypes/detectors.
\end{enumerate}

A key element of any cryogenic implementation is a refrigerator that is as quiet and vibration free as possible. R\&D devoted to design and construct silent refrigerator machine is required.  Several ideas have been proposed, for example the use of PT cryocoolers with symmetric cold heads.  Collaboration with cryogenics industry and the accelerator community is strongly recommended. 

For sapphire core optics at 20\,K:   Items 1 and 3 have been/will be demonstrated using KAGRA.  As the current sensitivity goal of KAGRA is modest at low frequencies, it is not clear how well KAGRA performance will inform item 2 and 4.  May occur with KAGRA upgrade.

Silicon core optics at 18\,K, 123\,K : Items 1-3:  a number of facilities are planned or under construction.  Existing facilities  at Stanford, Caltech and Gingin are currently being reconfigured. A  new facility at Maastricht in Europe has been designed to investigate  both 123K and  below 20\,K. Much of the required technology for cryogenic operation is also needed in the near term for  exploration of  the material properties of mirrors, coatings, etc.  \textit{Tight coordination is needed between these groups to  ensure optimum focus and timely progress.}

Ultimately full system integration tests (Item 4) need to be done on large, sensitive, prototypes, demonstrating sustained high performance.  The prototypes need to be large enough to allow reliable scaling to full size 3G systems.  \textit{\textbf{This is a major undertaking and global collaboration is needed to resource, develop and manage these prototypes}}.  These could be new facilities, or,  potentially, an existing 4\,km facility(ies) could be re-purposed.  This would not only demonstrate technical readiness but would produce detector(s) with a significantly extended range.  This is effectively the Voyager concept.

Cooperation with high-energy particle physics, where comparable requirements are addressed, could create valuable synergies. Furthermore, the use of cryogenics in the electronics and the magnetic actuators\cite{cryo:OSEM} has the potential to be of great benefit. These techniques should be explored in parallel with the more fundamental cryogenic interferometry.

\subsubsection{Roadmap}

\textit{Sapphire at 20\,K}


\begin{itemize}
\item  2019-2026  	Use KAGRA to diagnose  feasibility and performance.
\item  2027  		First Cryogenic downselect.  
\item  2028-2032  		If GO:  final design and fabrication.
\end{itemize}



\textit{Silicon at 	123\,K}
•  \begin{itemize}
\item 2019-2025  	R\&D in university labs on 1. and 3.
\item 2022-2028	Integration testing with SIS and other subsystems  (item 2).
\item 2028……  	Cryogenic downselect
\item 2029-2033	Large  Phase noise prototype
\item 2034-2039		If GO:  final design and fabrication.
\end{itemize}



\textit{Silicon at 	18\,K}
\begin{itemize}
\item SAME CYCLE AS SILICON AT 123\,K.
\end{itemize}

Note that it is feasible that both Silicon 123\,K and Silicon 18\,K cryogenic subsystems will be deployed at different 3G facilities.  Depending on facility finding and  technology readiness, it is  feasible, perhaps even likely,  that 3G facilities will initially operate room temperature detectors whilst cryogenic development continues.

\subsection{Newtonian Noise}

\subsection{Light Sources}
\subsection{Quantum enhancements}
\subsection{Suspensions and Seismic Isolation Systems}
\subsection{Auxiliary Optics}
\subsection{Simulation and Controls}
\subsection{Calibration}


\input{ExecSummary.tex}