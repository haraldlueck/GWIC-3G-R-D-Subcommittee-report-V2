\chapterimage{Figures/HeaderNNsmall.jpg} % Chapter heading image
% tiltmeter at aLIGO detector copyright Krishna Venkateswara

\chapter{Newtonian Noise}
\label{sec:Newtonian_Noise}

Newtonian noise (NN) is predicted to be one of the limiting noise sources in third-generation detectors at frequencies below 30\,Hz~\cite{Saulson:NN,Har2015}. Sources of NN include seismic fields, atmospheric sound and temperature fields, and vibrating infrastructure~\cite{HuTh1998,BeEA1998,Cre2008,FiEA2018,Har2015}. Mitigation of NN can be achieved by suppressing density perturbations in the environment near the test masses~\cite{HaHi2014}, and by cancellation of NN in gravitational-wave data using environmental sensors~\cite{Cel2000,CoEA2016a}.

\begin{samepage} % this vodoo is needed to stop Latex from skipping half a page

In second-generation detectors, the dominant contribution to seismic and acoustic fields, and therefore to the associated NN, is mostly produced by detector infrastructure such as pumps and ventilation fans. The natural environment will become more important in third-generation detectors, since, (a), we have learned and will continue to learn how to build detector infrastructure that does not significantly disturb the environment in the NN frequency band, and (b), our targets for NN cancellation will be so ambitious that we will care both about the dominant anthropogenic perturbations and the weaker disturbances cased by nature. Therefore, NN research has a potentially big impact on site selection~\cite{BeEA2010}. 


\section{State of the art}
% \subsection{Modeling of NN} 
Much effort has gone into modeling NN. Analytic models of seismic NN were calculated for homogeneous half spaces with spherical cavities for arbitrary wave polarizations including the scattered waves~\cite{Har2015} and for specific seismic sources such as point forces and point moments in homogeneous half spaces~\cite{HaEA2015,Har2016}. Numerical simulations were performed to simulate seismic fields from point forces in laterally homogeneous half spaces~\cite{BeEA2010c}. Analytic models of atmospheric NN were calculated for temperature fields in laminar flows~\cite{Cre2008}, turbulence induced pressure fluctuations~\cite{Har2015}, and homogeneous sound fields~\cite{FiEA2018}. 

\end{samepage} % this vodoo is needed to stop Latex from skipping half a page

% \subsection{Mitigation of NN}
Surface seismometer arrays optimized for NN cancellation were calculated using analytic and numerical methods~\cite{Har2015,CoEA2016a}. Analytic expressions were derived for Rayleigh waves to express NN and correlations between NN and seismometers in terms of observed seismic correlations~\cite{Har2015,CoEA2016a}. The potential impact of surface topography on NN cancellation was investigated~\cite{CoHa2012}. A tiltmeter signal was suppressed by more than an order of magnitude using Wiener filtering with a seismometer array, which serves as proxy of NN cancellation~\cite{HaVe2016,CoEA2018}. A factor 1000 suppression of seismic signals in seismometers using Wiener-filtered data was achieved with underground arrays at Homestake ~\cite{CoEA2014} and 100x at LIGO Hanford with surface arrays ~\cite{CoEA2018}. Extensive seismic array measurements were performed at Sanford Underground Research Facility, LIGO Hanford, and Virgo. Practical work on the mitigation of atmospheric NN has only started. Extensive sound correlation measurements were done at the Virgo site to characterize the sound field.

% \subsection{Impact of site selection and detector infrastructure on NN}
% We have a good understanding by now how certain choices concerning site and infrastructure can impact NN and its mitigation. 
Some aspects of how the site and infrastructure can impact NN are now well known. 
Ambient seismic noise is mostly understood from world-wide long-term observations and studies of how it depends on, for example, geology, and distance to major cities and coast~\cite{CoHa2012b}. Detailed studies of the connection between geology and ambient seismic fields were made at detector sites~\cite{HaOR2011} and at the Sanford Underground Research Facility. Seismic scattering from topography was studied in linear order to estimate the effect of scattering on NN~\cite{CoHa2012}. Sound and seismic noise between about 5\,Hz and 50\,Hz are dominated by sources that are part of LIGO/Virgo infrastructure (e.g., ventilation). Concrete plans are being developed for infrastructure changes at Virgo to mitigate NN from sound fields.

\section{Requirements}
% Understanding of how site selection and infrastructure design influences NN must be improved.
There remain, however, aspects of site and infrastructure influence on NN that are not well understood. 
While noise-cancellation systems can possibly enhance sensitivities of 3G detectors, methods to cancel NN from underground seismic fields and the atmosphere have not been developed yet even in theory. Consequently, atmospheric and underground seismic NN should currently be considered a fundamental noise limitation of 3G detectors. Generally, the aim of any new infrastructure and site selection should be to have natural sound and seismic noise levels as close as possible to the global low-noise models~\cite{Pet1993}, especially with the goal to extend the observation band to frequencies below 10\,Hz, and to perturb the natural fields as little as possible with the infrastructure. 

% \subsection{Infrastructure R\&D and site selection}
We have seen that the infrastructure of current GW detectors is the main source of seismic and sound disturbances in the frequency band between 10\,Hz and 30\,Hz. It is therefore important to (a) learn from this experience, and develop low-noise infrastructure designs for 3G detectors % Assuming that this problem is solved, 
and (b) 
% the natural environment of the detectors becomes important, and so the goal is to 
develop a set of tools to characterize the environment, and to understand how it affects NN. This part needs to take into account what information about a site can realistically be obtained in a 1 to 2 year site-characterization study. It is recommended to establish globally accepted guidelines of how site-characterization measurements are to be carried out to be complete and of sufficient quality. We need to establish how much NN is reduced as a function of detector depth. Finally, (c) improved models especially of atmospheric NN are required based either on analytical calculations or numerical simulations.

% \subsection{R\&D for the cancellation of NN}
Noise-cancellation systems need to be developed 
% if the goal is 
to go beyond the infrastructural noise limitations. (a) Concerning surface detectors, new technologies are required to monitor fluctuations of the atmospheric mass-density field, which is connected to sound and temperature fields. Coherent LIDAR was proposed as a possible sensor, but it is unclear whether the required sensitivity can be achieved. This problem plays a minor role in underground detectors, where atmospheric NN is strongly suppressed. However, (b) a cancellation system of NN from sound fields inside buildings and underground caverns might be required also for underground detectors. In this case, a simple microphone array can in principle be used, but how to design it based on sound-correlation studies is still an unsolved problem. These studies rely on (c) new numerical simulations and advanced analytic models of atmospheric fields including effects such as turbulence and sound scattering, especially for underground detectors where the goal is to observe GWs down to a few Hertz. (d) cancellation of NN from underground seismic fields needs to be developed. Here, the main questions are where seismic sensors should be ideally placed, and what type of sensors are to be used (single-axis or three-axis seismometers, seismic tiltmeters and strainmeters). Special attention should be given to (e) how local geology and above all surface topography increase the complexity of the seismic field through scattering of seismic waves. This will have an important impact on 
% how easy it is to 
the design of seismometer arrays for seismic NN cancellation.

\section{Impact/relation to 2G and upgrades}
Newtonian-noise cancellation techniques will evolve continuously from 2G to 3G surface detectors, and insight gained from 2G R\&D will be applicable to 3G detectors. This is true for seismic as well as atmospheric NN in surface detectors. This development extends from the current 2G detectors, to their minor and major upgrades, into the 3G era, whenever the goal is to also achieve low-frequency sensitivity improvements. 

There are aspects of NN modeling and cancellation unique to 3G underground detectors, as for example the question how quickly surface disturbances are suppressed with increasing detector depth, and how to cancel seismic NN in underground detectors. The KAGRA underground detector in Japan might 
% in principle 
provide some continuity of underground NN R\&D towards the 3G era, but NN is currently not expected to be a limiting noise source for KAGRA.

\section{Pathways and required facilities}
The main facilities of NN R\&D are the natural environments of 3G detector candidate sites. In addition, since we have very little experience in observing and modelling underground seismic and sound fields, underground studies done anywhere can provide important information. There are significant differences between sites in terms of sources of environmental disturbance, local geology and topography, so transferring results from one site to another should be done with caution.

Finite-element simulations of environmental fields and algorithms for the optimization of array configurations for noise cancellation are computationally expensive. Significant computing time on clusters will be required. 

Collaboration between groups is strongly recommended with respect to site characterization. It requires substantial expertise to set up robust, high-quality environmental measurements, to understand, which properties of environmental fields are important, and how to analyze environmental data. Collaboration is also encouraged between groups working on numerical simulations to share generic knowledge, such as how to assess the accuracy of numerical simulations, e.g., by comparing simulations using different software, or by running simple simulations that allow comparison with analytic models.

\section{Timeline}

As the level and character of NN and the choice of the 3G sites are strongly interdependent, a solid understanding of the NN-relevant characteristics of candidate sites is required imminently. After this, significant R\&D into the above open questions should continue through the 3G era such that the NN of the chosen site and infrastructure can be reduced as much as possible.  

% \magentacomment{hal: timelines?}
