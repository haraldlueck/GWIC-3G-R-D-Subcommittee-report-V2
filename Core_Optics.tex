\chapterimage{Figures1_3/Core_Optics_1_3.png} % Chapter heading image
% aLIGO optics copyright LSC/LIGO  
%need better resolution picture  Photo Mike Fyffe Caltech
%\section{Introduction}
\chapter{Core Optics}
%\section{Core Optics}
\label{sec:Core_optics}
%\section{Background}

% For 3G detectors there is a strong interdependence among the core optic substrate material, operating temperature, and laser wavelength.

The substrate materials that will be used for core optics in the 3G detectors are interdependent on the operating temperatures of those detectors (Section~\ref{sec:Cryogenics}) and the laser wavelengths to be used (Section~\ref{sec:Light_sources}). 
Thermal noise plays a strongly limiting role in all current gravitational-wave detectors. At room temperature, fused silica, which is used in LIGO and Virgo, is an excellent substrate material due to its very low thermoelastic effect and ultra low optical~\cite{GEO_Absorption} and mechanical losses~\cite{Ageev_2004}. Many planned detectors
% Thermal noise will limit the sensitivity of all current gravitational wave detectors. Many planned detectors and upgrades will use cryogenic cooling of the mirrors to reduce thermal noise -- 
including KAGRA~\cite{KAGRA2013}, Voyager~\cite{VoyagerDCC2018}, the Einstein Telescope low-frequency detector (ET-LF)~\cite{ET2011} and the upgrade to Cosmic Explorer (CE2)~\cite{CosmicExplorer2017} will operate at cryogenic temperatures to reduce thermal noise. 
% At cryogenic temperatures, fused silica suffers from a broad and high peak in mechanical loss~\cite{Travasso_2007}, and cannot be used. 
Fused silica is not a suitable mirror substrate material at cryogenic temperatures due to strongly increased low-temperature mechanical loss~\cite{Travasso_2007}.
In general, crystalline materials have much less Brownian noise at cryogenic temperatures than at room temperature due to their ordered lattice structure.
% a very large peak in mechanical loss centred on $\sim$40\,K, which would result in a significant increase in thermal noise on cooling a silica mirror to temperatures below 100 K.
Sapphire, which is used in KAGRA at 20\,K~\cite{Hirose_2014a}, and silicon, which is planned for use in Voyager, ET, and CE2, have especially promising performance including low mechanical loss at cryogenic temperatures. 
% for the key considerations of cryogenic gravitational-wave detectors.
% The two materials that are most commonly considered for use at cryogenic temperatures are sapphire, which is used in Kagra at 20\,K~\cite{Hirose_2014a}, and silicon, which is planned for use in Voyager, ET, and CE2. 
% Both materials can have low mechanical loss at cryogenic temperatures, essential for keeping the substrate contribution to thermal noise low. 

%In addition to low thermal noise, 
3G detector core optics must also meet a number of other challenging requirements. 
%\magentacomment{Beverly: Has any thought been given to all reflective IFOs or is that known to not work?} 
Uniform and pure masses of several hundred kg (see Table~\ref{Tab:FutIfos}), significantly more than currently in use, are required to reduce radiation pressure noise and accommodate larger diameter laser beams in order to better reduce (through averaging, see figure~\ref{fig:Thermal_Noise}) thermal noise. Optical absorption and scatter must both be low. The ET Design Study~\cite{ET2011} specifies a scatter loss of 37.5\,ppm per mirror surface. We note that there is excess scatter loss in the arm cavities of the advanced detectors, and further understanding of this will be important to reach the 3G scatter requirements. Finally, requirements for test mass cooling for Voyager and ET-LF lead to requirements for the optical absorption of the input test mass substrates to be less than $\sim$10\,ppm/cm.

The major open questions for core optics materials that must be addressed by 3G R\&D are described below with a focus on fused silica, silicon and sapphire. More details are given in Appendix~\ref{sec:Appendix_Core_optics}.


%\magentacomment{hal: Is this paragraph required or does a link to suspensions and cryogenics suffice?} Extraction of the laser power absorbed in the mirrors (and their coatings) will be an important consideration, impacting on the operating temperature of the mirror. For mirrors at 123 K, as proposed for LIGO Voyager, significant cooling power is provided by thermal radiation. Mirrors operating at lower temperatures will need to rely largely on conduction through the suspension fibres for heat extraction. It should be noted that heat extraction at 10 K has significant implications for the design of suspension fibres, and thus for suspension thermal noise, with one study suggesting silicon fibre diameters approaching 4mm may be required.

%The optical absorption requirement is set by a combination of the desired operating temperature, the potential for heat extraction from the cold mirror and the desired laser power in the interferometer. For LIGO Voyager, with  3\,kW power on the central interferometer and assuming a maximum radiative cooling power of 10\,W, the allowable absorption in the ITM is less than 15\,ppm/cm.  For ET-LF, with $\sim$850\,W %18\,W 
%of central interferometer power and assuming a cooling power of 100\,mW via conduction through the suspension fibres, a similar absorption requirement of less than 11\,ppm/cm is found. \magentacomment{hal: pretty sure the numbers were wrong. check powers stated here. It seems that the PRC power and the power absorbed in the ITMs has been disregarded (forgotten???) in the ET DS. 11ppm/cm in the ET-LF ITMS leads to 17mW absorbed power, similar to what was assumed for intracavity surface absorption.}


%\section{Candidate materials for low noise substrates}
% \magentacomment{This paragraph and the first one above are so similar...I'll merge them later}
 
% The main challenge is the production of homogeneous large volume crystalline substrates with low enough defects and optical absorption.

\section{Fused Silica} 
For fused silica, homogeneity of the refractive index is an important consideration. For the substrates of the cavity mirrors, excellent homogeneity is only important in the two dimensions perpendicular to the beam axis. This can be achieved even for large volumes (corresponding to a total mass of several hundred kilograms). However, the beam splitter requires a very high homogeneity in all three dimensions and this can currently only be guaranteed by the manufacturers for masses up to 40 kg (diameter 55 cm, thickness 7 cm). The company Heraeus has planned some tests to push this limit to about 100 kg. Whether such large beam splitters are required depends on the optical layout of the detector and is under investigation. Other issues with fused silica include charging and relatively low Young's modulus.

\section{Silicon}
Silicon has a low mechanical loss at cryogenic temperatures (similar below 10 K to fused silica's at room temperature).
% , resulting in low substrate thermal noise. 
In addition, the thermal expansion coefficient of silicon is zero at $\sim$123\,K and 18\,K which allows, with temperature control, the suppression of substrate thermoelastic noise and thermal expansion effects due to absorbed laser power.

Silicon is opaque at the currently used wavelength of 1064\,nm. Initially, the telecommunications wavelength of 1550\,nm was proposed for use with silicon mirrors, due to wide availability of high-powered lasers and optical components. More recently, there has been growing interest in using a wavelength close to 2000\,nm. A major driver towards 2000\,nm is the development of amorphous silicon as a possible low thermal noise cryogenic coating material. Amorphous silicon exhibits significantly lower absorption (a factor of $\sim$7) at 2000\,nm than at 1550\,nm. It seems likely, therefore, that the choice of mirror coatings will be a major factor in the choice of wavelength for future detectors. 

Sufficiently low optical absorption can be obtained from silicon refined with the Float Zone technique, however, the maximum diameter for this method is $\sim$200\,mm. This is too small for the requirements of future detectors (e.g. ET-LF requires 450\,mm diameter, 550\,mm thick optics and CE2 up to 700\,mm diameter optics). While larger diameter silicon pieces can be produced using the Czochralski method, the optical absorption of this type of silicon is too high, due to impurities related to the production method. A magnetic Czochralski process (MCz) exists, in which a magnetic field is used to reduce the impurity concentration in the centre of the ingot. This process can produce diameters of up to 450 mm, and a production line for manufacturing silicon of this diameter does exist at the company Shin Etsu, but is currently not operational. Initial studies of the optical absorption have shown low values at room temperature of $\sim$3\,ppm/cm at 1550\,nm and $\sim$5\,ppm/cm at 2000\,nm. The measurements showed an increase towards lower temperatures, reaching approximately 10\,ppm/cm at 50\,K, meeting the requirement for cryogenic silicon mirrors of the ET design study. However, initial studies indicate that the absorption of this material can vary significantly, both along the radius and along the length of an ingot, and more studies of the homogeneity of the absorption and its dependence on the thermal history of the sample are required. 

There is evidence that polishing silicon surfaces can increase their optical absorption. 
The presence of surface absorption has been confirmed~\cite{SiliconSurfaceAbsorpBell2017}, and it was shown that a proprietary polishing process can be used which does not produce this effect. 
%The presence of surface absorption was confirmed in a study in the IGR in Glasgow \cite{SiliconSurfaceAbsorpBell2017} and it was shown that a proprietary polishing process can be used which does not produce this effect. 
While this has been consistently demonstrated, further work is required to test whether a silicon surface can be polished to the specifications required for a GW detector without resulting in surface absorption.

Non-linear absorption in silicon is not expected to set a major limit to performance, contributing <\,0.5\,ppm/cm absorption for a wavelength of 2\,$\mu$m at the light intensity assumed for inside the Voyager ITM. At the significantly lower intensity within an ET-LF ITM, these effects are even less significant. 
Two-photon absorption generates free carriers in silicon: the absorption due to these free carriers depends crucially on the carrier life time. 
%Experiments to measure this for magnetic Czochralski silicon are underway in a collaboration between Stanford and Glasgow.
% \greencomment{Dave: I like including absorption mechanisms in this section.  But you don't talk about free carrier absorption, impurity absorption, etc... Presumably there are the main sources of absorption?  }
It will be important to test the optical scattering from MCz silicon, particularly as the MCz growth process is known to produce a high void content in the material. 
%Work on this is underway at Glasgow and at Caltech. 
Initial scattering estimates
%at Glasgow~\cite{SiliconScatter2017} 
suggest that the scattering is higher than in fused silica, but is likely to be within the required limits~\cite{SiliconScatter2017}.

\section{Sapphire}
Sapphire is transparent at 1064\,nm and hence does not require changing the currently used laser wavelength. It has low mechanical loss at room temperature~\cite{Rowan_2000a} and even lower loss at cryogenic temperatures~\cite{uchiyama1999mechanical}. 
% On mirror-size samples measurements are affected by the suspension systems used. 
Sapphire's elastic constants are about 3 times higher than silicon's, helping to reduce thermal noise (\ref{fig:Thermal_Noise}). The high Young's modulus of elasticity has two additional advantages: fewer parametric instabilities and a higher internal resonance frequencies. The thermal conductivity of sapphire increases with decreasing temperature and reaches a peak of several $10^3$\,W/(m \,K) around 20-40K. Thermoelastic noise is high at room temperature, but quite low due to the high thermal conductivity at low temperatures. This high conductivity (and the low temperature coefficient) make the thermal lens effect negligible. 
The optical absorption of sapphire has been found to vary strongly from crystal to crystal and for crystals from different suppliers. % There is only a very small chance of finding sapphire crystals with low absorption by cherry-picking among a large number of products on the market. 
In KAGRA, it turned out that it was necessary to develop sapphire crystals with low absorption ($<$50ppm/cm) by working closely with the crystal manufacturers. 
% Although it took some time, sapphire crystals were developed to meet 50ppm/cm with sufficient margin \cite{Hirose_2014a}. The production success rate due to the inclusion of bubbles is 1/8.
Theoretical work on scattering in sapphire sets a lower limit of 0.21\,ppm/cm, with higher measured values of around 13 ppm/cm being attributed to impurities and vacancies.
Sapphire's high Mohs hardness of 9 and its crystalline structure, resulting in orientation dependent machinability, makes it harder to process sapphire substrates. 
% Polishing the KAGRA test masses took much longer than polishing the aLIGO test masses, although the size is smaller. Despite this hardness, no degradation in micro roughness or surface figure of the polished surfaces has been observed in KAGRA sapphire test masses\cite{Hirose_2014a}, compared to aLIGO or advanced VIRGO test masses made of fused silica. 
%More specifically, looking into Power Spectral Density (PSD) of uncoated surfaces at different spatial frequencies from $10^{-2}$ to $10^3$\,1/mm of both sapphire and fused silica,  no apparent degradation in sapphire has ever been observed.
Finally, Sapphire is birefringent, though KAGRA has taken steps to address this including alignment of the c-axis with the beam axis and locally adjusting the substrate thickness with ion beam figuring.  
% Since sapphire is a uniaxial crystal, the alignment of the optical axis, called the c-axis, to the beam axis should theoretically make the refractive index uniform in the plane perpendicular to the beam axis. In reality, however, even with optimized alignment, it is common that the inhomogeneity of the refractive index of sapphire is one order of magnitude worse than that of fused silica. It is therefore essential to compensate for this by locally adjusting the thickness of the substrates. 
% This can be addressed by aspherically polishing the back of the mirror, and the ion beam figuring technique (IBF) has been applied to the KAGRA ITMs.

%Dielectric multilayer coating of SiO2 and Ta2O5 on sapphire substrates appear to be as good as such coatings deposited on silica substrates\cite{Hirose_2014b}. The biggest c-axis sapphire window currently available in the market for gravitational wave detectors is about 220mm in diameter and 150mm in thickness (mass of 23 kg) although the ingot size is actually much larger. These are grown by either HEM or TSMG methods which assure the lowest dislocation density among all the production methods. If we allow crystals to have bubbles or apparent defects inside, the size could be much larger. The ingot size has been simply limited by size of furnace where sapphire crystal is grown.

%The KAGRA monolithic suspensions (will) feature sapphire components for the core optic, the ears, fibers, and blade springs, attached using hydroxide catalysis and indium bonding~\cite{Kumar:2016_KAGRA}. KAGRA has demonstrated that sapphire can be successfully bonded through hydroxide catalysis bonding. Sapphire technology has also been extensively developed in order to produce mono--crystalline fibres of length greater than 1\,m and large plates of a large variety of thickness. \magentacomment{hal:what plates are those? for cutting fibres from them?}

%A collaborative research project with a Japanese company got started targeting 100kg sapphire windows whose diameter is 400mm and thickness is 200mm. Developing that large size windows compatible with lower absorption and higher homogeneity will be a key to success toward 3G detectors with sapphire test mass mirrors


\section{Outlook and Recommendations} 
The choice of core optic materials is strongly dependent on wavelength, temperature, and coatings for 3G detectors and has a strong impact on most other subsystems. In addition, much of the R\&D described here has long lead times and high cost. Steering the direction needs decision points well ahead of time. Decisions on core optics, based on the community's ongoing R\&D, must be made a decade before 3G construction, i.e., in the coming six years. 

We recommend that
\begin{itemize}
\item a laboratory dedicated to the development and optimization of large size crystal growth be identified or established.
\item International consortia should be formed to work on core optics challenges together with industry
partners, in a globally coordinated way.  This is an area where there could be large investments needed with industry, and strong international collaboration could ease the burden and be much more cost effective.
%There is not a research laboratory that is dedicated to the development and optimization of the large size crystal growth required by 3G gravitational-wave detectors; We recommend such a laboratory be identified. 
% This could have a severe impact on the development of 3G detectors.
\item For fused silica, continued work with Heraeus to ensure that the homogeneity of the refractive index will be sufficient for future detectors, and in parallel to explore the use of telescopes to use small beam-splitters but still have large beams in the arms. 
\item For silicon, the community should continue to study scatter, absorption, and absorption uniformity, at the temperatures and wavelengths of interest for 3G. 
\item sources of phase noise in silicon optics, thermo-refractive noise and carrier density noise, be studied experimentally  to ensure that they do not set unexpected limits of silicon mirror performance. 
%An experiment targeted at measuring thermo-refractive noise in silicon is currently being constructed in Glasgow. 
\item For Sapphire, learning as much as possible from the KAGRA experience and continuing to reduce absorption in large sapphire crystals are the main recommendations. 
\end{itemize}

