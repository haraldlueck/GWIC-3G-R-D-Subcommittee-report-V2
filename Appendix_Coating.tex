
\chapter{Coatings}
%\section{Coatings - roadmap to readiness --- Martin Fejer}
\label{sec:Appendix_Coatings}

Different techniques for minimizing thermal coating noise for the candidate wavelengths (1064\,nm, 1550\,nm, 2+\textmu m) and envisaged temperatures (293\,K, 124\,K, 18\,K) are listed in the following section. \\

\begin{figure}[ht]
\centering
\includegraphics*[width=\textwidth]{Figures/Thermal_noise.jpg}
\caption{Parameters contributing to Thermal noise in GWDs. Equation from \cite{Harry:CQG2002}}
\label{fig:Thermal_Noise}
\end{figure}

\section{Amorphous Oxides}
Empirical results for amorphous oxide mirrors deposited by IBS show that the mechanical loss at room temperature is reduced by post-deposition annealing, and that the reduction is greater at higher annealing temperatures \cite{vajente2018effect}. The maximum annealing temperature is limited by crystallization of the film. One approach under investigation is to chemically frustrate crystallization through introduction of an appropriate dopant into the coating to increase the temperature at which crystallization occurs. There is also some evidence that low deposition rates correlate with reduced losses. A promising result combining these two approaches, the low rate deposition of Zr:tantala and subsequent high temperature annealing, yielded a film with the 4-fold reduction in mechanical loss at room temperature required for aLIGO+ and AdVirgo+ mirrors, though this result is not fully reproducible. It should also be noted that no measurements of cryogenic losses are available and changes in room temperature losses and cryogenic losses are often anticorrelated, so the effects of this material on cryogenic 2.5G and 3G detectors are not yet clear \cite{martin2010effect}. Some measurements have begun to support a different approach to suppress crystallization and thus allow higher annealing temperatures, geometrically frustrating crystallization using nano-layers, where each quarter-wave layer consists of alternating layers of two materials with a thickness of several nanometers. Suppression of crystallization has been observed in such titania/silica nanolayers \cite{pan2014thickness}. Concerns such as homogeneity, scattering and interface effects in the nanolayers have not yet been characterized. 

Until recently, progress in reducing mechanical losses has been based on empirical studies alone. Recently, there has been progress in theoretical tools to guide the empirical work. Molecular dynamics methods enable calculation of the atomic structures that correspond to the energy spectrum of TLS in amorphous materials. Theoretical results correctly predict empirical data for the temperature dependence (at cryogenic temperatures) of losses in model systems like silica, and the trend in loss vs Ti-doping in tantala \cite{trinastic2016molecular}. Extension to room-temperature regimes is currently under development. These methods also have suggested dopants whose promise has been borne out experimentally, e.g. Zr in tantala. Closely related are experimental methods to determine atomic structure in amorphous films via electron or X-ray diffraction techniques \cite{bassiri2013correlations,hart2016medium,shyam2016measurement}. A “virtuous circle” of theoretical modeling, atomic structure characterization, and macroscopic property measurements (in particular mechanical loss) has reached the stage where the methods mutually reinforce and speed developments in the respective methods.

A theoretical concept that has emerged is that of an ultra-stable glass, i.e. one that has an atomic structure of low internal energy with greatly reduced density of TLS, which is inaccessible to conventional annealing processes on realistic time scales, but can be reached through vapor deposition \cite{singh2013ultrastable}. According to this picture, deposition at elevated temperatures (to increase surface mobility) and at low rates (to give longer times for surface atoms to explore the energy landscape) should lower the density of TLS. The first empirical example of such an ultra-stable inorganic material, elevated temperature deposition of amorphous silicon with two-order-of-magnitude reduction in cryogenic mechanical losses compared to room-temperature deposition, indicates the physical reality of the concept \cite{liu2014hydrogen}. This result both shows that a-Si is a promising material for long-wavelength mirrors (though unacceptable levels of optical absorption, discussed below, remain an issue), and motivates the search for ultra-stable amorphous oxides suitable for 1\,$\mu$m wavelength operation. Recent results for amorphous alumina show lower cryogenic losses even after annealing than in room-temperature deposited films, which appears, subject to further verification, the first example of an ultra-stable amorphous oxide. It is also a promising material for a low-index layer in a cryogenic mirror, since its mechanical loss is almost an order of magnitude lower than silica at cryogenic temperatures. These studies are in an early stage; further work (theoretical and empirical) is necessary to establish whether the ultra-stable glass concept is a route to a low-noise cryogenic mirror.  

\section{Semiconductor Materials}

\noindent The results for the mechanical loss of amorphous silicon deposited at elevated temperatures show that a mirror consisting of a-Si as a high-index layer together with silica as a low index layer would meet the thermal noise requirements for cryogenic detectors \cite{steinlechner2018silicon}. The issue that remains in this case is the optical loss, which is more than an order of magnitude higher than typical requirements. This absorption is associated with ``dangling bonds'' in silicon atoms that are three-fold rather than four-fold coordinated. As such, this mechanism is not intrinsic, and depends strongly on deposition conditions and post-deposition processing. For example, low deposition rates, post-deposition annealing and annealing in hydrogen can all reduce the absorption significantly, and are topics of active current research \cite{birney2018amorphous}. The absorption cross-section decreases with increasing wavelength, resulting in a strong preference for 2\,$\mu$m vs 1.5\,$\mu$m wavelength operation, and probably eliminating this material for use at 1\,$\mu$m. It is also worth noting that the best results for cryogenic losses in a-Si have been obtained with films fabricated by evaporation rather than sputtering \cite{liu2014hydrogen}. Verification of these results in sputtered films is important, since meeting optical quality requirements is not practical via evaporative deposition methods.

Silicon nitride is another material widely used in the semiconductor industry with potential for low-noise mirrors. Low-pressure chemical vapor deposition (LPCVD) produces films with adequate mechanical loss that, together with silica low-index layers, could form a mirror that meets thermal noise specifications for ET-LF or Voyager \cite{pan2018silicon}. IBS deposited silicon nitride has been also produced with a 1.8-fold reduction of mechanical losses at room temperature with respect to titania doped tantala \cite{pan2018silicon}. Both processes result in excess optical absorption, which is currently more than order of magnitude too large to meet specifications, but, with further development, silicon nitride, unlike amorphous silicon, has the potential to be used with 1\,$\mu$m wavelength operation. Another potential problem is the high Young's modulus, which makes the material more suitable for silicon or sapphire than for silica substrates.

\section{Multi-Material Coatings}

In this sense, the problem for semiconductor materials is the converse of that for oxide materials: the mechanical properties are adequate, but the optical absorption is too high. A general approach that can address this problem is the use of multi-material coatings, in which the top layers, where the optical intensity is highest, consist of materials with low optical absorption but too large mechanical loss, while the lower layers consist of materials with low mechanical loss but too large optical absorption \cite{yam2015multimaterial,steinlechner2015thermal}. In this way, the limitations of the types of material can be traded off against each other. Examples exist of multi-material coatings based on demonstrated material properties that could meet the requirements of ET-LF; the method is reviewed in \cite{Craig2018ETmultimaterialDCC}. While the additional complexity of depositing such multi-material coatings could pose a fabrication challenge, especially if the optimal deposition method differs for the different materials involved, it remains an important option if simpler material combinations fail to meet requirements.

\section{Crystalline Coatings}

Epitaxially grown coatings consisting of alternating layers of high and low index layers of crystalline materials present an alternative method that avoids the mechanical loss issues associated with TLS in amorphous materials. The best developed of these materials are AlGaAs/GaAs mirrors grown by molecular beam epitaxy (MBE) on GaAs substrates and then transferred by wafer-bonding methods to mirror substrates such as silica, silicon or sapphire \cite{cole2013tenfold}. The mechanical and optical losses measured in small cavities with mirrors made in this fashion appear adequate for 2.5G and 3G detectors. As compressional mechanical losses are considerably higher than shear losses, it needs to be clarified whether thermal noise is as low for larger beam diameters, where compressional losses become more important.

The issues with respect to applying these as general solutions for gravitational wave interferometry are related to scaling. One scaling issue is associated with developing the infrastructure necessary for fabrication of $\mathrm{\sim}$\,40\,cm (or larger) diameter optics: growth of the necessary GaAs crystal substrates, and development of MBE and wafer-bonding tools meeting the optical homogeneity requirements. An order of magnitude estimate of the cost of that effort is $\mathrm{\sim}$\,\$\,25\,M \cite{Cole2018AlGaAsCost}. The other issue is related to MBE growth, in which an areal density of discrete defects generally appears. While these defects can be avoided in experiments with small beams by alignment in small cavities, this would not be the case with beam sizes characteristic of full-scale detector configurations. The optical scatter, absorption, and elastic loss associated with these defects are not yet well characterized, nor is their anticipated density known. Samples with several inch diameters are currently being characterized to better understand these issues, which presumably should be clarified before the investment in scaling the tooling is seriously considered. The effects of the electro-optic and piezoelectric sensitivities of these crystalline materials must also be investigated. 

A less well-developed approach is the use of GaP/AlGaP crystalline coatings \cite{lin2015epitaxial}. These have the advantage of being able to be grown directly onto silicon, removing the need for transferring the coating between substrates. Initial measurements of the cryogenic mechanical loss of these coatings appear very promising for use in cryogenic gravitational wave detectors. However, significant work on refining deposition parameters and reducing optical absorption is likely to be required.

\section{Current Research Programs}

There are several current and planned programs devoted to developing low-thermal-noise mirror coatings suitable for enhanced 2G, 2.5G, and 3G detectors. Participants are involved in all aspects of coating research, including various deposition methods, characterization of macroscopic properties at room and cryogenic temperatures, and atomic structure modeling and characterization. The recent incorporation into the project of several groups involved in coating deposition is particularly important, as the costs and time delays associated with commercial deposition of research coatings have been significant impediments to rapid progress. The efforts in the LSC and Virgo are presently somewhat independent although with good information exchange. 

\subsection{LSC}

There are approximately ten U.S. university research groups participating in various aspects of coating research. In late 2017, a more coordinated effort and additional funding for these groups were initiated under the Center for Coatings Research (CCR), jointly funded by the Gordon and Betty Moore Foundation and the NSF. Work in the U.S. also importantly includes that in the LIGO Laboratory. These efforts are complemented by groups not formally affiliated with the CCR, notably that by the large optical coating group at U. Montreal. Other major LSC coatings research programs are those in GEO (U. Glasgow, U. Strathclyde, U. West of Scotland, U. Hamburg, U. Hannover). The efforts of these groups are coordinated through biweekly telecons of the LSC Optics Working Group.

\subsection{VIRGO}

In Europe about ten universities are involved in various aspects of the VIRGO Coatings R\&D Project, including new material research, metrology and more recently simulation. The ViSIONs project, supporting six French laboratories, including the LMA and ILM, is focused on studying the relation between the physical properties of sputtered or evaporated materials and the structural and macroscopic properties of the deposited films. LMA is the only facility currently operational that is capable of producing coatings of the size and optical quality required for gravitational-wave interferometers. LMA has already started improving the uniformity of coating deposition in order to meet the challenges of the Advanced+ detectors. The AdVirgo+ project is considering the use of end cavity mirrors of 55 cm diameter. Therefore LMA has developed plans to upgrade their coaters and tools to deal with such increase of diameter and weight.

