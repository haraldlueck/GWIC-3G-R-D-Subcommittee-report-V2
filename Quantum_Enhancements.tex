%\chapterimage{Figures/speedmeterzoo_filt.png} % copy right: Stefan Danilishin
\chapterimage{Figures/squeezer.png} %copyright: Harald Lueck
% Chapter heading image
\chapter{Quantum Enhancements}
%\section{Quantum Enhancements}
\label{sec:Quantum}

%\section{Introduction} 
Quantum noise, originating from the quantized nature of light, is one of the key noise sources, limiting the performance of laser-interferometric gravitational wave detectors over a large part of the observational spectrum from the lower end up to the highest frequencies of observation. In contrast to many other fundamental noise sources, quantum noise can be strongly influenced and shaped by the actual interferometer configuration, e.g. type of interferometer, number of optical components and their masses, mirror reflectivities and optical losses, and use of additional optical cavities. Advanced interferometer topologies for quantum noise reduction beyond the current state of the art have been proposed. Such configurations have the potential to significantly improve the quantum noise in a specific frequency region of interest (e.g. at the low-frequency end for more accurate source parameter extraction or the potential of early warning for signals with an expected counterpart; at the high frequency end for improved neutron star physics) or even to provide a broadband improvement across the full observational spectrum. 

All quantum noise reduction schemes require additional space and flexibility in terms of infrastructure to not impose undesired limits on long term operation and upgrades beyond the initial implementation of 3G detectors. Broadly speaking, all advanced quantum noise reduction schemes rely on either the addition of long (up to km-scale) optical cavities (with potentially strongly reduced length noise requirements compared to the main interferometer), and/or major modifications to the optical configuration of the output port, i.e. the optical path between the main interferometer and main photodetectors used for reconstruction of the GW signal. 

Quantum noise reduction results from changes to the whole interferometer, that intrinsically links most subsystems of the observatory. The associated research and development therefore requires considerable effort, initially with detailed modelling, followed by extensive experimental testing of the complete interferometer configurations in suitable environments, covering the entire range from small-scale tabletop experiments to low-noise prototype interferometers. A worldwide coordination of the required R\&D activities will allow an efficient execution of the required research programmes and will also support the procurement of the required R\&D funds. 
 
\section{State of the art}
The impact of the quantum noise of the light field can be shaped and mitigated by smart optical schemes and interferometer designs. We distinguish between two quantum noise components: shot noise and radiation-pressure noise, which originate from the quantum fluctuations of light and its interaction with the test masses. Shot noise is inversely proportional to the optical power inside the arm cavity, and is dominant at higher frequencies (above 100 Hz in the case of Advanced LIGO); radiation-pressure noise is proportional to the optical power, and it is dominant at low frequencies. Radiation-pressure noise is sometimes called quantum `back-action' noise. Due to the Heisenberg Uncertainty Principle, the trade-off between these two, when varying the optical power, leads to the so-called Standard Quantum Limit (SQL). Developing quantum techniques for reducing quantum noise and surpassing the SQL is a very active field of research in the GW community. 

At high frequencies, the shot noise can be reduced by increasing the circulating laser power, by using squeezed light, or by methods which increase the signal response of the instrument in the frequency range of interest. High laser power poses a number of technical challenges, not least the thermal distortions of the test masses and optics which can lead to additional optical loss. \greencomment{Dave: mention parametric instabilities} Squeezing is anticipated to be applied for all future GW observatories, either as the main scheme to reduce quantum noise or in combination with more advanced interferometer schemes. Increasing the signal response at high frequencies can be achieved by enhancing the detector response in a certain frequency range, but usually this involves a trade-off leading to worse sensitivity outside this targeted band.  

At low frequencies, the radiation-pressure noise can be addressed by various schemes adopting aspects of quantum-non-demolition techniques. Theoretical research into radiation pressure noise reduction is very active and has produced a wide range of schemes that show a potential for significant sensitivity improvement. At the same time table-top experiments on quantum limited systems have become accessible to a wider community. However, the experimental investigations of possible schemes lag behind the theoretical work. We expect the range of interesting options for quantum noise reduction to shrink over the next years based on result from experimental research at small experiments and prototype interferometers.

\section{Requirements for 3G}
Dual-recycled Fabry-Perot Michelson interferometers combined with squeezed light injection are the state-of-the-art for current GW detectors and serve as a robust, low-risk baseline design for 3G observatories.  More advanced interferometer schemes, less researched so far, may offer potentially higher quantum noise suppression factors, but they require further R\&D to be fully developed. 

The following list details some key requirements for 3G detectors:

\begin{itemize}
    \item High circulating light power is an essential ingredient in many QN reduction schemes, for example to reduce shot noise, or to make use of correlation effects in opto-mechanical systems. Low-loss and stable operation of interferometers at high power is a key requirement for these schemes.
\item Stable and well controllable squeezed light sources, usually combined with long (100s of meters to km-scale) filter cavities, implemented to give a quantum noise reduction of more than 10dB.
\item Efficient implementation of QND schemes and squeezed-light technology relies on a couple of key technologies. This includes the development of low-loss optical components such as Faraday isolators, and reduction of optical scattering due to improved mirror surface figure errors. Another important contribution to the total optical loss is the matching of light fields to different optical resonators in the system. Leveraging adaptive optics for 3G detectors will contribute to a partial solution of the mode-matching problem. 
\item High-quantum-efficiency photodiodes (>98\%) at the operating light source wavelengths.
\item Heavy test masses, possibly up to 500\,kg, to suppress the effect of radiation pressure noise.
\item Adequate low-noise control systems for the main interferometer as well as auxiliary systems like filter cavities and readout system, compatible with astrophysical sensitivity in the sub-10Hz band.   
\item It is well know that the SQL is not a fundamental limit for the quantum limited sensitivity of interferometric detectors. Instead the limit will always be set by the achievable optical losses in the system. Future R\&D will be able to quantify how the loss distribution and different optical layouts set this limit.
\end{itemize}

\section{Benefits for 2G} 
The fundamental shape of 3G and 2G detectors will be based on long baseline L-shaped interferometers with additional cavities. It is expected that some quantum noise reduction schemes developed for 3G can be installed in current facilities and should be considered for further upgrades to 2G observatories. However, only 3G facilities will be able to fully support the potential of advanced quantum noise reduction schemes, and quantum noise R\&D for 3G is not a subset of ongoing efforts but demands a new dedicated approach, requiring funding and person power. It should be noted that future-looking R\&D for 3G observatories has the potential to attract additional early career scientists who will be able to also support the needs of the detectors today (as set out by collaboration agreements).

\section{Recommendations}
\greencomment{Dave: This is nice, but can probably be deleted for space.}
\magentacomment{hal: partial implementation in 2G can be regarded as a risk mitigation means}
A significant R\&D effort is required to make an informed selection of the optimal quantum noise reduction strategies for 3G detectors and to enable their risk free application to maximise the science return of 3G observatories. In order to enable these goals we recommend:
\begin{itemize}
\item Development of unified classification scheme of QND techniques and common approach to analysing and comparing their performances.  
\item National funding agencies to increase funding (staff and instrumentation) of the relevant lab activities and to enable the required long-term research programmes at the relevant prototype interferometers.
\greencomment{Dave: Certainly.  But this is a generic recommendation in the sense that all R\&D 3G efforts need more R\&D funding.  I don't think this needs to be here (or in any other chapter), but somewhere in the main set of recommendations.}
\item Existing GW collaborations to take ownership of 3G R\&D and integrate it into their programs and deliverables, in order to ensure sufficient support for the long-term future of the field (as it has been done so successfully for 2G R\&D during the operation of the initial detectors). \greencomment{Dave: ditto}
\item Establishment of worldwide discussion and coordination forum (meeting several times a year) across all relevant collaborations, in order to stipulate exchange of ideas and allowing (if deemed useful) for a coordination of the person-power intensive experimental activities.
\greencomment{Dave: ditto}
\end{itemize}



%\subsection{Quantum Noise Reduction Techniques under consideration}
%Quick overview of the most prominent quantum noise reduction techniques currently pursued. Maybe this could be done in a separate explanation box rather 
%than in the main text of the document?
%\subsection{Current State of the Art}
%\begin{itemize}
%\item Dual-recycled, Fabry-Perot Michelson
%\item Injection of squeezed light, obtaining about 4dB max
%\item A+ and Advanced Virgo+ will test frequency dependent squeezing with short (300\,m) filter cavities, targeting 6dB effective squeezing.
%\item Balanced homodyne readout will be part of A+, allowing in principle to change readout quadrature.
%\item Quick list of current losses for 1064nm: QE PDs, Faraday isolators etc.
%\end{itemize}
%\subsection{Requirements}
%Need to start here with a quick discussion on the relation of quantum noise 
%and laser wavelength: To first order quantum noise reduction schemes are independent of wavelength so all the drivers for changing towards longer wavelength come from other noise sources.Hence in the following assume that each of the discussed quantum noise reduction techniques can be realised equally well for any laser wavelength. However, we need to keep in mind that there could  be potential showstoppers, e.g. if it would turn out that there are no high quantum efficiency photo diodes available at 2 micron, which could lead for quantum noise considerations to rule out certain wavelengths.  
%\subsubsection{3G initial}
%\begin{itemize}
%\item \textbf{Result driven: 10\,dB effective quantum noise reduction over 2G for frequencies at mid and high frequency, plus pushing the radiation pressure noise below a few Hertz.}
%\item Current baselines: ET and CE designs are currently based on tuned (ET-HF and CE) or detuned (ET-LF) dual-recycled Fabry Perot Michelson interferometers, with frequency dependent squeezing created with km-scale filter cavities, combined with heavy test masses of the order 200\,kg to reduce radiation pressure noise at low frequencies. 
%\item Alternative approaches: Several alternative approaches are currently researched which provide more cost effective ways to improve the same sensitivity as the baselines mentioned above. Examples include conditional squeezing which uses an EPR measurement for generation of frequency dependent squeezed light and therefore removes the need for long baseline filter cavities; speedmeter interferometers which inherently suppress back action noise.     
%\end{itemize}
%\subsubsection{future}
%No limit. The more reduction the merrier. (Obviously need some better words for this section. Might need to discuss within full committee what level of speculation/optimism one should put in here to make it consistent with the other sections?)
%\subsection{Pathways and required facilities}
%\begin{itemize}
%\item Bench mark parameter set, for evaluation of different quantum noise reduction schemes. 
%\item Dedicated prototype test programmes for full interferometer schemes and low noise tests.
%\item Table top tests of interferometer building blocks, such as low loss optics, high QE photo diodes. 
%\end{itemize}
%\subsection{Type of collaboration required:  small/large}
%\begin{itemize}
%\item From feedback we received from community, the key that is needed is not 
%necessarily more collaboration (though this si also welcome), but rather more coordination, to make sure there are prototype experiments testing all the different promising configurations and we do not miss to develop one. 
%\end{itemize}
%\subsection{Suggested mechanisms}
%\begin{itemize}
%\item Community feedback asked for annual quantum noise workshop with special focus to bring together theorists and experimentalists. 
%\item Prototype coordination (partly happening already via LSC MOUs). Which body would take this on to cover more than just LSC?
%\end{itemize}
%\subsection{Impact/relation to 2G and upgrades}
%\begin{itemize}
%\item 2G consolidates state of the art and test baseline scheme, i.e. frequency
%dependent squeezing. 
%\item Benefit for 3G programme for 2G: Potentially improved sensitivity and faster commissioning plus risk reduction.
%\end{itemize}